\documentclass{article}
\usepackage[utf8]{inputenc}
\usepackage{mathrsfs}
\usepackage[spanish,es-lcroman]{babel}
\usepackage{amsthm}
\usepackage{amssymb}
\usepackage{enumitem}
\usepackage{graphicx}
\usepackage{caption}
\usepackage{float}
\usepackage{amsmath,stackengine,scalerel,mathtools}
\usepackage{xparse, tikz-cd, pgfplots}
\usepackage{comment}
\usepackage{faktor}
\usepackage[all]{xy}


\def\subnormeq{\mathrel{\scalerel*{\trianglelefteq}{A}}}
\newcommand{\Z}{\mathbb{Z}}
\newcommand{\La}{\mathscr{L}}
\newcommand{\crdnlty}[1]{
	\left|#1\right|
}
\newcommand{\lrprth}[1]{
	\left(#1\right)
}
\newcommand{\lrbrack}[1]{
	\left\{#1\right\}
}
\newcommand{\descset}[3]{
	\left\{#1\in#2\ \vline\ #3\right\}
}
\newcommand{\descapp}[6]{
	#1: #2 &\rightarrow #3\\
	#4 &\mapsto #5#6 
}
\newcommand{\arbtfam}[3]{
	{\left\{{#1}_{#2}\right\}}_{#2\in #3}
}
\newcommand{\arbtfmnsub}[3]{
	{\left\{{#1}\right\}}_{#2\in #3}
}
\newcommand{\fntfmnsub}[3]{
	{\left\{{#1}\right\}}_{#2=1}^{#3}
}
\newcommand{\fntfam}[3]{
	{\left\{{#1}_{#2}\right\}}_{#2=1}^{#3}
}
\newcommand{\fntfamsup}[4]{
	\lrbrack{{#1}^{#2}}_{#3=1}^{#4}
}
\newcommand{\arbtuple}[3]{
	{\left({#1}_{#2}\right)}_{#2\in #3}
}
\newcommand{\fntuple}[3]{
	{\left({#1}_{#2}\right)}_{#2=1}^{#3}
}
\newcommand{\gengroup}[1]{
	\left< #1\right>
}
\newcommand{\stblzer}[2]{
	St_{#1}\lrprth{#2}
}
\newcommand{\cmmttr}[1]{
	\left[#1,#1\right]
}
\newcommand{\grpindx}[2]{
	\left[#1:#2\right]
}
\newcommand{\syl}[2]{
	Syl_{#1}\lrprth{#2}
}
\newcommand{\grtcd}[2]{
	mcd\lrprth{#1,#2}
}
\newcommand{\lsttcm}[2]{
	mcm\lrprth{#1,#2}
}
\newcommand{\amntpSyl}[2]{
	\mu_{#1}\lrprth{#2}
}
\newcommand{\gen}[1]{
	gen\lrprth{#1}
}
\newcommand{\ringcenter}[1]{
	C\lrprth{#1}
}
\newcommand{\zend}[2]{
	End_{\mathbb{Z}}^{#2}\lrprth{#1}
}
\newcommand{\genmod}[2]{
	\left< #1\right>_{#2}
}
\newcommand{\genlin}[1]{
	\mathscr{L}\lrprth{#1}
}
\newcommand{\opst}[1]{
	{#1}^{op}
}
\newcommand{\ringmod}[3]{
	\if#3l
	{}_{#1}#2
	\else
	\if#3r
	#2_{#1}
	\fi
	\fi
}
\newcommand{\ringbimod}[4]{
	\if#4l
	{}_{#1-#2}#3
	\else
	\if#4r
	#3_{#1-#2}
	\else 
	\ifstrequal{#4}{lr}{
		{}_{#1}#3_{#2}
	}
	\fi
	\fi
}
\newcommand{\ringmodhom}[3]{
	Hom_{#1}\lrprth{#2,#3}
}

\ExplSyntaxOn

\NewDocumentCommand{\functor}{O{}m}
{
	\group_begin:
	\keys_set:nn {nicolas/functor}{#2}
	\nicolas_functor:n {#1}
	\group_end:
}

\keys_define:nn {nicolas/functor}
{
	name     .tl_set:N = \l_nicolas_functor_name_tl,
	dom   .tl_set:N = \l_nicolas_functor_dom_tl,
	codom .tl_set:N = \l_nicolas_functor_codom_tl,
	arrow      .tl_set:N = \l_nicolas_functor_arrow_tl,
	source   .tl_set:N = \l_nicolas_functor_source_tl,
	target   .tl_set:N = \l_nicolas_functor_target_tl,
	Farrow      .tl_set:N = \l_nicolas_functor_Farrow_tl,
	Fsource   .tl_set:N = \l_nicolas_functor_Fsource_tl,
	Ftarget   .tl_set:N = \l_nicolas_functor_Ftarget_tl,	
	delimiter .tl_set:N= \_nicolas_functor_delimiter_tl,	
}

\dim_new:N \g_nicolas_functor_space_dim

\cs_new:Nn \nicolas_functor:n
{
	\begin{tikzcd}[ampersand~replacement=\&,#1]
		\dim_gset:Nn \g_nicolas_functor_space_dim {\pgfmatrixrowsep}		
		\l_nicolas_functor_dom_tl
		\arrow[r,"\l_nicolas_functor_name_tl"] \&
		\l_nicolas_functor_codom_tl
		\\[\dim_eval:n {1ex-\g_nicolas_functor_space_dim}]
		\l_nicolas_functor_source_tl
		\xrightarrow{\l_nicolas_functor_arrow_tl}
		\l_nicolas_functor_target_tl
		\arrow[r,mapsto] \&
		\l_nicolas_functor_Fsource_tl
		\xrightarrow{\l_nicolas_functor_Farrow_tl}
		\l_nicolas_functor_Ftarget_tl
		\_nicolas_functor_delimiter_tl
	\end{tikzcd}
}
\ExplSyntaxOff

\theoremstyle{definition}
\newtheorem{define}{Definición}
\newtheorem{lem}{Lema}
\newtheorem{teo}{Teorema}
\title{Lista 4}
\author{Arruti, Sergio, Jesús}
\date{}


\begin{document}
	\maketitle
	\begin{enumerate}[label=\textbf{Ej \arabic*.}]
		\setcounter{enumi}{47}
\item 
\item
\item Sean $f\colon B\rightarrow C$ en $Mod(R)$ y $g\colon B\longrightarrow B$ tales que $fg=f$. Pruebe que:\\
$g\colon f\longrightarrow f$ es un isomorfismo en \,\,$\faktor{Mod(R)}{C}$ \,\,si y sólo si $g\colon B\longrightarrow B$ es un isomorismo en $Mod(R)$.
\begin{proof}

$g\colon f\longrightarrow f$ es un isomorfismo en \,\,$\faktor{Mod(R)}{C}$\,\,\\
$\iff \,\,\,\exists g^{-1}\colon f\longrightarrow f$  tal que $g^{-1}g=1_f$ y $gg^{-1}=1_f$\\
$\iff \,\,\,\exists g^{-1}\colon f\longrightarrow f$  tal que $g^{-1}g=Id_B$ y $gg^{-1}=Id_B$\\
$\iff \,\,\,\exists g^{-1}\in \operatorname{Hom}_R(B,B)$ tal que $g^{-1}g=Id_B$ y $gg^{-1}=Id_B$\\
$\iff \,\,\,g\colon B\longrightarrow B$ es isomorfismo en $Mod(R)$.
\end{proof}

\item
\item
\item Sean 
\begin{tikzcd}
 M \arrow{r}{f} & N \arrow{r}{f'} & M
\end{tikzcd}
en $Mod(R)$ tal que $ff'=1_N.$ Pruebe que $M=Ker(f)\oplus Im(f')$.
\begin{proof}
Como $ff'=1_N$, entonces $Ker(f')=0$ y $Im(f)=N$, es decir, $f'$ es monomorfismo, $f$ es epimorfismo y  $Im(f')+Ker(f)\leq M$.\\
Si $x\in Im(f')\cap Ker(f)$ entonces existe $y\in N$ tal que $f'(y)=x$ y además $f(x)=0$ entonces $0=f(x)=ff'(y)=1_N(y)$
por lo que $x=0$ y así $Im(f´)\cap Ker(f)=0$.\\

Si $x\in M$ entoces $f(x-f'f(x))=f(x)-f(x)=0$, \quad y \\
$x=x+(x-f'f(x))+f'f(x)\in Ker (f)+Im(f').$
\end{proof}

\item
\item
\item Sean $f\colon A\longrightarrow B$ y $g\colon B\longrightarrow B$ en $Mod(R)$ tal que $gf=f$. Prueba que\\

\begin{tikzcd}
g\colon f \arrow{r}{\sim} & f
\end{tikzcd}
en $Mod(R)\backslash A$\quad $\iff$ \quad 
\begin{tikzcd}
g\colon B \arrow{r}{\sim} & B
\end{tikzcd}
en $Mod(R)$.

\begin{proof}
\begin{tikzcd}
g\colon f \arrow{r}{\sim} & f
\end{tikzcd}
en $Mod(R)\backslash A$\\
 $\iff \exists g^{-1}\colon f\longrightarrow f$ tal que $g^{-1}g=1_f$ y $gg^{-1}=1_f$\\
 $\iff \exists g^{-1}\colon f\longrightarrow f$ tal que $g^{-1}g=Id_B$ y $gg^{-1}=Id_B$\\
 $\iff \exists g^{-1}\in \operatorname{Hom}_R(B,B)$ tal que $g^{-1}g=Id_B$ y $gg^{-1}=Id_B$\\
$\iff$ \quad 
\begin{tikzcd}
g\colon B \arrow{r}{} & B
\end{tikzcd}
es isomorfismo en $Mod(R)$.
\end{proof}
\item
\item
\item Sean $F\colon A\longrightarrow B$ un funtor contravariante aditivo entre categorías preaditivas. Pruebe que si $F$ es fiel y pleno, entonces\\
$F\colon End_\mathcal{A}(A) \longrightarrow End_\mathcal{B} (F(A))^{op}$ es isomorismo de anillos.
\begin{proof}
Como $F$ es funtor contravariante aditivo, entonces es un morfismo de grupos abelianos. Considerando la composición, tenemos que 
$End_ \mathcal{A}(A)$ y $End_\mathcal{B}(F(A))$ son anillos, así $End_\mathcal{A}(F(A))^{op}$ es anillo.\\

Por definición de funtor contravariate para cada $f,g\in End_ \mathcal{A}(A)$ se tiene que 
\[F(f\circ g)=F(g)\circ F(f)\quad \text{y}\quad F(1_A)=1_{F(A)}\quad \forall A\in Obj(\mathcal{A}).
\]
Entonces $F$ es morfismo de anillos entre $End_\mathcal{A}(A)$ y $End_\mathcal{B}(F(A))^{op}$.
\end{proof}

\item
\item \,\\ \,

\textbf{Lemma}*\\
(Andeson, Fuller) 16.6\\
El funtor $\operatorname{Hom}_R(M,Y)$ es exacto izquierdo. En particular si $U$ es un $R$-módulo, entonces para cada sucesión exacta 
\begin{tikzcd}
0 \arrow{r}{} &K \arrow{r}{f} & M \arrow{r}{g} & N \arrow{r}{} &0
\end{tikzcd}
en $Mod(R)$ las sucesiones 
\begin{tikzcd}\displaystyle
0 \arrow{r}{} &\operatorname{Hom}_R(U,K) \arrow{r}{f_*} & 
\operatorname{Hom}_R(U,M) \arrow{r}{g_*} & \operatorname{Hom}_R(U,N) \arrow{r}{} &0
\end{tikzcd}\\
y \,\,
\begin{tikzcd}\displaystyle
0 \arrow{r}{} &\operatorname{Hom}_R(M,X) \arrow{r}{g^*} & 
\operatorname{Hom}_R(M,Y) \arrow{r}{f^*} & \operatorname{Hom}_R(M,Z) \arrow{r}{} &0
\end{tikzcd}\\
son exactas\\

\item Para $M\in Mod(R)$ pruebe que las siguientes condiciones son equivalentes: 
\begin{itemize}
\item[a)] $M$ es proyectivo.
\item[b)] Toda sucesión exacta 
\begin{tikzcd}
0 \arrow{r}{} &X \arrow{r}{} & Y \arrow{r}{} & M \arrow{r}{} &0
\end{tikzcd}
en $Mod(R)$ se escinde.
\item[c)] $M$ es isomorfo a un sumando directo de $R$-módulos libre.
\item[d)]Para toda sucesión exacta 
\begin{tikzcd}
0 \arrow{r}{} &X \arrow{r}{f} & Y \arrow{r}{g} & Z \arrow{r}{} &0
\end{tikzcd}
 en $Mod(R)$, se tiene que \\
\begin{tikzcd}\displaystyle
0 \arrow{r}{} &\operatorname{Hom}_R(M,X) \arrow{r}{f_*} & 
\operatorname{Hom}_R(M,Y) \arrow{r}{g_*} & \operatorname{Hom}_R(M,Z) \arrow{r}{} &0
\end{tikzcd}\\
es exacta en $Mod(\Z)$, donde $f_*=\operatorname{Hom}_R(M,f)$\quad y \quad $g_*=\operatorname{Hom}_R(M,g)$.
\end{itemize}
\begin{proof}
\boxed{a)\Rightarrow b)}\\
Sea $M$ proyectivo y 
\begin{tikzcd}
0 \arrow{r}{} &X \arrow{r}{} & Y \arrow{r}{} & M \arrow{r}{} &0
\end{tikzcd}
una sucesión exacta en $Mod(R)$. Como $Y \stackrel{h}{\longrightarrow} M$ es epi, entonces el morfismo $I_M\colon M\longrightarrow M$
se puede factorizar a través de $h$, es decir, existe $g\colon M\longrightarrow Y$ tal que $Id_M=hg$. Por lo tanto $h$ es split-epi y por el 
ejercicio 54 la sucesión se escinde.
\boxed{b)\Rightarrow c)} \\
Sea $\displaystyle F=\bigoplus_{y\in M}Ry$ el módulo libre generado por los elementos de $M$, entonces existe un epimorfismo 
$g\colon F\longrightarrow M$, por lo que \\
\begin{tikzcd}
0 \arrow{r}{} &Ker(g) \arrow{r}{i} & F \arrow{r}{g} & M \arrow{r}{} &0
\end{tikzcd}
es exacta con $i$ la inclusión.\\
Por hipótesis esta sucesión exacta se escinde, por lo tanto $M\oplus Ker(g)=F$, es decir, $M$ es un sumando directo de un módulo libre.\\
\boxed{c)\Rightarrow a)}\\
Supongamos que tenemos el siguiente diagrama con $g$ epi:\\
\xymatrix {
	X\ar[r]^{g}& Y\ar[r]& 0\\
	                 & M\ar[u]^{h}      
}
\\
Por c) sabemos que existe $F,K$ módulos tales que $M\oplus K=F$ con $F$ un móduulo libre. Ahora,  como todo módulo libre es proyectivo y 
considerando a $\pi\colon F\longrightarrow M$, se tiene que existe $f\colon F\longrightarrow X$ tal que $h\pi=fg$, así
$h\pi i=gfi$ con $i$ la inclusión de $M$ en $F$, por lo que $h=g\circ f_0$ con $f_0\colon M\longrightarrow X$.\\
\boxed{a)\iff d)}\\
Por el lema* la condición d) se cumple si y sólo si por cada epimorfismo 
\begin{tikzcd}
Y \arrow{r}{f} &Z \arrow{r}{g} & 0
\end{tikzcd}
la sucesión 
\begin{tikzcd}
\operatorname{Hom}_R(M,Y) \arrow{r}{f_*} &\operatorname{Hom}_R(M,Z) \arrow{r}{} & 0
\end{tikzcd}
es exacta. Pero $f_*$ es epi si y sólo si por cada $\gamma\in \operatorname{Hom}_R(M,Z)$ existe un $\hat{\gamma}\in \operatorname{Hom}_R(U,M)$
tal que $\gamma=f_*(\hat{\gamma})=f\hat{\gamma}.$

\end{proof}
\item
\item
\item Para $M\in Mod(R)$, pruebe que las siguientes condiciones son equivalentes.
\begin{itemize}
\item[a)] $M$ es inyectivo.
\item[b)] Toda sucesión exacta 
\begin{tikzcd}
0 \arrow{r}{} &M \arrow{r}{} & X \arrow{r}{} & Y \arrow{r}{} &0
\end{tikzcd}
en $Mod(R)$ se escinde.
\item[c)] Para toda sucesión exacta 
\begin{tikzcd}
0 \arrow{r}{} &X \arrow{r}{f} & Y \arrow{r}{g} & Z \arrow{r}{} &0
\end{tikzcd}
 en $Mod(R)$, se tiene que \\
\begin{tikzcd}\displaystyle
0 \arrow{r}{} &\operatorname{Hom}_R(Z,M) \arrow{r}{g^*} & 
\operatorname{Hom}_R(Y,M) \arrow{r}{f^*} & \operatorname{Hom}_R(X,M) \arrow{r}{} &0
\end{tikzcd}\\
es exacta en $Mod(\Z)$, donde $f^*=\operatorname{Hom}_R(f,M)$\quad y \quad $g^*=\operatorname{Hom}_R(g,M)$.
\end{itemize}
\begin{proof}
\boxed{a)\Rightarrow b)} \\
Sea \begin{tikzcd}
0 \arrow{r}{} &M \arrow{r}{f} & X \arrow{r}{} & Y \arrow{r}{} &0
\end{tikzcd}
exacta en $Mod(R)$. Como $f$ es mono, entonces, considerando $I_M\colon M\longrightarrow M$, tenemos que existe $h\colon M\longrightarrow Y$
tal que $I_M$ se factoriza de $f$, es decir, $I_M=hf$ por lo tanto 
\begin{tikzcd}
0 \arrow{r}{} &M \arrow{r}{f} & X \arrow{r}{} & Y \arrow{r}{} &0
\end{tikzcd}
es split-mono y por el ejercicio 54 se escinde.\\
\boxed{b)\Rightarrow a)} \\
Sean $X,Y$ $R$-módulos y $f\colon X\longrightarrow Y$ mono. Si $h\in \operatorname{Hom}_R(X,Y)$ tenemos el siguiente diagrama\quad
\xymatrix {
	0\ar[r] & X\ar[r]^{f}& Y\\
	           & M\ar[u]^{h}
}\\
que se extiende a un pushout \quad\quad
\xymatrix {
	0\ar[r] & X\ar[r]^{f}\ar[d]^{h}& Y\ar[d]^{h'}\\
	           & M\ar[r]_{f'} & D
}\\
donde $D=(\faktor{X\oplus Y}{W}),\,\, W=\{(fa-ga):a\in R\}, \quad h'(b)=(0,b)+W$ y $g'(c)=(c,0)+W$.\\

Así $f'$ es mono. Por hipótesis existe un morfismo $\beta\colon D\longrightarrow M$ con $\beta f'=1_M$. Definamos $g=\beta  h'$ entonces 
$g\colon Y\longrightarrow M$ y $gf=\beta  h'f=\beta f'h=h$, por lo que $M$ es inyectivo.\\
\boxed{a)\iff c)}\\
Como $\operatorname{Hom}_R(\cdot,M)$ es contravariante exacto izquierdo, es suficiente mostrar que $M$ es inyectivo si y sólo si
$\operatorname{Hom}_R(\cdot,M)$ convierte monomorfismos en epimorfismos:\\
Si $\alpha\colon A\longrightarrow B$ es mono, entonces $\alpha^*\colon\operatorname{Hom}_R(B,M)\longrightarrow \operatorname{Hom}_R(A,M)$
es epi si y sólo si para cada $f\in \operatorname{Hom}_R(A,M)$ existe $g\in \operatorname{Hom}_R(B,M)$ tal que $\alpha^*(g)=f$, y esto pasa si y sólo si para cada $f\in \operatorname{Hom}_R(A,M)$
 existe $g\in \operatorname{Hom}_R(B,M)$ tal que $g\alpha=f$, es decir, $M$ es inyectivo.
\end{proof}






\end{enumerate}
\end{document}