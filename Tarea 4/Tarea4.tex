\documentclass{article}
\usepackage[utf8]{inputenc}
\usepackage{mathrsfs}
\usepackage[spanish,es-lcroman]{babel}
\usepackage{amsthm}
\usepackage{amssymb}
\usepackage{enumitem}
\usepackage{graphicx}
\usepackage{caption}
\usepackage{float}
\usepackage{eufrak}
\usepackage{nicefrac}
\usepackage{amsmath,stackengine,scalerel,mathtools}
\usepackage{tikz-cd}
\usepackage{comment}%Paquete para añadir comentarios largos.}


\def\subnormeq{\mathrel{\scalerel*{\trianglelefteq}{A}}}
\newcommand{\Z}{\mathbb{Z}}
\newcommand{\La}{\mathscr{L}}
\newcommand{\crdnlty}[1]{
	\left|#1\right|
}
\newcommand{\lrprth}[1]{
	\left(#1\right)
}
\newcommand{\lrbrack}[1]{
	\left\{#1\right\}
}
\newcommand{\descset}[3]{
	\left\{#1\in#2\ \vline\ #3\right\}
}
\newcommand{\descapp}[6]{
	#1: #2 &\rightarrow #3\\
	#4 &\mapsto #5#6 
}
\newcommand{\arbtfam}[3]{
	{\left\{{#1}_{#2}\right\}}_{#2\in #3}
}
\newcommand{\arbtfmnsub}[3]{
	{\left\{{#1}\right\}}_{#2\in #3}
}
\newcommand{\fntfmnsub}[3]{
	{\left\{{#1}\right\}}_{#2=1}^{#3}
}
\newcommand{\fntfam}[3]{
	{\left\{{#1}_{#2}\right\}}_{#2=1}^{#3}
}
\newcommand{\fntfamsup}[4]{
	\lrbrack{{#1}^{#2}}_{#3=1}^{#4}
}
\newcommand{\arbtuple}[3]{
	{\left({#1}_{#2}\right)}_{#2\in #3}
}
\newcommand{\fntuple}[3]{
	{\left({#1}_{#2}\right)}_{#2=1}^{#3}
}
\newcommand{\gengroup}[1]{
	\left< #1\right>
}
\newcommand{\stblzer}[2]{
	St_{#1}\lrprth{#2}
}
\newcommand{\cmmttr}[1]{
	\left[#1,#1\right]
}
\newcommand{\grpindx}[2]{
	\left[#1:#2\right]
}
\newcommand{\syl}[2]{
	Syl_{#1}\lrprth{#2}
}
\newcommand{\grtcd}[2]{
	mcd\lrprth{#1,#2}
}
\newcommand{\lsttcm}[2]{
	mcm\lrprth{#1,#2}
}
\newcommand{\amntpSyl}[2]{
	\mu_{#1}\lrprth{#2}
}
\newcommand{\gen}[1]{
	gen\lrprth{#1}
}
\newcommand{\ringcenter}[1]{
	C\lrprth{#1}
}
\newcommand{\zend}[2]{
	End_{\mathbb{Z}}^{#2}\lrprth{#1}
}
\newcommand{\genmod}[2]{
	\left< #1\right>_{#2}
}
\newcommand{\genlin}[1]{
	\mathscr{L}\lrprth{#1}
}
\newcommand{\opst}[1]{
	{#1}^{op}
}
\newcommand{\ringmod}[3]{
	\if#3l
	{}_{#1}#2
	\else
	\if#3r
	#2_{#1}
	\fi
	\fi
}
\newcommand{\ringbimod}[4]{
	\if#4l
	{}_{#1-#2}#3
	\else
	\if#4r
	#3_{#1-#2}
	\else 
	\ifstrequal{#4}{lr}{
		{}_{#1}#3_{#2}
	}
	\fi
	\fi
}
\newcommand{\ringmodhom}[3]{
	Hom_{#1}\lrprth{#2,#3}
}

\ExplSyntaxOn

\NewDocumentCommand{\functor}{O{}m}
{
	\group_begin:
	\keys_set:nn {nicolas/functor}{#2}
	\nicolas_functor:n {#1}
	\group_end:
}

\keys_define:nn {nicolas/functor}
{
	name     .tl_set:N = \l_nicolas_functor_name_tl,
	dom   .tl_set:N = \l_nicolas_functor_dom_tl,
	codom .tl_set:N = \l_nicolas_functor_codom_tl,
	arrow      .tl_set:N = \l_nicolas_functor_arrow_tl,
	source   .tl_set:N = \l_nicolas_functor_source_tl,
	target   .tl_set:N = \l_nicolas_functor_target_tl,
	Farrow      .tl_set:N = \l_nicolas_functor_Farrow_tl,
	Fsource   .tl_set:N = \l_nicolas_functor_Fsource_tl,
	Ftarget   .tl_set:N = \l_nicolas_functor_Ftarget_tl,	
	delimiter .tl_set:N= \_nicolas_functor_delimiter_tl,	
}

\dim_new:N \g_nicolas_functor_space_dim

\cs_new:Nn \nicolas_functor:n
{
	\begin{tikzcd}[ampersand~replacement=\&,#1]
		\dim_gset:Nn \g_nicolas_functor_space_dim {\pgfmatrixrowsep}		
		\l_nicolas_functor_dom_tl
		\arrow[r,"\l_nicolas_functor_name_tl"] \&
		\l_nicolas_functor_codom_tl
		\\[\dim_eval:n {1ex-\g_nicolas_functor_space_dim}]
		\l_nicolas_functor_source_tl
		\xrightarrow{\l_nicolas_functor_arrow_tl}
		\l_nicolas_functor_target_tl
		\arrow[r,mapsto] \&
		\l_nicolas_functor_Fsource_tl
		\xrightarrow{\l_nicolas_functor_Farrow_tl}
		\l_nicolas_functor_Ftarget_tl
		\_nicolas_functor_delimiter_tl
	\end{tikzcd}
}
\ExplSyntaxOff

\theoremstyle{definition}
\newtheorem{define}{Definición}
\newtheorem{lem}{Lema}
\newtheorem{teo}{Teorema}

\title{Lista 4}
\author{Arruti, Sergio, Jesús}
\date{}

\newcommand\equivalencia{\mathrel{\stackrel{\makebox[0pt]{\mbox{\normalfont\tiny \sim}}}{\longrightarrow}}}

\begin{document}
\maketitle

\begin{enumerate}[label=\textbf{Ej \arabic*.}]
	%%%%%%%%% Ej 49 %%%%%%%5
	\item Sean $M \in Mod\lrprth{R}$ y $X \subseteq M$. Considere el morfismo de $R$-módulos $\overline{\varepsilon}_{X,M}:F(X) \longrightarrow M$, dado por $\overline{\varepsilon}_{X,M}\lrprth{\arbtfam{t}{x}{X}}=\displaystyle\Sigma_{x \in X}t_{x}x$. Note que la composición
	\begin{tikzcd}
		X \arrow{r}{\varepsilon_{x}} & F(X) \arrow{r}{\overline{\varepsilon}_{X,M}} & M
	\end{tikzcd}
	coincide con la inclusión $X \subseteq M$. Pruebe que:
	\begin{enumerate}
		\item $im\lrprth{\overline{\varepsilon}_{X,M}}=\genmod{x}{R}$
		\item $M=\genmod{x}{R}\Leftrightarrow\overline{\varepsilon}_{X,M}$ es un epimorfismo.
		\item $X$ es $R$-linealmente independiente$\Leftrightarrow\overline{\varepsilon}_{X,M}$ es un monomorfismo.
		\item $X$ es una $R$-base$\Leftrightarrow\overline{\varepsilon}_{X,M}$ es un isomorfismo.
	\end{enumerate}
	\begin{proof}
		$\boxed{\text{a}}$ Primero, como $\genmod{x}{R}$ es un submódulo de $M$, se tiene que $im\lrprth{\overline{\varepsilon}_{X,M}}\subseteq\genmod{x}{R}$. Por otro lado, sea $m\in\genmod{x}{R}$. Entonces $m$ tiene una descomposición $m=\displaystyle\Sigma_{x \in X}t_{x}x$, donde $t_{x} \in F(X)$. En consecuencia, $\overline{\varepsilon}_{X,M}\lrprth{\arbtfam{t}{x}{X}}=\displaystyle\Sigma_{x \in X}t_{x}x=m$. $\therefore im\lrprth{\overline{\varepsilon}_{X,M}}=\genmod{x}{R}$\\
		
		$\boxed{\text{(b)}}$ Este inciso se deduce del anteior. $M=\genmod{x}{R}\Leftrightarrow M=im\lrprth{\overline{\varepsilon}_{X,M}}\Leftrightarrow\overline{\varepsilon}_{X,M}$ es un epimorfismo.\\
		
		$\boxed{\text{(c)}} \boxed{\Rightarrow )}$ Suponga que $\arbtfam{t}{x}{X} \in Ker\lrprth{\overline{\varepsilon}_{X,M}}$. De modo que $\displaystyle\Sigma_{x \in X}t_{x}x=\overline{\varepsilon}_{X,M}\lrprth{\arbtfam{t}{x}{X}}=0$. Dado que $X$ es $R$-linealmente independiente, para cada $x \in X$, $t_{x}=0$. Por tanto, $Ker\lrprth{\overline{\varepsilon}_{X,M}}=0$. $\therefore\overline{\varepsilon}_{X,M}$ es monomorfismo.\\
		
		$\boxed{\Leftarrow )}$ Sean $x_{1},...,x_{n} \in X$ y $r_{x_{1}},...r_{x_{n}} \in R$ tales que $\displaystyle\Sigma_{k=1}^{n}r_{x_{k}}x_{k}=0$. Completamos a un elemento de $F(X)$ como $r_{x}=0$, con $x\not\in\lrbrack{x_{1},...,x_{n}}$. Con lo cual tenemos que:
		\begin{align*}
			\overline{\varepsilon}_{X,M}\lrprth{\arbtfam{r}{x}{X}} &= \displaystyle\Sigma_{x \in X}r_{x}x\\
			&=\displaystyle\Sigma_{k=1}^{n}r_{x_{k}}x_{k}\\
			&=0
		\end{align*}
		Entonces $\lrbrack{r}{x}{X} \in Ker\lrprth{\overline{\varepsilon}_{X,M}} = 0$. Por tanto, $r_{x_{1}}=...=r_{x_{n}}=0$.\\
		$\therefore X$ es $R$-linealmente independiente.\\
		
		$\boxed{\text{(d)}}$ Este resultado se concluye de los anteriores. En efecto,
		\begin{align*}
			\overline{\varepsilon}_{X,M}\ es\ un\ isomorfismo & \Leftrightarrow\overline{\varepsilon}_{X,M}\ es\ un\ epimorfismo\ y\ monomorfismo\\
			& \Leftrightarrow M=im\lrprth{\overline{\varepsilon}_{X,M}}\ y\ X\ es\ R-l.i.\\
			& \Leftrightarrow X\ es\ una\ R-base.
		\end{align*}
	\end{proof}
	
	%%%%%%% Ej 52 %%%%%%
	\item Sean $\psi : B' \longrightarrow B$ un isomorfismo y $f:B \longrightarrow C$ es $Mod\lrprth{R}$. Pruebe que: Si $f$ es minimal a derecha, entonces $f\circ\psi :B' \longrightarrow C$ es minimal a derecha.
	\begin{proof}
		Sea $g:f\circ\psi \longrightarrow f\circ\psi$ un morfismo en $Mod\lrprth{R}/C$. Entonces, por el \textbf{ejercicio 50}., $g:B' \longrightarrow B'$ es un homomorfismo en $Mod\lrprth{R}$. Más aún, $\psi \circ g:B' \longrightarrow B$ también es un homomorfismo. Dado que $f$ es minimal a derecha, se tiene que $\psi \circ g$ es un isomorfismo en $Mod\lrprth{R}$. En virtud de que $\psi$ es un isomorfismo, $g:B' \longrightarrow B'$ es un isomorfismo en $Mod\lrprth{R}$. Aplicando el \textbf{ejercicio 50.}, se tiene que $g:f\circ\psi \longrightarrow f\circ\psi$ es un isomorfismo. $\therefore f\circ\psi$ es minimal a derecha.
	\end{proof}
	
	%%%%%% Ej 55 %%%%%%%
	\item Sea $\eta:$
	\begin{tikzcd}
		0 \arrow{r} & M_{1} \arrow{r}{f_{1}} & M \arrow{r}{g_{2}} & M_{2} \arrow{r} & 0
	\end{tikzcd}
	una sucesión en $Mod\lrprth{R}$. Pruebe que las siguientes condiciones son equivalentes
	\begin{enumerate}
		\item $\eta$ es una sucesión que se parte.
		\item Existe una sucesión
		\begin{tikzcd}
				0 \arrow{r} & M_{2} \arrow{r}{f_{2}} & M \arrow{r}{g_{1}} & M_{1} \arrow{r} & 0
		\end{tikzcd}
		en $Mod\lrprth{R}$ tal que $g_{1}f_{1}=1_{M_{1}}$, $g_{2}f_{2}=1_{M_{2}}$, $g_{2}f_{1}=g_{1}f_{2}=0$ y $g_{1}f_{1}+g_{2}f_{2}=1_{M}$.
		\item Existe un isomorfismo $h:M_{1} \times M_{2} \longrightarrow M$ tal que el siguiente diagrama conmuta\\
		\begin{tikzcd}
				0 \arrow{r} & M_{1} \arrow{r}{i_{1}}\arrow{d} & M_{1} \times M_{2} \arrow{r}{\pi_{2}}\arrow{d}{h} & M_{2} \arrow{r}\arrow{d} & 0\\
				0 \arrow{r} & M_{1} \arrow{r}{f_{1}} & M \arrow{r}{g_{2}} & M_{2} \arrow{r} & 0
		\end{tikzcd}
	\end{enumerate}
	\begin{proof}
		$\boxed{\text{a)}\Rightarrow\text{c)}}$ Dado que $\eta$ es una sucesión que se parte, existe un morfismo de $R$-módulos, $f_{2}:M_{2} \longrightarrow M$, tal que $g_{2}f_{2}=1_{M_{2}}$. Luego, $g_{2}$, $f_{2}$ inducen un isomorfismo $h:M_{1} \times M_{2} \longrightarrow M$. En efecto, definimos $h$ como el morfismo $h\lrprth{m_{1},m_{2}}=f_{1}\lrprth{m_{1}}+f_{2}\lrprth{m_{2}}$.\\
		
		En primer lugar, veremos que $h$ es un monomorfismo. En este sentido, sea $\lrprth{m_{1},m_{2}} \in Ker\lrprth{h}$, entonces $0=h\lrprth{m_{1},m_{2}}=f_{1}\lrprth{m_{1}}+f_{2}\lrprth{m_{2}}$. En consecuencia, $f_{2}\lrprth{m_{2}}=-f_{1}\lrprth{m_{1}}\in Im\lrprth{f_{1}}=Ker\lrprth{g_{2}}$. Así, 
		\begin{align*}
			m_{2}=g_{2}f_{2}\lrprth{m_{2}}=0
		\end{align*}
		Por consiguiente, $f_{1}\lrprth{m_{1}}=h\lrprth{m_{1},m_{2}}=0$. Dado que $f_{1}$ es mono, $m_{1}=0$. Por lo que $h$ es mono.\\
		
		Ahora, $h$ es epi. Sea $m \in M$. Entonces 
		\begin{align*}
			g_{2}\lrprth{m-f_{2}g_{2}\lrprth{m}}=g_{2}\lrprth{m}-g_{2}\lrprth{m}=0
		\end{align*}
		De esta forma, $m-f_{2}g_{2}\lrprth{m} \in Im\lrprth{f_{1}}$. Ésto aunado a la exactitud de $\eta$ garantiza la existencia de un elemento $x \in M_{1}$ tal que $f_{1}\lrprth{x}=m-f_{2}g_{2}\lrprth{m}$, con lo cual,
		\begin{align*}
			h\lrprth{x,g_{2}\lrprth{m}}&=f_{1}\lrprth{x}+f_{2}\lrprth{g_{2}\lrprth{m}}\\
			&=m
		\end{align*}
		
		Una vez demostrado que $h$ es un isomorfismo, podemos proceder a mostrar que el diagrama presentado anteriormente conmuta bajo este isomorfismo. Primero, note que para $m \in M_{1}$ se tiene que 
		\begin{align*}
			h\textit{i}_{1}\lrprth{m}&=h\lrprth{m,0}\\
			&=f_{1}\lrprth{m}\\
			&=f_{1}1_{M_{1}}\lrprth{m}
		\end{align*}
		Por el otro lado, dado $\lrprth{m_{1},m_{2}} \in M_{1} \times M_{2}$, se satisface que
		\begin{align*}
			g_{2}h\lrprth{m_{1},m_{2}}&=g_{2}\lrprth{f_{1}\lrprth{m_{1}}+f_{2}\lrprth{m_{2}}}\\
			&=g_{2}f_{1}\lrprth{m_{1}}+g_{2}f_{2}\lrprth{m_{2}}\\
			&=0+m_{2}\\
			&=m_{2}\\
			&=1_{M_{2}}\pi_{2}\lrprth{m_{1},m_{2}}
		\end{align*}
		
		$\boxed{\text{c)}\Rightarrow\text{b)}}$ Sea $h:M_{1} \times M_{2} \longrightarrow M$ el morfismo proporcionado por la hipótesis. Además, la sucesión
		\begin{tikzcd}
				0 \arrow{r} & M_{2} \arrow{r}{i_{2}} & M_{1} \times M_{2} \arrow{r}{\pi_{1}} & M_{1} \arrow{r} & 0
		\end{tikzcd}
		se parte. Definimos $f_{2}:M_{2} \longrightarrow M$ como $f_{2}=hi_{2}$, y $g_{1}:M \longrightarrow M_{1}$ como $g_{1}=\pi_{1}h^{-1}$.\\
		
		Luego, se satisfacen las siguientes igualdades
		\begin{align*}
			& g_{1}f_{1}=g_{1}hi_{1}=\pi_{1}h^{-1}hi_{1}=\pi_{1}i_{1}=1_{M_{1}}\\
			& g_{2}f_{2}=g_{2}hi_{2}=\pi_{2}i_{2}=1_{M_{2}}\\
			& g_{1}f_{2}=g_{1}hi_{2}=\pi_{1}h^{-1}hi_{2}=\pi_{1}i_{2}=0\\
			& g_{2}f_{1}=g_{2}hi_{1}=\pi_{2}h^{-1}hi_{1}=\pi_{2}i_{1}=0\\
			& g_{1}f_{1}+g_{2}f_{2}=1_{M_{1}}+1_{M_{2}}=1_{M_{1} \times M_{2}}=1_{M}
		\end{align*}
		
		$\boxed{\text{b)}\Rightarrow\text{a)}}$ Por hipótesis, existe un morfismo de $R$-módulos $f_{2}:M_{2} \longrightarrow M$ tal que $g_{2}f_{2}=1_{M_{2}}$. Por tanto, $\eta$ es una sucesión que se parte.
	\end{proof}
	
	%%%%%%%% Ej 58 %%%%%%%
	\item Sean $\varphi_{i}:A_{i} \longrightarrow B_{i}$, con $i=1,2$, minimales a derecha en $Mod\lrprth{R}$. Pruebe que $\varphi_{1}\coprod\varphi_{2}:A_{1} \coprod A_{2} \longrightarrow B_{1} \coprod B_{2}$ es minimal a derecha.
	\begin{proof}
		Sea $\psi:\varphi_{1}\coprod\varphi_{2}\longrightarrow\varphi_{1}\coprod\varphi_{2}$. Entonces $\psi$ es de la forma $\psi = \psi_{1}\coprod\psi_{2}$, con $\psi_{i}:A_{i} \longrightarrow B_{i}$, $i=1,2$. En efecto, si denotamos por $\eta_{i}:A_{1} \coprod A_{2} \longrightarrow B_{i}$, $i=1,2$, a la proyección canónica, entonces $\psi = \eta_{1}\psi\coprod\eta_{2}\psi$.\\
		
		Suponga, así, que $\psi = \psi_{1}\coprod\psi_{2}$. Luego, $\psi_{i} \in Hom\lrprth{\varphi_{i},\varphi_{i}}$, con $i=1,2$. Por la minimalidad a derecha de cada $\varphi_{1}$, se satisface que $\psi_{1}$ y $\psi_{2}$ son isomorfismos. Por lo que $\psi$ es un isomorfismo.\\
		$\therefore\varphi_{1}\coprod\varphi_{2}$ es minimal a derecha.
	\end{proof}
	
	%%%%%%%% Ej 61 %%%%%%%
	\item Sea $X\in\ringbimod{R}{S}{Mod}{lr}$. Pruebe que:
	\begin{enumerate}
		\item $\ringmodhom{R}{-}{X}: Mod\lrprth{R} \longrightarrow Mod\lrprth{\opst{S}}$ es un funtor contravariante aditivo.
		\item Para $\fntfam{M}{i}{n}$ en $Mod\lrprth{R}$ se tiene que
		\begin{align*}
			\ringmodhom{R}{\displaystyle\coprod_{i=1}^{n}M_{i}}{\ringbimod{R}{S}{X}{lr}}=\displaystyle\coprod_{i=1}^{n}\ringmodhom{R}{\ringmod{R}{M_{i}}{l}}{\ringbimod{R}{S}{X}{lr}}
		\end{align*}
		en $Mod\lrprth{\opst{S}}$
	\end{enumerate}
	\begin{proof}
		$\boxed{\text{(a)}}$ Primeramente, ya sabemos que $\ringmodhom{R}{-}{X}$ es un funtor contravariante. Entonces bastará probar que éste es aditivo.\\
		
		Sean $M,N \in Mod\lrprth{R}$. Veremos que $\varphi=\ringmodhom{R}{-}{X}$, con
		\begin{align*}
			\varphi:\ringmodhom{R}{M}{N}\longrightarrow\ringmodhom{\opst{S}}{\ringmodhom{R}{N}{X}}{\ringmodhom{R}{M}{X}},
		\end{align*}
		es un isomorfismo.\\
		
		Sea $f\in\ringmodhom{R}{M}{N}$. Entonces $\varphi\lrprth{f}\ringmodhom{\opst{S}}{\ringmodhom{R}{N}{X}}{\ringmodhom{R}{M}{X}}$ es el morfismo $\varphi\lrprth{f}\lrprth{g}=g \circ f$. De esta manera, $\varphi$ es un morfismo. En efecto, sean $f,g\in\ringmodhom{R}{M}{N}$, $r \in R$ y $h\in\ringmodhom{R}{N}{X}$, entonces
		\begin{align*}
			\varphi\lrprth{f+rg}\lrprth{h} &= \lrprth{f+rg} \circ h\\
			&=f \circ h + \lrprth{rg} \circ h\\
			&=f \circ h + r\lrprth{g \circ h}\\
			&=\varphi\lrprth{f}\lrprth{h}+r\varphi\lrprth{g}\lrprth{h}\\
			&=\lrprth{\varphi\lrprth{f}+r\varphi\lrprth{g}}\lrprth{h}
		\end{align*}
		Por tanto, $\varphi$ es morfismo. $\therefore\ringmodhom{R}{-}{X}$ es aditivo.
	
		$\boxed{\text{(b)}}$ Definimos $\rho : \ringmodhom{R}{\displaystyle\coprod_{i=1}^{n}M_{i}}{\ringbimod{R}{S}{X}{lr}} \longrightarrow \displaystyle\coprod_{i=1}^{n}\ringmodhom{R}{\ringmod{R}{M_{i}}{l}}{\ringbimod{R}{S}{X}{lr}}$ como $\rho\lrprth{\varphi}=\fntuple{\varphi\iota}{i}{n}$.\\
		
		Veamos que $\rho$ es un morfismo en $Mod\lrprth{\opst{S}}$. Para dicho fin, considere $\varphi , \psi \in \ringmodhom{R}{\displaystyle\coprod_{i=1}^{n}M_{i}}{\ringbimod{R}{S}{X}{lr}}$ y $s \in S$.
		\begin{align*}
			\rho\lrprth{\varphi + \psi s} &= \fntuple{\lrprth{\varphi + \psi s}\iota}{i}{n}\\
			&= \fntuple{\varphi\iota}{i}{n} + \fntuple{\lrprth{\psi s}\iota}{i}{n}\\
			&= \fntuple{\varphi\iota}{i}{n} + \fntuple{\psi\iota}{i}{n} s\\
			&= \rho\lrprth{\varphi} + \rho\lrprth{\psi} s
		\end{align*}
		
		Por otro lado, $\rho$ es un inyectivo. En efecto, si $\rho\lrprth{\varphi}=0$, entonces se tiene que $\fntuple{\varphi\iota}{i}{n}=0$. Luego, $\varphi=0$. Por tanto $Ker\lrprth{\rho}=0$.\\
		
		Ahora, sea $\fntuple{\varphi}{i}{n} \in \displaystyle\coprod_{i=1}^{n}\ringmodhom{R}{\ringmod{R}{M_{i}}{l}}{\ringbimod{R}{S}{X}{lr}}$. Entonces cada $\varphi_{i}$ es un morfismo $\varphi_{i}:M_{i} \longrightarrow X$. Así, por la propiedad universal del coproducto, existe $\varphi:\displaystyle\coprod_{i=1}^{n}M_{i} \longrightarrow X$ tal que $\varphi\iota_{i}=\varphi_{i}$. De esta manera, $\rho\lrprth{\varphi}=\fntuple{\varphi}{i}{n}$. Por tanto, $\rho$ es un isomorfismo.\\
		$\therefore\ringmodhom{R}{\displaystyle\coprod_{i=1}^{n}M_{i}}{\ringbimod{R}{S}{X}{lr}}=\displaystyle\coprod_{i=1}^{n}\ringmodhom{R}{\ringmod{R}{M_{i}}{l}}{\ringbimod{R}{S}{X}{lr}}$
	\end{proof}
	
	%%%%%%%% Ej 64 %%%%%%%%%
	\item Sea $M \in Mod\lrprth{R}$. Pruebe que:\\
	$M$ es proyectivo y f.g.$\Leftrightarrow$existe $n\in\mathbb{N}$ tal que $M$ es isomorfo a un sumando directo de $\ringmod{R}{R^{n}}{l}$.
	\begin{proof}
		$\boxed{\Rightarrow )}$ Puesto que $M$ es f.g., existe $n\in\mathbb{N}$ tal que la siguiente sucesión en $Mod\lrprth{R}$
		\begin{tikzcd}
			0 \arrow{r} & Ker(f) \arrow{r} & R^{n} \arrow{r}{f} & M \arrow{r} & 0
		\end{tikzcd}
		es exacta. Ésta a su vez se parte, toda vez que $M$ es proyectivo.\\
		$\therefore M$ es sumando directo de $R^{n}$.\\
		
		$\boxed{\Leftarrow )}$ Suponga que $\ringmod{R}{R^{n}}{l} \simeq M \oplus K$. Entonces $M$ es f.g., y la sucesión en $Mod\lrprth{R}$
		\begin{tikzcd}
			0 \arrow{r} & K \arrow{r} & R^{n} \arrow{r} & M \arrow{r} & 0
		\end{tikzcd}
		se parte.\\
		$\therefore M$ es proyectivo y f.g.
	\end{proof}
	
	%%%%%%%%% Ej 67 %%%%%%%%%
	\item Sea $R$ un anillo no trivial. Pruebe que:\\
	$R$ es semisimple y conmutativo$\Leftrightarrow R \simeq \displaystyle\bigtimes_{i=1}^{t} K_{i}$ como anillos, donde $K_{i}$ es un campo $\forall i \in [1,t]$
	\begin{proof}
		$\boxed{\Leftarrow )}$ Dado que cada $K_{i}$ es un campo y $R \simeq \displaystyle\bigtimes_{i=1}^{t} K_{i}$, se satisface que $R$ es semisimple y conmutativo.\\
		
		$\boxed{\Rightarrow )}$ En virtud del teorema de \textbf{Wedderburn-Artin}, $R$ es isomorfo a $\displaystyle\bigtimes_{i=1}^{t} Mat_{n_{i} \times n_{i}} \lrprth{D_{i}}$, con $n_{i}\in\mathbb{N}$ y $D_{i}$ un anillo con división. Ahora, por la conmutatividad de $R$, la única posibilidad es que $n_{i}=1$ y $D_{i}$ sea conmutativo, para $i \in [1,t]$.\\
		$\therefore R \simeq \displaystyle\bigtimes_{i=1}^{t} K_{i}$, con $K_{i}$ un campo, $\forall i \in [1,t]$
	\end{proof}
\end{enumerate}
\end{document}