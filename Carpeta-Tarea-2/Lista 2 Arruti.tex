\documentclass{article}
\usepackage[utf8]{inputenc}
\usepackage{mathrsfs}
\usepackage[spanish,es-lcroman]{babel}
\usepackage{amsthm}
\usepackage{amssymb}
\usepackage{enumitem}
\usepackage{graphicx}
\usepackage{caption}
\usepackage{float}
\usepackage{eufrak}
\usepackage{nicefrac}
\usepackage{amsmath,stackengine,scalerel,mathtools}
\usepackage{xparse, tikz-cd, pgfplots}
\usepackage{comment}
\def\subnormeq{\mathrel{\scalerel*{\trianglelefteq}{A}}}

\newcommand{\crdnlty}[1]{
	\left|#1\right|
}
\newcommand{\lrprth}[1]{
	\left(#1\right)
}
\newcommand{\lrbrack}[1]{
	\left\{#1\right\}
}
\newcommand{\descset}[3]{
	\left\{#1\in#2\ \vline\ #3\right\}
}
\newcommand{\descapp}[6]{
	#1: #2 &\rightarrow #3\\
	#4 &\mapsto #5#6 
}
\newcommand{\arbtfam}[3]{
	{\left\{{#1}_{#2}\right\}}_{#2\in #3}
}
\newcommand{\arbtfmnsub}[3]{
	{\left\{{#1}\right\}}_{#2\in #3}
}
\newcommand{\fntfmnsub}[3]{
	{\left\{{#1}\right\}}_{#2=1}^{#3}
}
\newcommand{\fntfam}[3]{
	{\left\{{#1}_{#2}\right\}}_{#2=1}^{#3}
}
\newcommand{\fntfamsup}[4]{
	\lrbrack{{#1}^{#2}}_{#3=1}^{#4}
}
\newcommand{\arbtuple}[3]{
	{\left({#1}_{#2}\right)}_{#2\in #3}
}
\newcommand{\fntuple}[3]{
	{\left({#1}_{#2}\right)}_{#2=1}^{#3}
}
\newcommand{\gengroup}[1]{
	\left< #1\right>
}
\newcommand{\stblzer}[2]{
	St_{#1}\lrprth{#2}
}
\newcommand{\cmmttr}[1]{
	\left[#1,#1\right]
}
\newcommand{\grpindx}[2]{
	\left[#1:#2\right]
}
\newcommand{\syl}[2]{
	Syl_{#1}\lrprth{#2}
}
\newcommand{\grtcd}[2]{
	mcd\lrprth{#1,#2}
}
\newcommand{\lsttcm}[2]{
	mcm\lrprth{#1,#2}
}
\newcommand{\amntpSyl}[2]{
	\mu_{#1}\lrprth{#2}
}
\newcommand{\gen}[1]{
	gen\lrprth{#1}
}
\newcommand{\ringcenter}[1]{
	C\lrprth{#1}
}
\newcommand{\zend}[2]{
	End_{\mathbb{Z}}^{#2}\lrprth{#1}
}
\newcommand{\genmod}[2]{
	\left< #1\right>_{#2}
}
\newcommand{\genlin}[1]{
	\mathscr{L}\lrprth{#1}
}
\newcommand{\opst}[1]{
	{#1}^{op}
}
\newcommand{\ringmod}[3]{
	\if#3l
	{}_{#1}#2
	\else
	\if#3r
	#2_{#1}
	\fi
	\fi
}
\newcommand{\ringbimod}[4]{
	\if#4l
	{}_{#1-#2}#3
	\else
	\if#4r
	#3_{#1-#2}
	\else 
	\ifstrequal{#4}{lr}{
		{}_{#1}#3_{#2}
	}
	\fi
	\fi
}
\newcommand{\ringmodhom}[3]{
	Hom_{#1}\lrprth{#2,#3}
}

\ExplSyntaxOn

\NewDocumentCommand{\functor}{O{}m}
{
	\group_begin:
		\keys_set:nn {nicolas/functor}{#2}
		\nicolas_functor:n {#1}
	\group_end:
}

\keys_define:nn {nicolas/functor}
{
	name     .tl_set:N = \l_nicolas_functor_name_tl,
	dom   .tl_set:N = \l_nicolas_functor_dom_tl,
	codom .tl_set:N = \l_nicolas_functor_codom_tl,
	arrow      .tl_set:N = \l_nicolas_functor_arrow_tl,
	source   .tl_set:N = \l_nicolas_functor_source_tl,
	target   .tl_set:N = \l_nicolas_functor_target_tl,
	Farrow      .tl_set:N = \l_nicolas_functor_Farrow_tl,
	Fsource   .tl_set:N = \l_nicolas_functor_Fsource_tl,
	Ftarget   .tl_set:N = \l_nicolas_functor_Ftarget_tl,	
	delimiter .tl_set:N= \_nicolas_functor_delimiter_tl,	
}

\dim_new:N \g_nicolas_functor_space_dim

\cs_new:Nn \nicolas_functor:n
{
	\begin{tikzcd}[ampersand~replacement=\&,#1]
		\dim_gset:Nn \g_nicolas_functor_space_dim {\pgfmatrixrowsep}		
		\l_nicolas_functor_dom_tl
		\arrow[r,"\l_nicolas_functor_name_tl"] \&
		\l_nicolas_functor_codom_tl
		\\[\dim_eval:n {1ex-\g_nicolas_functor_space_dim}]
		\l_nicolas_functor_source_tl
		\xrightarrow{\l_nicolas_functor_arrow_tl}
		\l_nicolas_functor_target_tl
		\arrow[r,mapsto] \&
		\l_nicolas_functor_Fsource_tl
		\xrightarrow{\l_nicolas_functor_Farrow_tl}
		\l_nicolas_functor_Ftarget_tl
		\_nicolas_functor_delimiter_tl
	\end{tikzcd}
}
\ExplSyntaxOff

\title{Lista 2}
\author{}
\date{}

\theoremstyle{definition}
\newtheorem{define}{Definición}

\theoremstyle{plain}
\newtheorem{teor}{Teorema}[section]

\theoremstyle{plain}
\newtheorem{prop}{Proposición}[section]

\theoremstyle{definition}
\newtheorem{ejemp}{Ejemplo}[section]

\theoremstyle{definition}
\newtheorem*{coro}{Corolario} 

\theoremstyle{definition}
\newtheorem*{obs}{Observaci\'on}

\theoremstyle{definition}
\newtheorem*{pron}{Proposición}

\theoremstyle{definition}

\newtheorem{lem}{Lema}

\theoremstyle{definition}
\newtheorem*{nota}{Nota}

\begin{document}
	
	\maketitle
\begin{define}
	Sean $R$ y $S$ anillos. Decimos que un grupo abeliano, $M$, es un $R$-derecho y $S$-derecho bimódulo si
	\begin{enumerate}[label=\roman*)]
		\item $M\in\ringmod{R}{\text{Mod}}{r}\cap\ringmod{S}{\text{Mod}}{r}$;
		\item $(mr)s=(ms)r$, $\forall\ r\in R$, $\forall\ s\in S$ y $\forall\ m\in M$.
	\end{enumerate}
	En tal caso denotamos a $M$ como $\ringbimod{R}{S}{M}{r}$.
\end{define}
	\begin{enumerate}[label=\textbf{Ej \arabic*.}]
		\setcounter{enumi}{11}
\item Sea $M\in\ringmod{R}{\text{Mod}}{l}\cap\ringmod{S}{\text{Mod}}{l}$. Entonces $M\in\ringbimod{R}{S}{\text{Mod}}{l}$ si y sólo si $M\in\ringbimod{R}{\opst{S}}{\text{Mod}}{lr}$.
\begin{proof}
	Como $M\in\ringmod{R}{\text{Mod}}{l}\cap\ringmod{S}{\text{Mod}}{l}$ y, por el Ej. 8, $M\in\ringmod{\opst{S}}{\text{Mod}}{r}$, entonces $M\in\ringmod{R}{\text{Mod}}{l}\cap\ringmod{\opst{M}}{\text{Mod}}{r}$.\\ Sean $r\in R$, $s\in S$ y $m\in M$. 
	Dado que, ver Ej 8, $sm=ms^{op}$ entonces 
	\begin{align*}
		r(sm)=s(rm)&\iff r(m\opst{s})=(rm)\opst{s}\\
		\therefore\  M\in\ringbimod{R}{S}{\text{Mod}}{l}&\iff M\in\ringbimod{R}{\opst{S}}{\text{Mod}}{lr}.
	\end{align*}
\end{proof}
\item\item
\begin{define}
	Sean $M\in\ringmod{R}{\text{Mod}}{l}$ y $\arbtfam{X}{i}{I}$ una familia de $R$-submódulos de $M$. Definimos la suma de la familia $\arbtfam{X}{i}{I}$ como
	\begin{equation*}
		\sum_{i\in I}X_i:=\gengroup{\bigcup_{i\in I}X_i}_R.
	\end{equation*}
\end{define}
\item Sean $M\in\ringmod{R}{\text{Mod}}{l}$ y $\arbtfam{X}{i}{I}$ una familia de $R$-submódulos de $M$. Entonces
\begin{enumerate}[label=(\alph*)]
	\item \begin{equation*}
		\sum_{i\in I}X_i=\left\{\begin{tabular}{cc}
			$\lrbrack{0}$ &, $I=\varnothing$  \\
			\(\displaystyle
			\lrbrack{\sum_{j\in\lrbrack{i_1,\dotsc,i_n}\subseteq I}x_j\ \vline\ x_j\in X_j,\ n\in\mathbb{N}\setminus\lrbrack{0}}\) & , $I\neq\varnothing$
		\end{tabular}\right.
	\end{equation*}
	\item $\lrbrack{\genlin{M},\leq}$ es un reticulado completo. Más aún, si $\arbtfam{X}{i}{I}$ es una familia no vacía de $R$-submódulos de $M$,
	\begin{align*}
		sup\arbtfam{X}{i}{I}&=\sum_{i\in I}X_i,\\
		inf\arbtfam{X}{i}{I}&=\bigcap_{i\in I}X_i.
	\end{align*}
\end{enumerate}
\begin{proof}
	Verifiquemos primeramente el siguiente lema:
	\begin{lem}
		Sea $M\in\ringmod{R}{\text{Mod}}{l}$. Si $\mathcal{A}\subseteq\genlin{M}$ entonces $\bigcap \mathcal{A}\in\genlin{M}$.
	\end{lem}
	\begin{proof}
		Notemos que 
		\begin{align*}
			0+0&=0;\\
			r\bullet 0&=r\bullet(0+0)=r\bullet 0+r\bullet 0\\
			&\implies r\bullet 0=0,\ \forall\ r\in R.\\
			\implies &\lrbrack{0}\in\genlin{M}.
			\intertext{Además}
			0_R\bullet x&=\lrprth{0_R+0_R}\bullet x=0_R\bullet x+0_R\bullet x\\
			&\implies 0_R\bullet x=0,\ \forall\ x\in M.
			\intertext{Por lo anterior, y dado que si $X\in\genlin{M}$ entonces $X\neq\varnothing$, se tiene que}
			&\lrbrack{0}\subseteq X,\ \forall\ X\in \genlin{M}.
		\end{align*}
		Con lo cual $\bigcap \mathcal{A}\neq\varnothing$, pues $\lrbrack{0}\subseteq \bigcap \mathcal{A}$. Sean $r\in R,\ a,b\in \bigcap\mathcal{A}$ y $A\in\mathcal{A}$. Como $A\leq M$
		\begin{align*}
			ra,a+b&\in A\\
			\implies ra,a+b&\in A, \forall\ A\in\mathcal{A}\\
			\implies ra,a+b&\in A, \forall\ \bigcap\mathcal{A}\\
			&\implies \bigcap\mathcal{A}\in\genlin{M}.
		\end{align*}
	\end{proof}
	Por el lema anterior el submódulo generado por un conjunto $A\supseteq X$ está bien definido y, más aún, es el mínimo submódulo de $M$, con respecto a $\subseteq$, que contiene a $A$.
	$\boxed{\text{(a)}}$
	Supongamos que $I=\varnothing$, entonces $\bigcup_{i\in I}X_i=\varnothing$, y así 
	\begin{align*}
		\sum_{i\in I}X_i&=\bigcap\descset{X}{\genlin{M}}{\varnothing\subseteq X}\\
		&=\bigcap\genlin{M}.
		\intertext{Del lema se tiene que $0\in X,\ \forall\ X\in \genlin{M}$ y que $\lrbrack{0}\in\genlin{M}$, con lo cual}
		\lrbrack{0}&\subseteq\bigcap_{X\in\genlin{M}}X=\bigcap\genlin{M}\subseteq\lrbrack{0}\\
		&\therefore\ \sum_{i\in I}X_i=\lrbrack{0}.
	\end{align*}
	Supongamos ahora que $I\neq\varnothing$. Si 
	\begin{equation*}
		S:=\lrbrack{\sum_{j\in\lrbrack{i_1,\dotsc,i_n}\subseteq I}x_j\ \vline\ x_j\in X_j,\ n\in\mathbb{N}\setminus\lrbrack{0}}
	\end{equation*}
	afirmamos que $S\in\genlin{M}$. En efecto:\\
	Como $I\neq\varnothing$ y $X_i\neq\varnothing$, $\forall\ i\in I$, entonces $S\neq\varnothing$. Sean $r\in R$ y $a,b\in S$, luego $\exists\ n,m\in\mathbb{N}$ tales que
	\begin{align*}
		a&=\sum_{j\in\lrbrack{i_1,\dotsc,i_n}}x_j\\
		b&=\sum_{j\in\lrbrack{k_1,\dotsc,k_m}}y_j.
		\intertext{En caso que $\lrbrack{i_1,\dotsc,i_n}= \lrbrack{k_1,\dotsc,k_n}$}
		a+b&=\sum_{j\in\lrbrack{i_1,\dotsc,i_n}}\lrprth{x_j+y_j}
		\intertext{Como $X_j\leq M$, $x_j+y_j\in X_j$, $\forall\ j\in\lrbrack{i_1,\dotsc,i_n}$; luego $a+b\in S$. Si ahora $\lrbrack{i_1,\dotsc,i_n}\cap
			\lrbrack{k_1,\dotsc,k_n}=\varnothing$ consideremos}
		l_r&:=i_r,\ \forall r\in[1,n]\\
		l_{n+r}&:=k_r,\ \forall r\in[1,m].
		\intertext{Así}
		a+b&=\sum_{j\in\lrbrack{l_1,\dotsc,l_{n+m}}}z_j\\
		&\implies a+b\in S.
		\intertext{Finalmente, reetiquetando de ser necesario, si}
		A&:=\lrbrack{i_1,\dotsc,i_n}\\
		B&:=\lrbrack{k_1,\dotsc,k_m}\\
		D&:=\lrbrack{i_1,\dotsc,i_n}\cap
		\lrbrack{k_1,\dotsc,k_n}\\
		E&:=\lrbrack{i_1,\dotsc,i_n}\cup
		\lrbrack{k_1,\dotsc,k_n}=\lrbrack{l_1,\dotsc,l_{r},l_{r+1},\dotsc,l_{t}},\\
		\intertext{con $\crdnlty{D}=r>0$, $\crdnlty{E\setminus D}=t>0$, entonces}
		a+b&=\sum_{j\in D}\lrprth{x_j+y_j}+\sum_{j\in A\setminus D}x_j+\sum_{j\in B\setminus D}y_j \\
		&=\sum_{j\in E}z_j\\
		&\implies a+b\in S.
	\end{align*}
	Por otro lado
	\begin{align*}
		r\bullet a&=r\bullet\sum_{j\in A}x_j=\sum_{j\in A}r\bullet x_j.
	\end{align*}
	De modo que $r\bullet a\in S$, pues $A\subseteq I$ es finito y, como $X_j\leq M$, $r\bullet x_j\in X_j$, $\forall\ j\in A$; y por lo tanto $S\leq M$. \\
	Como $\lrbrack{i}\subseteq I$, $\forall\ i\in I$, entonces
	$\bigcup_{i\in I}X_i\subseteq S$. De modo que
	\begin{equation*}
		\sum_{i\in I}X_i=\gengroup{\bigcup_{i\in I}X_i}_R\subseteq S.
	\end{equation*}
	Ahora si $Y\leq M$ es tal que $Y\supseteq \bigcup_{i\in I}X_i$, $J:=\lrbrack{i_1,\dotsc,i_n}\subseteq I$ y $a_j\in X_j\subseteq\bigcup_{i\in I}X_i,\ \forall\ j\in J$, entonces
	\begin{align*}
		\sum_{j\in J}a_j&\in Y\\
		\implies S&\subseteq Y,\ \forall\ Y\in\genlin{M}\text{ tal que } Y\supseteq \bigcup_{i\in I}X_i\\
		\implies S&\subseteq \gengroup{\bigcup_{i\in I}X_i}_R=\sum_{i\in I}X_i\\
		&\therefore\ S=\sum_{i\in I}X_i.
	\end{align*}
	$\boxed{\text{(b)}}$ El par $(\genlin{M},\leq)$ es un CPO puesto que la relación $\subseteq$ es un orden parcial.\\
	Sea $C\leq M$ cota superior de $S$. Entonces $X_i\leq C$, $\forall\ i\in I$; luego $C\subseteq \bigcup_{i\in I}X_i$. Dado $\sum_{i\in I}X_i$ es el mínimo submódulo, con respecto a $\subseteq$, que contiene a $\bigcup_{i\in I}X_i$ se tiene que $\sum_{i\in I}X_i\leq C$ y por lo tanto $sup\lrprth{S}=\sum_{i\in I}X_i$.\\
	Sea $C\leq M$ cota inferior de $S$. Entonces $C\leq X_i$, $\forall\ i\in I$; luego $C\supseteq \bigcap_{i\in I}X_i$. y así $\bigcap_{i\in I}X_i\leq C$. Por lo tanto $inf\lrprth{S}=\bigcap_{i\in I}X_i$.\\
	\begin{equation*}
		\therefore\ \lrprth{\genlin{M},\leq}\text{ es un reticulado completo.}
	\end{equation*}
\end{proof}
\item\item
\item Sea $M\in\ringmod{R}{\text{Mod}}{l}$. Las siguientes condiciones son equivalentes:
\begin{enumerate}[label=(\alph*)]
	\item $M$ es finitamente generado.
	\item $\exists\ n\in\mathbb{N}\setminus\lrbrack{0}$ y $f:\mathbb{R}^n\rightarrow M$ epimorfismo de $R$-módulos.
\end{enumerate}
\begin{proof}
	Verifiquemos primero el siguiente lema:
	\begin{lem}
		Sea $M\in\ringmod{R}{\text{Mod}}{l}$. Si $X\subseteq M$ entonces
		\begin{equation*}
			\genmod{X}{R}=\left\{\begin{tabular}{cl}
				$\lrbrack{0}$ & , $X=\varnothing$\\
				$\lrbrack{\sum_{i=1}^nr_ix_i\ \vline\ n\in\mathbb{N}\setminus\lrbrack{0},\ r_i\in R,\ x_i\in X\ \forall\ i\in[1,n]}$ & , $X\neq\varnothing$
			\end{tabular}\right.
		\end{equation*}
	\end{lem}
	\begin{proof}
		El caso $X=\varnothing$ se verificó en el Ej. 15(a) (en el caso $I=\varnothing$). Supongamos que $X\neq\varnothing$.\\
		Sea $S:=\lrbrack{\sum_{i=1}^nr_ix_i\ \vline\ n\in\mathbb{N}\setminus\lrbrack{0},\ r_i\in R,\ x_i\in X\ \forall\ i\in[1,n]}$. $S\neq\varnothing$, pues $R\neq\varnothing\neq X$ y $S\subseteq M$, pues $M\in\ringmod{R}{\text{Mod}}{l}$.\\
		Sean $a,b\in S$. Existen $n,m\in\mathbb{N}\setminus\lrbrack{0}$ y $r_i,s_i\in R$, $x_i,y_i\in X$ tales que $a=\sum_{i=1}^{n}r_ix_i$ y $b=\sum_{i=1}^{m}s_iy_i$ y
		\begin{align*}
			t_i&=\left\{\begin{tabular}{cc}
				$r_i$ & , $i\in[1,n]$\\
				$s_{i-n}$ & , $i\in[n+1,n+m]$
			\end{tabular}\right.\\
			z_i&=\left\{\begin{tabular}{cc}
				$r_i$ & , $i\in[1,n]$\\
				$s_{i-n}$ & , $i\in[n+1,n+m]$
			\end{tabular}\right.  .
		\intertext{Entonces}
			a+b&=\sum_{i=1}^{n}r_ix_i + \sum_{i=1}^{m}s_iy_i\\
				&=\sum_{i=1}^{n+m}t_iz_i\\
			&\implies a+b\in S.\\
			\intertext{Sea $r\in R$. Entonces}
			ra&=r\lrprth{\sum_{i=1}^{n}r_ix_i}=\sum_{i=1}^{n}\lrprth{rr_i}x_i
			\intertext{Si $u_i:=rr_i$ entonces}						ra&=\sum_{i=1}^{n}u_ix_i\in S\\
			\implies & S\in\genlin{M}.
		\end{align*}
	Además, si $x\in X$ entonces $x=1_Rx\in S$, con lo cual $X\subseteq S$ y por lo tanto $\genmod{X}{R}\subseteq S$.\\
	Por otro lado, si $Y\leq M$, $X\subseteq Y$, $r_i\in R, x_i\in X\ \forall\ i\in[1,n], n\in\mathbb{N}\setminus\lrbrack{0}$ entonces
	\begin{align*}
		\sum_{i=1}^{n}r_ix_i&\in Y\\
		&\implies S\subseteq Y\\
		&\implies S\subseteq \genmod{X}{R}\\
		\therefore\ & S=\genmod{X}{R}.
	\end{align*}
	\end{proof}
	$\boxed{\implies}$ Existe $X\subseteq M$ finito tal que $M=\genmod{X}{R}$. Si $X=\varnothing$ entonces $M=\lrbrack{0}$ y en tal caso la aplicación
	\begin{align*}
		\descapp{f}{R}{M}{r}{0}{}
	\end{align*}
	es un epimorfismo de $R$-módulos a izquierda. \\
	Supongamos ahora que $X\neq\varnothing$. Entonces $\exists\ m\in\mathbb{N}\setminus\lrbrack{0}$ tal que $X=\lrbrack{x_1,\dotsc,x_m}$. Consideremos la aplicación
		\begin{align*}
			\descapp{f}{R^m}{M}{\fntuple{r}{i}{m}}{\sum_{i=1}^{m}r_ix_i}{.}
		\end{align*}
	Sean $r\in R$ y $\fntuple{s}{i}{m},\fntuple{t}{i}{m}\in R^m$
	\begin{align*}
		f\lrprth{r\lrprth{\fntuple{s}{i}{m}+\fntuple{t}{i}{m}}}&=		f\lrprth{\lrprth{rs_i+rt_i}_{i=1}^m}=\sum_{i=1}^{m}\lrprth{rs_i+rt_i}x_i\\
		&=\sum_{i=1}^{m}(rs_i)x_i+\sum_{i=1}^{m}(rt_i)x_i=r\lrprth{\sum_{i=1}^{m}s_ix_i+\sum_{i=1}^{n}t_ix_i}\\
		&=r\lrprth{f\lrprth{\fntuple{s}{i}{n}}+f\lrprth{\fntuple{t}{i}{n}}}\\
		\implies & f \text{ es un morfismo de $R$ módulos a izquierda.}
	\end{align*}
	Notemos que por el Lema 2, dado que $X=\lrbrack{x_1,\dotsc,x_m}$ y, $\forall\ m\in M$, $0_Rm=0$, se tiene que
	\begin{equation*}
		\genmod{X}{R}=\lrbrack{\sum_{i=1}^mr_ix_i\ \vline\ r_i\in R\ \forall\ i\in[1,n]}.
	\end{equation*}
	Sea $y\in M=\genmod{X}{R}$. Por la observación anterior $\exists\ r_i\in R$ tales que $y=\sum_{i=1}^{m}x_i$, con lo cual, si $x:=\fntuple{r}{i}{m}$, $y=f(x)$. Por lo tanto $f$ es un epimorfismo de $R$-módulos a izquierda.\\
	$\boxed{\impliedby}$ Verifiquemos primero los siguientes resultados:
	\begin{lem}
		Sea $f:M\rightarrow N$ un morfismo de $R$-módulos a izquierda. Entonces $f\lrprth{A}\in\genlin{N}$, $\forall\ A\in\genlin{M}$.
	\end{lem}
	\begin{proof}
		Sea $A\in \genlin{M}$. Como $A\neq\varnothing$ entonces $f\lrprth{A}\neq\varnothing$, además $f(A)\subseteq M$. Sean $r\in R$ y $x,y\in f(A)$. Existen $a,b\in A$ tales que $f(a)=x$ y $f(b)=y$. Así
		\begin{align*}
			rx+y&=rf(x)+f(b)=f(rx+b) && f\in\ringmodhom{R}{M}{N}\\
			f(rx+b)&\in f\lrprth{A} && A\in\genlin{M}\\
			\implies & f(A)\in\genlin{N}.
		\end{align*}
	\end{proof}
	\begin{lem}
	Sea $f:M\rightarrow N$ un morfismo de $R$-módulos a izquierda. Entonces $f\lrprth{\genmod{A}{R}}=\genmod{f\lrprth{A}}{R}$, $\forall\ A\subseteq M$.
	\end{lem}
	\begin{proof}
		Sea $A\subseteq M$. Como $A\subseteq\genmod{A}{R}$ entonces $f\lrprth{A}\subseteq f\lrprth{\genmod{A}{R}}$, de modo que, por el Lema 3, $f\lrprth{\genmod{A}{R}}$ es un $R$-submódulo de $N$ que contiene a $f(A)$ y por lo tanto $\genmod{f(A)}{R}\subseteq f\lrprth{\genmod{A}{R}}$.\\
		Sean $n\in\mathbb{N}\setminus\lrbrack{0}$ y, $\forall\ i\in[1,n], r_i\in R$ y $x_i\in A$. Así, como $f\in\ringmodhom{R}{M}{N}$,
		\begin{align*}
			f\lrprth{\sum_{i=1}^{n}r_ix_i}&=\sum_{i=1}^{n}r_i f(x_i)\in\genmod{f(A)}{R}\\
			 & \implies f\lrprth{\genmod{A}{R}} \subseteq \genmod{f(A)}{R}\\
			 \therefore\ & f\lrprth{\genmod{A}{R}} = \genmod{f(A)}{R}.
		\end{align*}
	\end{proof}
	\begin{lem}
		Sea $f:M\rightarrow N$ un epimorfismo de $R$-módulos a izquierda. Si $M\in\ringmod{R}{\text{mod}}{l}$ entonces $N\in\ringmod{R}{\text{mod}}{l}$.
	\end{lem}
	\begin{proof}
		Como $M\in\ringmod{R}{\text{mod}}{l}$ $\exists\ X\subseteq M$ finito tal que $M=\genmod{X}{R}$ y así
		\begin{align*}
			N&=f(M) && f\text{ es sobre}\\
			&=f\lrprth{\genmod{X}{R}}=\genmod{f(X)}{R} && \text{Lema 4}
		\end{align*}
	Y como $\crdnlty{f(X)}\leq\crdnlty{X}$ entonces $f(X)$ es finito. Por lo tanto $N\in\ringmod{R}{\text{mod}}{l}$.\\
	\end{proof}
	Así, como $\exists\ n\in\mathbb{N}\setminus\lrbrack{0}$ y $f:R^n\rightarrow M$ epimorfismo de $R$-módulos a izquierda, por el Lema 5 basta verificar que $R^n\in\ringmod{R}{\text{mod}}{l}$.\\
	Sean $e_j:=\lrprth{u_i^j}_{i=1}^n\in R^n$, donde
		\begin{equation*}
		u_i^j=\left\{
		\begin{tabular}{cc}
			$0_R$ & , $i\neq j$ \\
			$1_R$ & , $i=j$ 
		\end{tabular}
		\right. ,
	\end{equation*}
	$\forall\ j\in[1,n]$, y $E:=\lrbrack{e_1,\dotsc,e_n}$. Así si $\fntuple{r}{i}{n}\in R^n$, entonces $\fntuple{r}{i}{n}=\sum_{i=1}^{n}r_ie_i$, con lo cual $R^n=\genmod{E}{R}$. Por lo tanto $R^n\in\ringmod{R}{\text{mod}}{l}$.\\
\end{proof}
\item\item
\item Sean $R$ y $S$ anillos.
\begin{enumerate}[label=(\alph*)]
	\item Sean 
	\begin{align*}
		Obj\lrprth{\mathcal{C}}&:=\ringbimod{R}{S}{\text{Mod}}{r},\\
		Hom\lrprth{\mathcal{C}}&:=\bigcup_{\lrprth{M,N}\in Obj\lrprth{\mathcal{C}}^2}Hom\lrprth{\ringbimod{R}{S}{M}{r},\ringbimod{R}{S}{N}{r}},\\
		\intertext{con}
		Hom\lrprth{\ringbimod{R}{S}{M}{r},\ringbimod{R}{S}{N}{r}}&:=Hom_R\lrprth{\ringmod{R}{M}{r},\ringmod{R}{N}{r}}\cap Hom_S\lrprth{\ringmod{S}{M}{r},\ringmod{S}{N}{r}},
	\end{align*}
	y $\circ$ la composición usual de funciones.\\ Entonces la clase $\ringbimod{R}{S}{\text{Mod}}{r}$ tiene estructura de categoría por medio de la tercia $\lrprth{Obj\lrprth{\mathcal{C}},Hom\lrprth{\mathcal{C}},\circ}$.
	\item $\ringbimod{R}{S}{\text{Mod}}{lr}\simeq\ringbimod{\opst{R}}{S}{\text{Mod}}{r}$, $\ringbimod{R}{S}{\text{Mod}}{lr}\simeq\ringbimod{R}{\opst{S}}{\text{Mod}}{l}$ y $\ringbimod{R}{S}{\text{Mod}}{l}\simeq\ringbimod{R}{\opst{S}}{\text{Mod}}{lr}$.
\end{enumerate}
\begin{proof}
	$\boxed{\text{(a)}}$ Si $M,N\in\ringbimod{R}{S}{\text{Mod}}{r}$, entonces $M$ y $N$ son conjuntos, con lo cual 
	\begin{equation*}
		N^M:=\lrbrack{f:M\rightarrow N\ |\ f\text{ es una función}}
	\end{equation*} 
    es un conjunto y así $\ringmodhom{R}{M_R}{N_R}$ es un conjunto, pues $$\ringmodhom{R}{M_R}{N_R}\subseteq B^A.$$
     Similarmente se encuentra que $\ringmodhom{S}{M_S}{N_S}$ es un conjunto, y así $Hom\lrprth{\ringbimod{R}{S}{M}{r},\ringbimod{R}{S}{N}{r}}$ es un conjunto $\forall\ M, N\in\ringbimod{R}{S}{\text{Mod}}{r}$. Además por definición $Hom\lrprth{\mathcal{C}}=\bigcup_{\lrprth{M,N}\in Obj\lrprth{\mathcal{C}}^2}Hom\lrprth{\ringbimod{R}{S}{M}{r},\ringbimod{R}{S}{N}{r}}$, con lo cual se satisface (P1).\\
Recordemos que, si $W,X,Y,Z$ son conjuntos y $f:W\rightarrow X, g:Y\rightarrow Z$ son funciones entonces $f=g$ si y sólo si $W=Y$, $X=Z$ y $f(w)=g(w)\ \forall\ w\in W$. Con lo cual si $(M,N)\neq\lrprth{O,P}$ entonces $N^M\cap P^O=\varnothing$ y por lo tanto $Hom\lrprth{\ringbimod{R}{S}{M}{r},\ringbimod{R}{S}{N}{r}}\cap Hom\lrprth{\ringbimod{R}{S}{O}{r},\ringbimod{R}{S}{P}{r}}=\varnothing$. Por lo tanto se satisface (P2).\\
Finalmente para verificar que (P3) se satisface, dado que la composición usual de funciones es asociativa y claramente $Id_X\in Hom\lrprth{\ringbimod{R}{S}{M}{r},\ringbimod{R}{S}{
		M}{r}}$ $\forall\ M\in\ringbimod{R}{S}{\text{Mod}}{r}$, basta probar que si $f\in\ringmodhom{R}{N}{O}, g\in\ringmodhom{R}{M}{N}$ entonces $f\circ g\in\ringmodhom{R}{M}{O}$; ya que en tal caso se tiene que la composición usual de funciones se restringe a una función asociativa
\begin{equation*}
	\circ:Hom\lrprth{\ringbimod{R}{S}{N}{r},\ringbimod{R}{S}{
			O}{r}}\times Hom\lrprth{\ringbimod{R}{S}{M}{r},\ringbimod{R}{S}{N}{r}}\rightarrow Hom\lrprth{\ringbimod{R}{S}{M}{r},\ringbimod{R}{S}{
			O}{r}}
\end{equation*}  
que admite identidades.\\
	Sean $f\in\ringmodhom{R}{N}{O}]$ y $g\in\ringmodhom{R}{M}{N}$. En partícular $f:N\rightarrow O$ y $g:M\rightarrow N$ son morfismos de grupo abelianos, con lo cual $f\circ g$ es un morfismo de grupos abelianos. Sean $r\in R$ y $m\in M$, así
	\begin{align*}
		f\circ g (rm)&=f\lrprth{g\lrprth{rm}}=f\lrprth{rg(m)}=rf\lrprth{g\lrprth{m}}\\
		&=r\lrprth{f\circ g(m)}.\\
		\implies & f\circ g\in\ringmodhom{R}{M}{O}.
	\end{align*} 
	$\boxed{\text{(b)}}$ Recordemos que, por el Ej. 8, $\lrprth{M,\bullet}\in\ringmod{R}{M}{l}$ si y sólo si $\lrprth{M,\opst{\bullet}}\in\ringmod{\opst{R}}{M}{r}$ (en adelante no haremos mención explícita de $\bullet$ y $\opst{\bullet}$). Por lo cual, si $r\in R, s\in S$ y $m\in M,$
	\begin{align*}
		r(ms)=(rm)s&\iff (ms)\opst{r}=(m\opst{r})s \tag{$*$}\label{lriffrr}
		\intertext{y así}
		M\in\ringbimod{R}{S}{\text{Mod}}{lr}&\iff M\in\ringbimod{\opst{R}}{S}{\text{Mod}}{r}.\tag{I}\label{lrbiffrrb}
	\end{align*}
 Más aún, si $f:M\rightarrow N$ es un morfismo de grupos abelianos, entonces
	\begin{align*}
		f\lrprth{r(ms)}=r\lrprth{f(m)s}&\iff 		f\lrprth{(ms)\opst{r}}=\lrprth{f(m)s}\opst{r}, \\& \forall\ r\in R, \forall\ s\in S,\forall\ m\in M.
		\intertext{De modo que, considerando el caso partícular $s=1_S$, se tiene que}
		f\in\ringmodhom{R}{\ringmod{R}{M}{l}}{\ringmod{R}{N}{l}}&\iff f\in\ringmodhom{\opst{R}}{\ringmod{\opst{R}}{M}{r}}{\ringmod{\opst{R}}{N}{r}}\\
		 \therefore\ f\in Hom\lrprth{\ringbimod{R}{S}{M}{lr},\ringbimod{R}{S}{N}{lr}}&\iff f\in Hom\lrprth{\ringbimod{\opst{R}}{S}{M}{r},\ringbimod{\opst{R}}{S}{N}{r}}.\tag{II}\label{lrmiffrrm}
	\end{align*}
	 De (\ref{lrbiffrrb}) y (\ref{lrmiffrrm}) se sigue que la correspondencia de categorías
	\begin{center}
		\functor{
			name=F_1,
			dom=\ringbimod{R}{S}{\text{Mod}}{lr},
			codom=\ringbimod{\opst{R}}{S}{\text{Mod}}{r},
			arrow=f,
			source=\ringbimod{R}{S}{M}{lr},
			target=\ringbimod{R}{S}{N}{lr},
			Farrow=f,
			Fsource=\ringbimod{\opst{R}}{S}{M}{r},
			Ftarget=\ringbimod{\opst{R}}{S}{N}{r},
		}
	\end{center}
	está bien definida y, más aún, por construcción $F\lrprth{Id_M}=Id_{F\lrprth{M}},\ \forall M\in\ringbimod{R}{S}{\text{Mod}}{lr}$. \\
	Sean $M\xrightarrow{f}N,N\xrightarrow{g} O\in\ringbimod{R}{S}{\text{Mod}}{lr}$, entonces
	\begin{align*}
		F\lrprth{g\circ f}&=g\circ f=F(g)\circ F(f)\\
		&\therefore\ F\text{ es un funtor.}
	\end{align*}
	Empleando que $\opst{\lrprth{\opst{R}}}=R$ en conjunto a los puntos (\ref{lriffrr}), (\ref{lrbiffrrb}) y (\ref{lrmiffrrm}), se tiene que la correspondencia de categorías
	\begin{center}
		\functor{
			name=G_1,
			codom=\ringbimod{R}{S}{\text{Mod}}{lr},
			dom=\ringbimod{\opst{R}}{S}{\text{Mod}}{r},
			arrow=g,
			Fsource=\ringbimod{R}{S}{M}{lr},
			Ftarget=\ringbimod{R}{S}{N}{lr},
			Farrow=g,
			source=\ringbimod{\opst{R}}{S}{M}{r},
			target=\ringbimod{\opst{R}}{S}{N}{r},
		}
	\end{center}
 es un funtor que satisface $G_1F_1=1_{\ringbimod{R}{S}{\text{Mod}}{lr}}$ y $F_1G_1=1_{\ringbimod{\opst{R}}{S}{\text{Mod}}{r}}$, y por lo tanto $\ringbimod{R}{S}{\text{Mod}}{lr}\simeq\ringbimod{\opst{R}}{S}{\text{Mod}}{r}$.\\
 Aplicando ahora el Ej. 8 al anillo $S$, por medio de un procedimiento análogo a lo previamente desarrollado, se verifica que $M\in\ringbimod{R}{S}{\text{Mod}}{lr}$ si y sólo si  $M\in\ringbimod{R}{\opst{S}}{\text{Mod}}{l}$, y que $f\in Hom\lrprth{\ringbimod{R}{S}{M}{lr},\ringbimod{R}{S}{N}{lr}}$ si y sólo si $f\in Hom\lrprth{\ringbimod{\opst{R}}{S}{M}{l},\ringbimod{\opst{R}}{S}{N}{l}}$. De modo que
 \begin{center}
 	\functor{
 		name=F_2,
 		dom=\ringbimod{R}{S}{\text{Mod}}{lr},
 		codom=\ringbimod{R}{\opst{S}}{\text{Mod}}{l},
 		arrow=f,
 		source=\ringbimod{R}{S}{M}{lr},
 		target=\ringbimod{R}{S}{N}{lr},
 		Farrow=f,
 		Fsource=\ringbimod{R}{\opst{S}}{M}{l},
 		Ftarget=\ringbimod{R}{\opst{S}}{N}{l},
 	}
 \end{center}
es un isomorfismo de categorías, con inversa
 \begin{center}
 	\functor{
 		name=G_2,
 		codom=\ringbimod{R}{S}{\text{Mod}}{lr},
		dom=\ringbimod{R}{\opst{S}}{\text{Mod}}{l},
		arrow=g,
		Fsource=\ringbimod{R}{S}{M}{lr},
		Ftarget=\ringbimod{R}{S}{N}{lr},
		Farrow=g,
		source=\ringbimod{R}{\opst{S}}{M}{l},
		target=\ringbimod{R}{\opst{S}}{N}{l},
		delimiter=\text{.},
 	}
 \end{center}
  Finalmente empleando, el Ej. 12 y un procedimiento análogo al previamente desarrollado se verifica que $M\in\ringbimod{R}{S}{\text{Mod}}{l}$ si y sólo si  $M\in\ringbimod{R}{\opst{S}}{\text{Mod}}{lr}$, y que $f\in Hom\lrprth{\ringbimod{R}{S}{M}{l},\ringbimod{R}{S}{N}{l}}$ si y sólo si $f\in Hom\lrprth{\ringbimod{R}{\opst{S}}{M}{lr},\ringbimod{R}{\opst{S}}{N}{lr}}$; y así
  \begin{center}
  	\functor{
  		name=F_3,
  		dom=\ringbimod{R}{S}{\text{Mod}}{l},
  		codom=\ringbimod{R}{\opst{S}}{\text{Mod}}{lr},
  		arrow=f,
  		source=\ringbimod{R}{S}{M}{l},
  		target=\ringbimod{R}{S}{N}{l},
  		Farrow=f,
  		Fsource=\ringbimod{R}{\opst{S}}{M}{lr},
  		Ftarget=\ringbimod{R}{\opst{S}}{N}{lr},
  	}
  \end{center}
  es un isomorfismo de categorías, con inversa
  \begin{center}
  	\functor{
  		name=G_3,
  		codom=\ringbimod{R}{S}{\text{Mod}}{lr},
  		dom=\ringbimod{R}{\opst{S}}{\text{Mod}}{l},
  		arrow=g,
  		Fsource=\ringbimod{R}{S}{M}{lr},
  		Ftarget=\ringbimod{R}{S}{N}{lr},
  		Farrow=g,
  		source=\ringbimod{R}{\opst{S}}{M}{l},
  		target=\ringbimod{R}{\opst{S}}{N}{l},
  		delimiter=\text{.},
  	}
  \end{center}
\end{proof}
\item\item
\item Sean $R$, $S$ y $T$ anillos.
\begin{enumerate}[label=(\alph*)]
	\item Sean $M\in\ringbimod{R}{S}{\text{Mod}}{lr}$, $N\in\ringbimod{R}{T}{\text{Mod}}{lr}$, $H:=\ringmodhom{R}{\ringbimod{R}{S}{M}{lr}}{\ringbimod{R}{T}{N}{lr}}$ y las siguientes aplicaciones, denotadas por medio del mismo simbolo para simplificar la notación,
	\begin{align*}
		\descapp{\bullet}{S\times H}{H}{(s,f)}{s\bullet f}{,}
		\intertext{con}
		\descapp{s\bullet f}{M}{N}{x}{f(xs)}{;}\\
		\descapp{\bullet}{H\times T}{H}{(f,t)}{f\bullet t}{,}
		\intertext{con}
		\descapp{f\bullet t}{M}{N}{x}{f(x)t}{.}
	\end{align*}
	A través de las aplicaciones anteriores  $H\in\ringbimod{S}{T}{\text{Mod}}{lr}$.
	\item Sean $M\in\ringbimod{S}{R}{\text{Mod}}{lr}$, $N\in\ringbimod{T}{R}{\text{Mod}}{lr}$, $H':=\ringmodhom{R}{\ringbimod{S}{R}{M}{lr}}{\ringbimod{T}{R}{N}{lr}}$ y las siguientes aplicaciones, denotadas por medio del mismo simbolo para simplificar la notación,
	\begin{align*}
		\descapp{\bullet}{T\times H}{H}{(t,f)}{t\bullet f}{,}
		\intertext{con}
		\descapp{t\bullet f}{M}{N}{x}{tf(x)}{;}\\
		\descapp{\bullet}{H\times S}{H}{(f,s)}{f\bullet s}{,}
		\intertext{con}
		\descapp{f\bullet s}{M}{N}{x}{f(sx)}{.}
	\end{align*}
	A través de las aplicaciones anteriores $H\in\ringbimod{T}{S}{\text{Mod}}{lr}$.
\end{enumerate}
\begin{proof}
	$\boxed{\text{(a)}}$ Notemos primeramente que $G:=Hom\lrprth{M,N}$ es un grupo con la suma usual de funciones, pues $M$ y $N$ lo son con sus respectivas operaciones, y que, si $f,g\in H$, $r\in R$ y $m\in M$, entonces
	\begin{align*}
		\lrprth{f-g}(rm)&=f(rm)-g(rm)=rf(m)-rg(m)=r\lrprth{f(m)-g(m)}\\
		&=r\lrprth{\lrprth{f-g}(m)}.\\
		&\implies f-g\in H\\
		&\implies H\leq G.
	\end{align*}
	Así, en partícular, $H$ es un grupo abeliano. Verifiquemos ahora que por medio de la primera aplicación $H\in\ringmod{S}{\text{Mod}}{l}$. Sean $f,g\in H, s,s'\in S$ y $m\in M$, entonces
	\begin{align*}
		\lrprth{\lrprth{s+s'}\bullet f}(m)&=f\lrprth{m\lrprth{s+s'}}\\
		&=f\lrprth{ms+ms'} && M\in\ringmod{S}{\text{Mod}}{r}\\
		&=f\lrprth{ms}+f\lrprth{ms'} && f\in Hom\lrprth{M,N}\\
		&=s\bullet f (m)+s'\bullet f (m)\\
		&=\lrprth{s\bullet f+ s'\bullet f}(m).\\
		\implies \lrprth{s+s'}\bullet f&= s\bullet f+ s'\bullet f.\\
		\lrprth{s\bullet\lrprth{f+g}}(m)&=\lrprth{f+g}\lrprth{ms}\\
		&=f(ms)+g(ms)\\
		&=s\bullet f (m)+ s\bullet g (m)\\
		&=\lrprth{s\bullet f+ s\bullet g }(m).\\
		\implies s\bullet\lrprth{f+g}&= s\bullet f+ s\bullet g.\\
		\lrprth{s\bullet\lrprth{s'\bullet f}}(m)&=\lrprth{s'\bullet f}\lrbrack{ms}\\
		&=f\lrprth{(ms)s'}\\
		&=f\lrprth{m(ss')} && M\in\ringmod{S}{\text{Mod}}{r}\\
		&=\lrprth{\lrprth{ss'}\bullet f}(m).\\
		\implies s\bullet\lrprth{s'\bullet f}&= \lrprth{ss'}\bullet f.\\
		\lrprth{1_S\bullet f}(m)&=f\lrprth{m1_s}\\
		&=f(m) && M\in\ringmod{S}{\text{Mod}}{r}.\\
		\implies 1_S\bullet f&= f.\\
		&\therefore\ H\in\ringmod{S}{\text{Mod}}{l}.
	\end{align*} 
	Verifiquemos ahora que por medio de la segunda aplicación $H\in\ringmod{T}{\text{Mod}}{r}$. Sean $f,g\in H, t,t'\in S$ y $m\in M$, entonces
	\begin{align*}
		\lrprth{\lrprth{f+g}\bullet t}(m)&=\lrprth{\lrprth{f+g}(m)}t\\
		&=\lrprth{f(m)+g(m)}t\\
		&=f(m)t+g(m)t && N\in\ringmod{T}{\text{Mod}}{r}\\
		&=f\bullet t(m)+g\bullet t(m)\\
		&=\lrprth{\lrprth{f+g}\bullet t}(m).\\
		\implies \lrprth{f+g}\bullet t&= \lrprth{f+g}\bullet t.\\
		\lrprth{f\bullet\lrprth{t+t'}}(m)&=f(m)\lrprth{t+t'}\\
		&=f(m)t+f(m)t'\\
		&=f\bullet t(m)+ f\bullet t' (m)\\
		&=\lrprth{f\bullet t+ f\bullet t' }(m).\\
		\implies f\bullet\lrprth{t+t'}&= f\bullet t+ f\bullet t'.\\
		\lrprth{\lrprth{f\bullet t}\bullet t'}(m)&=\lrprth{\lrprth{f\bullet t}(m)}t'\\
		&=\lrprth{f(m)t}t'\\
		&=f(m)\lrprth{tt'} && N\in\ringmod{T}{\text{Mod}}{r}\\
		&=\lrprth{f\bullet\lrprth{tt'}}(m).\\
		\implies \lrprth{f\bullet t}\bullet t'&= f\bullet\lrprth{tt'}.\\
		\lrprth{f\bullet 1_T}(m)&=f(m)1_T\\
		&=f(m) && N\in\ringmod{T}{\text{Mod}}{r}.\\
		\implies f\bullet 1_T&= f.\\
		&\therefore\ H\in\ringmod{T}{\text{Mod}}{r}.\\
		\intertext{Así}
		H&\in\ringmod{S}{\text{Mod}}{l}\cap \ringmod{T}{\text{Mod}}{r}.\\
		\intertext{Finalmente, notemos que}
		\lrprth{\lrprth{s\bullet f}\bullet t}(m)&=\lrprth{\lrprth{s\bullet f}(m)}t\\
		&=f(ms)t\\
		&=\lrprth{f\bullet t}(ms)\\
		&=\lrprth{s\bullet\lrprth{f\bullet t}}(m).\\
		\implies \lrprth{s\bullet f}\bullet t&= s\bullet\lrprth{f\bullet t}.\\
		&\therefore\ H\in\ringbimod{S}{T}{\text{Mod}}{lr}.
	\end{align*} 
	$\boxed{\text{(b)}}$ Es análogo a lo demostrado en (a), empleando ahora las propiedades de los morfismos de $R$-módulos a derecha para verificar que $H'$ es un grupo abeliano con la suma usual de funciones, y que $M\in\ringmod{S}{\text{Mod}}{l}, N\in\ringmod{T}{\text{Mod}}{l}$ para verificar que $H\in\ringbimod{T}{S}{\text{Mod}}{lr}$.
\end{proof}
\item\item
\item Sean $R$ y $S$ anillos, $e\in R$ y $\epsilon\in S$ idempotentes, $M\in\ringbimod{R}{S}{\text{Mod}}{lr}$, $R':=eRe$ y $S':=\epsilon S\epsilon$. Entonces:
\begin{enumerate}[label=(\alph*)]
	\item existen acciones tales que $Re\in\ringbimod{R}{R'}{\text{Mod}}{lr}$, $\epsilon S\in\ringbimod{S'}{S}{\text{Mod}}{lr}$, $eM\in\ringbimod{R'}{S}{\text{Mod}}{lr}$ y $M\epsilon\in\ringbimod{R}{S'}{\text{Mod}}{lr}$;
	\item las siguientes aplicaciones son morfismos de bimódulos
	\begin{enumerate}[label=(\roman*)]
		\item \begin{align*}
			\descapp{\rho}{\ringbimod{R'}{S}{eM}{lr}}{\ringmodhom{R}{\ringbimod{R}{R'}{Re}{lr}}{\ringbimod{R}{S}{M}{lr}}}{em}{\rho(em)}{,}
			\intertext{con}
			\descapp{\rho\lrprth{em}}{\ringbimod{R}{R'}{Re}{lr}}{\ringbimod{R}{S}{M}{lr}}{re}{\lrprth{re}m}{;}
		\end{align*}
		\item \begin{align*}
			\descapp{\lambda}{\ringbimod{R}{S'}{M\epsilon}{lr}}{\ringmodhom{S}{\ringbimod{S'}{S}{\epsilon S}{lr}}{\ringbimod{R}{S}{M}{lr}}}{m\epsilon}{\lambda(m\epsilon)}{,}
			\intertext{con}
			\descapp{\lambda\lrprth{m\epsilon}}{\ringbimod{S'}{S}{\epsilon S}{lr}}{\ringbimod{R}{S}{M}{lr}}{\epsilon s}{m\epsilon s}{;}
		\end{align*}
	\end{enumerate}
\end{enumerate}
\begin{proof}
	$\boxed{\text{(a)}}$ Por el Ej. 26 $R'$ y $S'$ son anillos. Además, notemos que $R\in\ringbimod{R}{R}{\text{Mod}}{lr}$, a través de las acciones naturales inducidas por el producto en $R$. Así pues, si se considera $M=R=S$ y $e=\epsilon$, se tiene que $Re\in\ringbimod{R}{R'}{\text{Mod}}{lr}$ como consecuencia de que $M\epsilon\in\ringbimod{R}{S'}{\text{Mod}}{lr}$; similarmente  $\epsilon S\in\ringbimod{S'}{S}{\text{Mod}}{lr}$ se sigue de que $eM\in\ringbimod{R'}{S}{\text{Mod}}{lr}$.\\
	Sea $s\in S$. Notemos que 
	\begin{align*}
		ms+(-m)s&=\lrprth{m-m}s=0_Ms=0_M.\\
		&\implies -\lrprth{ms}=(-m)s.
		\intertext{Por lo anterior, si $ms\in Ms$ entonces $-(ms)\in Ms$. Además, si $m's\in Ms$,}
		ms+m's&=\lrprth{m+m'}s\in Ms.\\
		&\implies Ms\leq M.
	\end{align*}
Así, en partícular $M\epsilon$ es un grupo abeliano.\\
	Consideremos las siguientes aplicaciones, denotadas por medio del mismo simbolo para simplificar la notación e inducidas a partir de las acciones como $R$-izquierdo $S$-derecho bimódulo en $M$,
	\begin{align*}
		\descapp{*}{R\times M\epsilon}{M\epsilon}{(r,m\epsilon)}{rm\epsilon}{,}\\
		\descapp{*}{M\epsilon\times S'}{M\epsilon}{(m\epsilon,\epsilon s\epsilon)}{m\epsilon s\epsilon}{.}
	\end{align*}
	Sean $m,m'\in M$, $r,r'\in R$. Entonces
	\begin{align*}
		\lrprth{r+r'}*m\epsilon&=\lrprth{r+r'}m\epsilon=\lrprth{rm+r'm}\epsilon=rm\epsilon+r'm\epsilon\\
		&=r*m\epsilon+r'*m\epsilon.\\
		r*\lrprth{m\epsilon+m'\epsilon}&=r*\lrprth{m+m'}\epsilon=\lrprth{r\lrprth{m+m'}}\epsilon=rm\epsilon+rm'\epsilon\\
		&=r*m\epsilon+r*m'\epsilon.\\
		\intertext{Como $\epsilon^2=\epsilon$}
		\lrprth{rr'}*\lrprth{m\epsilon}&=\lrprth{rr'}m\epsilon=\lrprth{r\lrprth{r'm}}\epsilon\epsilon=r\lrprth{r'm\epsilon}\epsilon\\
		&=r*\lrprth{r'*m\epsilon}.\\
		1_R*\lrprth{m\epsilon}&=\lrprth{1_Rm}\epsilon=m\epsilon.\\
		&\implies M\in\ringmod{R}{\text{Mod}}{l}.
		\intertext{Ahora, sean $s,s'\in S$. Entonces}
		m\epsilon*\lrprth{\epsilon s\epsilon+\epsilon s'\epsilon}&=m\epsilon*\lrprth{\epsilon\lrprth{s+s'}\epsilon}=\lrprth{m\epsilon\lrprth{s+s'}}\epsilon\\&=\lrprth{m\epsilon s}\epsilon+\lrprth{m\epsilon s'}\epsilon
		=m\epsilon*\epsilon s\epsilon+{m\epsilon}*{\epsilon s'\epsilon}.\\
		\lrprth{m\epsilon+m'\epsilon}*{\epsilon s\epsilon}&={\lrprth{m+m'}\epsilon}*{\epsilon s\epsilon}={\lrprth{m+m'}\epsilon s}\epsilon\\&={m\epsilon s}\epsilon+{m'\epsilon s}\epsilon
		=m\epsilon*{\epsilon s\epsilon}+m'\epsilon*{\epsilon s\epsilon}.\\
		\intertext{Como $\epsilon^2=\epsilon$}
		{m\epsilon}*\lrprth{\lrprth{\epsilon s\epsilon}\lrprth{\epsilon s'\epsilon}}&={m\epsilon}*\epsilon \lrprth{s\epsilon s'}\epsilon=\lrprth{m\epsilon\lrprth{s\epsilon s'}}\epsilon=\lrprth{\lrprth{m\epsilon s}\epsilon\epsilon}s\epsilon\\
		&=\lrprth{m\epsilon s\epsilon}s\epsilon=\lrprth{m\epsilon*\epsilon s\epsilon}*\epsilon s'\epsilon.\\
		\lrprth{m\epsilon}*1_{S'}&=\lrprth{m\epsilon}*\epsilon=\lrprth{m\epsilon}\epsilon=m\lrprth{\epsilon\epsilon}\\&=m\epsilon.\\
		&\implies M\in\ringmod{S'}{\text{Mod}}{r}.\\		
	\end{align*}
	Finalmente
\begin{align*}
		{r*\lrprth{{m\epsilon}*{\epsilon s\epsilon}}}&=r*\lrprth{{m\epsilon s}\epsilon}={r\lrprth{m\epsilon s}}\epsilon\\
	&={\lrprth{rm}\epsilon s}\epsilon && M\in\ringbimod{R}{S}{\text{Mod}}{lr}\\
	&={\lrprth{rm}\epsilon}*{\epsilon s\epsilon}\\
	&=\lrprth{r*{m\epsilon}}*{\epsilon s\epsilon}.\\
	&\implies M\epsilon\in\ringbimod{R}{S'}{\text{Mod}}{lr}.
\end{align*}
	En forma análoga a lo desarrollado anteriormente se verifica que, a través de las aplicaciones
	\begin{align*}
		\descapp{\cdot}{R'\times eM}{eM}{(ere,em)}{e{rem}}{,}\\
		\descapp{\cdot}{eM\times S}{eM}{(em,s)}{e{ms}}{,}
	\end{align*}
	$eM\in\ringbimod{R'}{S}{\text{Mod}}{lr}$.\\
	$\boxed{\text{(b)}}$ Sean $a,b,r\in R$. Entonces
	\begin{align*}
		{\rho\lrprth{em}}\lrprth{r\lrprth{ae+be}}&=\lrprth{r\lrprth{ae+be}}m=r\lrprth{\lrprth{ae+be}m}\\
		&=r\lrprth{\lrprth{ae}m+\lrprth{be}m}\\
		&=r\lrprth{\rho\lrprth{em}(ae)+\rho\lrprth{em}(be)}.\\
		&\implies \rho\lrprth{em}\in\ringmodhom{R}{\ringbimod{R}{R'}{Re}{lr}}{\ringbimod{R}{S}{M}{lr}}.
	\end{align*}
	Por lo anterior, y dado que por el Ej. 24  $\ringmodhom{R}{\ringbimod{R}{R'}{Re}{lr}}{\ringbimod{R}{S}{M}{lr}}\in\ringbimod{R'}{S}{\text{Mod}}{lr}$, $\rho$ es una función bien definida. Ahora, sean $m,m'\in M$, $s\in S$ y $x\in R$ y $\bullet$ las acciones definidas en el Ej. 24(a), entonces 
	\begin{align*}
		\rho\lrprth{ere\cdot\lrprth{em+em'}\cdot s}\lrprth{xe}&=\rho\lrprth{{ere}\cdot{e{\lrprth{m+m'}}\cdot s}}(xe)\\
		&=\rho\lrprth{ere\lrprth{m+m'}s}(xe)\\
		&=\lrprth{xe}\lrprth{re\lrprth{m+m'}s}\\
		&=\lrprth{xe}\lrprth{rems+rem's}\\
		&=\lrprth{xe}{rems}+\lrprth{xe}{rem's}\\
		&=\lrprth{\lrprth{xere}m}s+\lrprth{\lrprth{xere}m'}s\\
		&=\lrprth{\rho\lrprth{em}\lrprth{xere}}s+\lrprth{\rho\lrprth{em'}\lrprth{xere}}s\\
		&=\lrprth{\rho\lrprth{em}\lrprth{xe*ere}}s+\lrprth{\rho\lrprth{em'}\lrprth{xe*ere}}s\\
		&=\lrprth{ere\bullet\rho\lrprth{em}\lrprth{xe}}s+\lrprth{ere\bullet\rho\lrprth{em'}\lrprth{xe}}s\\
		&=\lrprth{ere\bullet\rho\lrprth{em}\bullet s}\lrprth{xe}+\lrprth{ere\bullet\rho\lrprth{em'}\bullet s}\lrprth{xe}\\
		&=\lrprth{ere\bullet\rho\lrprth{em}\bullet s+ere\bullet\rho\lrprth{em'}\bullet s}\lrprth{xe}\\
		\implies \rho\lrprth{{ere}\cdot\lrprth{em+em'}\cdot s}&=ere\bullet\rho\lrprth{em}\bullet s+ere\bullet\rho\lrprth{em'}\bullet s\\
		\therefore\  \rho&\in\ringmodhom{}{\ringbimod{R'}{S}{eM}{lr}}{\ringmodhom{R}{\ringbimod{R}{R'}{Re}{lr}}{\ringbimod{R}{S}{M}{lr}}}.
	\end{align*}
i.e. $\rho$ es un morfismo de $R$-izquierda $S'$-derecha bimódulos, de $eM$ en $\ringmodhom{R}{\ringbimod{R}{R'}{Re}{lr}}{\ringbimod{R}{S}{M}{lr}}$. En forma análoga, empleando ahora las acciones previamente definidas en conjunto a las acciones definidas en el Ej. 24(b), se verifica que $\lambda$ un morfismo de $R'$-izquierda $S$-derecha bimódulos, de $M\epsilon$ en $\ringmodhom{S}{\ringbimod{S'}{S}{Re}{lr}}{\ringbimod{R}{S}{M}{lr}}$.\\
\end{proof}
\end{enumerate}
\end{document}