\documentclass{article}
\usepackage[utf8]{inputenc}
\usepackage{mathrsfs}
\usepackage[spanish,es-lcroman]{babel}
\usepackage{amsthm}
\usepackage{amssymb}
\usepackage{enumitem}
\usepackage{graphicx}
\usepackage{caption}
\usepackage{float}
\usepackage{eufrak}
\usepackage{nicefrac}
\usepackage{amsmath,stackengine,scalerel,mathtools}
\usepackage{tikz-cd}
\usepackage{comment}
\def\subnormeq{\mathrel{\scalerel*{\trianglelefteq}{A}}}

\newcommand{\defref}[1]{
	Definición \ref{#1}
}
\newcommand{\crdnlty}[1]{
	\left|#1\right|
}
\newcommand{\lrprth}[1]{
	\left(#1\right)
}
\newcommand{\lrbrack}[1]{
	\left\{#1\right\}
}
\newcommand{\descset}[3]{
	\left\{#1\in#2\ \vline\ #3\right\}
}
\newcommand{\descapp}[6]{
	#1: #2 &\rightarrow #3\\
	#4 &\mapsto #5#6 
}
\newcommand{\arbtfam}[3]{
	{\left\{{#1}_{#2}\right\}}_{#2\in #3}
}
\newcommand{\arbtfmnsub}[3]{
	{\left\{{#1}\right\}}_{#2\in #3}
}
\newcommand{\fntfmnsub}[3]{
	{\left\{{#1}\right\}}_{#2=1}^{#3}
}
\newcommand{\fntfam}[3]{
	{\left\{{#1}_{#2}\right\}}_{#2=1}^{#3}
}
\newcommand{\fntfamsup}[4]{
	\lrbrack{{#1}^{#2}}_{#3=1}^{#4}
}
\newcommand{\arbtuple}[3]{
	{\left({#1}_{#2}\right)}_{#2\in #3}
}
\newcommand{\fntuple}[3]{
	{\left({#1}_{#2}\right)}_{#2=1}^{#3}
}
\newcommand{\gengroup}[1]{
	\left< #1\right>
}
\newcommand{\stblzer}[2]{
	St_{#1}\lrprth{#2}
}
\newcommand{\cmmttr}[1]{
	\left[#1,#1\right]
}
\newcommand{\grpindx}[2]{
	\left[#1:#2\right]
}
\newcommand{\syl}[2]{
	Syl_{#1}\lrprth{#2}
}
\newcommand{\grtcd}[2]{
	mcd\lrprth{#1,#2}
}
\newcommand{\lsttcm}[2]{
	mcm\lrprth{#1,#2}
}
\newcommand{\amntpSyl}[2]{
	\mu_{#1}\lrprth{#2}
}
\newcommand{\gen}[1]{
	gen\lrprth{#1}
}
\newcommand{\ringcenter}[1]{
	C\lrprth{#1}
}
\newcommand{\zend}[2]{
	End_{\mathbb{Z}}^{#2}\lrprth{#1}
}
\newcommand{\genmod}[2]{
	\left< #1\right>_{#2}
}
\newcommand{\genlin}[1]{
	\mathscr{L}\lrprth{#1}
}
\newcommand{\opst}[1]{
	{#1}^{op}
}
\newcommand{\ringmod}[3]{
	\if#3l
	{}_{#1}#2
	\else
	\if#3r
	#2_{#1}
	\fi
	\fi
}
\newcommand{\ringbimod}[4]{
	\if#4l
	{}_{#1-#2}#3
	\else
	\if#4r
	#3_{#1-#2}
	\else 
	\ifstrequal{#4}{lr}{
		{}_{#1}#3_{#2}
	}
	\fi
	\fi
}
\newcommand{\ringmodhom}[3]{
	Hom_{#1}\lrprth{#2,#3}
}
\title{Lista 2}
\author{}
\date{}

\theoremstyle{definition}
\newtheorem{define}{Definición}

\theoremstyle{plain}
\newtheorem{teor}{Teorema}[section]

\theoremstyle{plain}
\newtheorem{prop}{Proposición}[section]

\theoremstyle{definition}
\newtheorem{ejemp}{Ejemplo}[section]

\theoremstyle{definition}
\newtheorem*{coro}{Corolario} 

\theoremstyle{definition}
\newtheorem*{obs}{Observaci\'on}

\theoremstyle{definition}
\newtheorem*{pron}{Proposición}

\theoremstyle{definition}

\newtheorem{lem}{Lema}

\theoremstyle{definition}
\newtheorem*{nota}{Nota}

\begin{document}
	
	\maketitle
\begin{define}
	Sean $R$ y $S$ anillos. Decimos que un grupo abeliano, $M$, es un $R$-derecho y $S$-derecho bimódulo si
	\begin{enumerate}[label=\roman*)]
		\item $M\in\ringmod{R}{\text{Mod}}{r}\cap\ringmod{S}{\text{Mod}}{r}$;
		\item $(mr)s=(ms)r$, $\forall\ r\in R$, $\forall\ s\in S$ y $\forall\ m\in M$.
	\end{enumerate}
	En tal caso denotamos a $M$ como $\ringbimod{R}{S}{M}{r}$.
\end{define}
	\begin{enumerate}[label=\textbf{Ej \arabic*.}]
		\setcounter{enumi}{11}
\item Sea $M\in\ringmod{R}{\text{Mod}}{l}\cap\ringmod{S}{\text{Mod}}{l}$. Entonces $M\in\ringbimod{R}{S}{\text{Mod}}{l}$ si y sólo si $M\in\ringbimod{R}{\opst{S}}{\text{Mod}}{lr}$.
\begin{proof}
	Como $M\in\ringmod{R}{\text{Mod}}{l}\cap\ringmod{S}{\text{Mod}}{l}$ y, por el Ej. 8, $M\in\ringmod{\opst{S}}{\text{Mod}}{r}$, entonces $M\in\ringmod{R}{\text{Mod}}{l}\cap\ringmod{\opst{M}}{\text{Mod}}{r}$.\\ Sean $r\in R$, $s\in S$ y $m\in M$. 
	Dado que, ver Ej 8, $sm=ms^{op}$ entonces 
	\begin{align*}
		r(sm)=s(rm)&\iff r(m\opst{s})=(rm)\opst{s}\\
		\therefore\  M\in\ringbimod{R}{S}{\text{Mod}}{l}&\iff M\in\ringbimod{R}{\opst{S}}{\text{Mod}}{lr}.
	\end{align*}
\end{proof}
\item\item
\begin{define}
	Sean $M\in\ringmod{R}{\text{Mod}}{l}$ y $\arbtfam{X}{i}{I}$ una familia de $R$-submódulos de $M$. Definimos la suma de la familia $\arbtfam{X}{i}{I}$ como
	\begin{equation*}
		\sum_{i\in I}X_i:=\gengroup{\bigcup_{i\in I}X_i}_R.
	\end{equation*}
\end{define}
\item Sean $M\in\ringmod{R}{\text{Mod}}{l}$ y $\arbtfam{X}{i}{I}$ una familia de $R$-submódulos de $M$. Entonces
\begin{enumerate}[label=(\alph*)]
	\item \begin{equation*}
		\sum_{i\in I}X_i=\left\{\begin{tabular}{cc}
			$\lrbrack{0}$ &, $I=\varnothing$  \\
			\(\displaystyle
			\lrbrack{\sum_{j\in\lrbrack{i_1,\dotsc,i_n}\subseteq I}x_j\ \vline\ x_j\in X_j,\ n\in\mathbb{N}\setminus\lrbrack{0}}\) & , $I\neq\varnothing$
		\end{tabular}\right.
	\end{equation*}
	\item $\lrbrack{\genlin{M},\leq}$ es un reticulado completo. Más aún, si $\arbtfam{X}{i}{I}$ es una familia no vacía de $R$-submódulos de $M$,
	\begin{align*}
		sup\arbtfam{X}{i}{I}&=\sum_{i\in I}X_i,\\
		inf\arbtfam{X}{i}{I}&=\bigcap_{i\in I}X_i.
	\end{align*}
\end{enumerate}
\begin{proof}
	Verifiquemos primeramente el siguiente lema:
	\begin{lem}
		Sea $M\in\ringmod{R}{\text{Mod}}{l}$. Si $\mathcal{A}\subseteq\genlin{M}$ entonces $\bigcap \mathcal{A}\in\genlin{M}$.
	\end{lem}
	\begin{proof}
		Notemos que 
		\begin{align*}
			0+0&=0;\\
			r\bullet 0&=r\bullet(0+0)=r\bullet 0+r\bullet 0\\
			&\implies r\bullet 0=0,\ \forall\ r\in R.\\
			\implies &\lrbrack{0}\in\genlin{M}.
			\intertext{Además}
			0_R\bullet x&=\lrprth{0_R+0_R}\bullet x=0_R\bullet x+0_R\bullet x\\
			&\implies 0_R\bullet x=0,\ \forall\ x\in M.
			\intertext{Por lo anterior, y dado que si $X\in\genlin{M}$ entonces $X\neq\varnothing$, se tiene que}
			&\lrbrack{0}\subseteq X,\ \forall\ X\in \genlin{M}.
		\end{align*}
		Con lo cual $\bigcap \mathcal{A}\neq\varnothing$, pues $\lrbrack{0}\subseteq \bigcap \mathcal{A}$. Sean $r\in R,\ a,b\in \bigcap\mathcal{A}$ y $A\in\mathcal{A}$. Como $A\leq M$
		\begin{align*}
			ra,a+b&\in A\\
			\implies ra,a+b&\in A, \forall\ A\in\mathcal{A}\\
			\implies ra,a+b&\in A, \forall\ \bigcap\mathcal{A}\\
			&\implies \bigcap\mathcal{A}\in\genlin{M}.
		\end{align*}
	\end{proof}
	Por el lema anterior el submódulo generado por un conjunto $A\supseteq X$ está bien definido y, más aún, es el mínimo submódulo de $M$, con respecto a $\subseteq$, que contiene a $A$.
	$\boxed{\text{(a)}}$
	Supongamos que $I=\varnothing$, entonces $\bigcup_{i\in I}X_i=\varnothing$, y así 
	\begin{align*}
		\sum_{i\in I}X_i&=\bigcap\descset{X}{\genlin{M}}{\varnothing\subseteq X}\\
		&=\bigcap\genlin{M}.
		\intertext{Del lema se tiene que $0\in X,\ \forall\ X\in \genlin{M}$ y que $\lrbrack{0}\in\genlin{M}$, con lo cual}
		\lrbrack{0}&\subseteq\bigcap_{X\in\genlin{M}}X=\bigcap\genlin{M}\subseteq\lrbrack{0}\\
		&\therefore\ \sum_{i\in I}X_i=\lrbrack{0}.
	\end{align*}
	Supongamos ahora que $I\neq\varnothing$. Si 
	\begin{equation*}
		S:=\lrbrack{\sum_{j\in\lrbrack{i_1,\dotsc,i_n}\subseteq I}x_j\ \vline\ x_j\in X_j,\ n\in\mathbb{N}\setminus\lrbrack{0}}
	\end{equation*}
	afirmamos que $S\in\genlin{M}$. En efecto:\\
	Como $I\neq\varnothing$ y $X_i\neq\varnothing$, $\forall\ i\in I$, entonces $S\neq\varnothing$. Sean $r\in R$ y $a,b\in S$, luego $\exists\ n,m\in\mathbb{N}$ tales que
	\begin{align*}
		a&=\sum_{j\in\lrbrack{i_1,\dotsc,i_n}}x_j\\
		b&=\sum_{j\in\lrbrack{k_1,\dotsc,k_m}}y_j.
		\intertext{En caso que $\lrbrack{i_1,\dotsc,i_n}= \lrbrack{k_1,\dotsc,k_n}$}
		a+b&=\sum_{j\in\lrbrack{i_1,\dotsc,i_n}}\lrprth{x_j+y_j}
		\intertext{Como $X_j\leq M$, $x_j+y_j\in X_j$, $\forall\ j\in\lrbrack{i_1,\dotsc,i_n}$; luego $a+b\in S$. Si ahora $\lrbrack{i_1,\dotsc,i_n}\cap
			\lrbrack{k_1,\dotsc,k_n}=\varnothing$ consideremos}
		l_r&:=i_r,\ \forall r\in[1,n]\\
		l_{n+r}&:=k_r,\ \forall r\in[1,m].
		\intertext{Así}
		a+b&=\sum_{j\in\lrbrack{l_1,\dotsc,l_{n+m}}}z_j\\
		&\implies a+b\in S.
		\intertext{Finalmente, reetiquetando de ser necesario, si}
		A&:=\lrbrack{i_1,\dotsc,i_n}\\
		B&:=\lrbrack{k_1,\dotsc,k_m}\\
		D&:=\lrbrack{i_1,\dotsc,i_n}\cap
		\lrbrack{k_1,\dotsc,k_n}\\
		E&:=\lrbrack{i_1,\dotsc,i_n}\cup
		\lrbrack{k_1,\dotsc,k_n}=\lrbrack{l_1,\dotsc,l_{r},l_{r+1},\dotsc,l_{t}},\\
		\intertext{con $\crdnlty{D}=r>0$, $\crdnlty{E\setminus D}=t>0$, entonces}
		a+b&=\sum_{j\in D}\lrprth{x_j+y_j}+\sum_{j\in A\setminus D}x_j+\sum_{j\in B\setminus D}y_j \\
		&=\sum_{j\in E}z_j\\
		&\implies a+b\in S.
	\end{align*}
	Por otro lado
	\begin{align*}
		r\bullet a&=r\bullet\sum_{j\in A}x_j=\sum_{j\in A}r\bullet x_j.
	\end{align*}
	De modo que $r\bullet a\in S$, pues $A\subseteq I$ es finito y, como $X_j\leq M$, $r\bullet x_j\in X_j$, $\forall\ j\in A$; y por lo tanto $S\leq M$. \\
	Como $\lrbrack{i}\subseteq I$, $\forall\ i\in I$, entonces
	$\bigcup_{i\in I}X_i\subseteq S$. De modo que
	\begin{equation*}
		\sum_{i\in I}X_i=\gengroup{\bigcup_{i\in I}X_i}_R\subseteq S.
	\end{equation*}
	Ahora si $Y\leq M$ es tal que $Y\supseteq \bigcup_{i\in I}X_i$, $J:=\lrbrack{i_1,\dotsc,i_n}\subseteq I$ y $a_j\in X_j\subseteq\bigcup_{i\in I}X_i,\ \forall\ j\in J$, entonces
	\begin{align*}
		\sum_{j\in J}a_j&\in Y\\
		\implies S&\subseteq Y,\ \forall\ Y\in\genlin{M}\text{ tal que } Y\supseteq \bigcup_{i\in I}X_i\\
		\implies S&\subseteq \gengroup{\bigcup_{i\in I}X_i}_R=\sum_{i\in I}X_i\\
		&\therefore\ S=\sum_{i\in I}X_i.
	\end{align*}
	$\boxed{\text{(b)}}$ El par $(\genlin{M},\leq)$ es un CPO puesto que la relación $\subseteq$ es un orden parcial.\\
	Sea $C\leq M$ cota superior de $S$. Entonces $X_i\leq C$, $\forall\ i\in I$; luego $C\subseteq \bigcup_{i\in I}X_i$. Dado $\sum_{i\in I}X_i$ es el mínimo submódulo, con respecto a $\subseteq$, que contiene a $\bigcup_{i\in I}X_i$ se tiene que $\sum_{i\in I}X_i\leq C$ y por lo tanto $sup\lrprth{S}=\sum_{i\in I}X_i$.\\
	Sea $C\leq M$ cota inferior de $S$. Entonces $C\leq X_i$, $\forall\ i\in I$; luego $C\supseteq \bigcap_{i\in I}X_i$. y así $\bigcap_{i\in I}X_i\leq C$. Por lo tanto $inf\lrprth{S}=\bigcap_{i\in I}X_i$.\\
	\begin{equation*}
		\therefore\ \lrprth{\genlin{M},\leq}\text{ es un reticulado completo.}
	\end{equation*}
\end{proof}
\item\item
\item Sea $M\in\ringmod{R}{\text{Mod}}{l}$. Las siguientes condiciones son equivalentes:
\begin{enumerate}[label=(\alph*)]
	\item $M$ es finitamente generado.
	\item $\exists\ n\in\mathbb{N}\setminus\lrbrack{0}$ y $f:\mathbb{R}^n\rightarrow M$ epimorfismo de $R$-módulos.
\end{enumerate}
\begin{proof}
	k
\end{proof}
\item\item
\item Sean $R$ y $S$ anillos.
\begin{enumerate}[label=(\alph*)]
	\item Sean 
	\begin{align*}
		Obj\lrprth{\mathcal{C}}&:=\ringbimod{R}{S}{\text{Mod}}{r},\\
		Hom\lrprth{\mathcal{C}}&:=\bigcup_{\lrprth{M,N}\in Obj\lrprth{\mathcal{C}}^2}Hom\lrprth{\ringbimod{R}{S}{M}{r},\ringbimod{R}{S}{N}{r}},\\
		\intertext{con}
		Hom\lrprth{\ringbimod{R}{S}{M}{r},\ringbimod{R}{S}{N}{r}}&:=Hom_R\lrprth{\ringmod{R}{M}{r},\ringmod{R}{N}{r}}\cap Hom_S\lrprth{\ringmod{S}{M}{r},\ringmod{S}{N}{r}},
	\end{align*}
	y $\circ$ la composición usual de funciones.\\ Entonces la clase $\ringbimod{R}{S}{\text{Mod}}{r}$ tiene estructura de categería a por medio de la tercia $\lrprth{Obj\lrprth{\mathcal{C}},Hom\lrprth{\mathcal{C}},\circ}$.
	\item $\ringbimod{R}{S}{\text{Mod}}{lr}\simeq\ringbimod{\opst{R}}{S}{\text{Mod}}{r}$, $\ringbimod{R}{S}{\text{Mod}}{lr}\simeq\ringbimod{R}{\opst{S}}{\text{Mod}}{l}$ y $\ringbimod{R}{S}{\text{Mod}}{l}\simeq\ringbimod{R}{\opst{S}}{\text{Mod}}{lr}$.
\end{enumerate}
\begin{proof}
	s
\end{proof}
\item\item
\item Sean $R$, $S$ y $T$ anillos.
\begin{enumerate}[label=(\alph*)]
	\item Sean $M\in\ringbimod{R}{S}{\text{Mod}}{lr}$, $N\in\ringbimod{R}{T}{\text{Mod}}{lr}$, $H:=\ringmodhom{R}{\ringbimod{R}{S}{M}{lr}}{\ringbimod{R}{T}{N}{lr}}$ y las siguientes aplicaciones, denotadas por medio del mismo simbolo para simplificar la notación,
	\begin{align*}
		\descapp{\bullet}{S\times H}{H}{(s,f)}{s\bullet f}{,}
		\intertext{con}
		\descapp{s\bullet f}{M}{N}{x}{f(xs)}{;}\\
		\descapp{\bullet}{H\times T}{H}{(f,t)}{f\bullet t}{,}
		\intertext{con}
		\descapp{f\bullet t}{M}{N}{x}{f(x)t}{.}
	\end{align*}
	A través de las aplicaciones anteriores  $H\in\ringbimod{S}{T}{\text{Mod}}{lr}$.
	\item Sean $M\in\ringbimod{S}{R}{\text{Mod}}{lr}$, $N\in\ringbimod{T}{R}{\text{Mod}}{lr}$, $H:=\ringmodhom{R}{\ringbimod{S}{R}{M}{lr}}{\ringbimod{T}{R}{N}{lr}}$ y las siguientes aplicaciones, denotadas por medio del mismo simbolo para simplificar la notación,
	\begin{align*}
		\descapp{\bullet}{T\times H}{H}{(t,f)}{t\bullet f}{,}
		\intertext{con}
		\descapp{t\bullet f}{M}{N}{x}{tf(x)}{;}\\
		\descapp{\bullet}{H\times S}{H}{(f,s)}{f\bullet s}{,}
		\intertext{con}
		\descapp{f\bullet s}{M}{N}{x}{f(sx)}{.}
	\end{align*}
	A través de las aplicaciones anteriores  	$H\in\ringbimod{T}{S}{\text{Mod}}{lr}$.
\end{enumerate}
\item\item
\item Sean $R$ y $S$ anillos, $e\in R$ y $\epsilon\in S$ idempotentes, $M\in\ringbimod{R}{S}{\text{Mod}}{lr}$, $R':=eRe$ y $S':=\epsilon R\epsilon$. Entonces:
\begin{enumerate}[label=(\alph*)]
	\item existen acciones tales que $Re\in\ringbimod{R}{R'}{\text{Mod}}{lr}$, $\epsilon S\in\ringbimod{S'}{S}{\text{Mod}}{lr}$, $eM\in\ringbimod{R'}{S}{\text{Mod}}{lr}$ y $M\epsilon\in\ringbimod{R}{S'}{\text{Mod}}{lr}$;
	\item las siguientes aplicaciones son morfismos de bimódulos
	\begin{enumerate}[label=(\roman*)]
		\item \begin{align*}
			\descapp{\rho}{\ringbimod{R'}{S}{eM}{lr}}{\ringmodhom{R}{\ringbimod{R}{R'}{Re}{lr}}{\ringbimod{R}{S}{M}{lr}}}{em}{\rho(em)}{,}
			\intertext{con}
			\descapp{\rho\lrprth{em}}{\ringbimod{R}{R'}{Re}{lr}}{\ringbimod{R}{S}{M}{lr}}{re}{rem}{;}
		\end{align*}
		\item \begin{align*}
			\descapp{\lambda}{\ringbimod{R}{S'}{M\epsilon}{lr}}{\ringmodhom{S}{\ringbimod{S'}{S}{\epsilon S}{lr}}{\ringbimod{R}{S}{M}{lr}}}{m\epsilon}{\lambda(m\epsilon)}{,}
			\intertext{con}
			\descapp{\lambda\lrprth{m\epsilon}}{\ringbimod{S'}{S}{\epsilon S}{lr}}{\ringbimod{R}{S}{M}{lr}}{\epsilon s}{m\epsilon s}{;}
		\end{align*}
	\end{enumerate}
\end{enumerate}

\end{enumerate}
\end{document}