\documentclass{article}
\usepackage[utf8]{inputenc}
\usepackage{mathrsfs}
\usepackage[spanish,es-lcroman]{babel}
\usepackage{amsthm}
\usepackage{amssymb}
\usepackage{enumitem}
\usepackage{graphicx}
\usepackage{caption}
\usepackage{float}
\usepackage{eufrak}
\usepackage{nicefrac}
\usepackage{amsmath,stackengine,scalerel,mathtools}
\usepackage{tikz-cd}
\usepackage{comment}
\usepackage{amsmath}
\usepackage{faktor}
\newcommand{\Z}{\mathbb{Z}}
\newcommand{\La}{\mathscr{L}}


\def\subnormeq{\mathrel{\scalerel*{\trianglelefteq}{A}}}

\newcommand{\defref}[1]{
	Definición \ref{#1}
}
\newcommand{\crdnlty}[1]{
	\left|#1\right|
}
\newcommand{\lrprth}[1]{
	\left(#1\right)
}
\newcommand{\lrbrack}[1]{
	\left\{#1\right\}
}
\newcommand{\descset}[3]{
	\left\{#1\in#2\ \vline\ #3\right\}
}
\newcommand{\descapp}[6]{
	#1: #2 &\rightarrow #3\\
	#4 &\mapsto #5#6 
}
\newcommand{\arbtfam}[3]{
	{\left\{{#1}_{#2}\right\}}_{#2\in #3}
}
\newcommand{\arbtfmnsub}[3]{
	{\left\{{#1}\right\}}_{#2\in #3}
}
\newcommand{\fntfmnsub}[3]{
	{\left\{{#1}\right\}}_{#2=1}^{#3}
}
\newcommand{\fntfam}[3]{
	{\left\{{#1}_{#2}\right\}}_{#2=1}^{#3}
}
\newcommand{\fntfamsup}[4]{
	\lrbrack{{#1}^{#2}}_{#3=1}^{#4}
}
\newcommand{\arbtuple}[3]{
	{\left({#1}_{#2}\right)}_{#2\in #3}
}
\newcommand{\fntuple}[3]{
	{\left({#1}_{#2}\right)}_{#2=1}^{#3}
}
\newcommand{\gengroup}[1]{
	\left< #1\right>
}
\newcommand{\stblzer}[2]{
	St_{#1}\lrprth{#2}
}
\newcommand{\cmmttr}[1]{
	\left[#1,#1\right]
}
\newcommand{\grpindx}[2]{
	\left[#1:#2\right]
}
\newcommand{\syl}[2]{
	Syl_{#1}\lrprth{#2}
}
\newcommand{\grtcd}[2]{
	mcd\lrprth{#1,#2}
}
\newcommand{\lsttcm}[2]{
	mcm\lrprth{#1,#2}
}
\newcommand{\amntpSyl}[2]{
	\mu_{#1}\lrprth{#2}
}
\newcommand{\gen}[1]{
	gen\lrprth{#1}
}
\newcommand{\ringcenter}[1]{
	C\lrprth{#1}
}
\newcommand{\zend}[2]{
	End_{\mathbb{Z}}^{#2}\lrprth{#1}
}
\newcommand{\genmod}[2]{
	\left< #1\right>_{#2}
}
\newcommand{\genlin}[1]{
	\mathscr{L}\lrprth{#1}
}
\newcommand{\opst}[1]{
	{#1}^{op}
}
\newcommand{\ringmod}[3]{
	\if#3l
	{}_{#1}#2
	\else
	\if#3r
	#2_{#1}
	\fi
	\fi
}
\newcommand{\ringbimod}[4]{
	\if#4l
	{}_{#1-#2}#3
	\else
	\if#4r
	#3_{#1-#2}
	\else 
	\ifstrequal{#4}{lr}{
		{}_{#1}#3_{#2}
	}
	\fi
	\fi
}
\newcommand{\ringmodhom}[3]{
	Hom_{#1}\lrprth{#2,#3}
}
\title{Tarea 2}
\author{Sergio R.Z.}
\date{}

\theoremstyle{definition}
\newtheorem{define}{Definición}

\theoremstyle{plain}
\newtheorem{teor}{Teorema}[section]

\theoremstyle{plain}
\newtheorem{prop}{Proposición}[section]

\theoremstyle{definition}
\newtheorem{ejemp}{Ejemplo}[section]

\theoremstyle{definition}
\newtheorem*{coro}{Corolario} 

\theoremstyle{definition}
\newtheorem*{obs}{Observaci\'on}

\theoremstyle{definition}
\newtheorem*{pron}{Proposición}

\theoremstyle{definition}

\newtheorem{lem}{Lema}

\theoremstyle{definition}
\newtheorem*{nota}{Nota}

\begin{document}

	\maketitle
\begin{enumerate}[label=\textbf{Ej \arabic*.}]
		\setcounter{enumi}{11}
 \item
 \item
 \item Sean $X,M\in \prescript{}{R}{Mod}$\,\, tal que\,\, $X\subseteq M$. Pruebe que $X\leq M$ $\iff$ la inclusión $i_X:X\longrightarrow M,
 i_X(x):=x\quad \forall x\in X,$ es un morfismo de $R$ módulos.
 
 \begin{proof}
 \boxed{\Rightarrow} Supongamos que $X\leq M$, entonces dados $x,y\in X$ y $r\in R$ se tiene que 
 \[i_x(rx+y)=rx+y =ri_X(x)+i_X(y). \]
 Por lo que $i_X$ es morfismo.\\
 
 \boxed{\Leftarrow} Ahora supongamos que  $i_X:X\longrightarrow M$ es un morfismo de $R$ módulos.\\
 
 Sean $x,y\in X$ y $r\in R$, como $X$ es un $R$ módulo a izquierda entonces $x+y\in X$ y como $i_X$ es morfismo
 se tiene que, si $\cdot: R\times X\longrightarrow X$ es la acción de $R$ módulo en $X$, entonces 
 $r\cdot x = r\cdot i_X(x) = i_X(rx)=rx $. Así, como $X\subset M$, entonces $X\leq M$.
  \end{proof}
 
 \item
 \item
 \item Para un $M\in \prescript{}{R}{Mod}$, pruebe que las siguientes condiciones son equivalentes.
 \begin{itemize}
    \item[a)]  $M$ es simple.
    \item[b)]  $0$ es un submódulo maximal de $M$.
    \item[c)]  $M$ es un submódulo minimal de $M$.
    \item[d)]  $M\neq 0$ y $M=<m>_R\quad \forall m\in M-\{0\}$.
    \item[e)]  $M\neq 0$ y $\forall X\in \prescript{}{R}{Mod}, \,\forall f\in Hom_R(M,X)$ se tiene que $f=0$ o bien 
    $Ker(f)=0$.
    \item[f)]  $M\neq 0$ y $\forall X\in \prescript{}{R}{Mod}, \,\forall g\in Hom_R(X,M)$ se tiene que $g=0$ o bien 
 $Im(g)=0$..
 \end{itemize}
 
 \begin{proof}
Notemos que si $M\neq 0$, ningúna de los incisos se satisface, así que podemos tomar $M\neq 0$.\\

 \boxed{a)\Rightarrow b)} Supongamos $M$ es simple, entonces $\La(M)=\{0,M\}$ por lo que $0$ es el único
 submódulo de $M$ propio y por lo tanto es maximal.\\

 \boxed{b)\Rightarrow c)} Supongamos $0$ es un submódulo maximal de $M$, como \\ $\forall N\in \La(M)-\{M\}$ se tiene 
 que $0\leq  N$, entonces \\ $\forall N\in\La(M)-\{M\},\quad N=0$, por lo que $M$ es minimal al ser el único submódulo en 
 $\left(\La(M)-\{\,0\,\},\leq \right)$ tal que $M\neq 0$ y si $0\lneq N$ entonces \quad $\left(N\leq M\,\,\Rightarrow N=M\right)$.\\
 
 \boxed{c)\Rightarrow d)}Supongamos $M$ es un submódulo minimal de $M$. Por definición $M\neq 0$ entonces 
 $\forall m\in M-\{0\},$ pasa que $<m>_R\neq 0$, pero \\
 $<m>_R\in \La(M)$, entonces $<m>_R\geq M$ por ser $M$ minimal.
 Sin embargo $<m>_R\leq M$ pues $m\in M$, por lo tanto $M=<m>_R\quad \forall m\in M-\{0\}$.\\
 
 \boxed{d)\Rightarrow a)} Como $M\neq 0$, si $N\neq 0$ y $N\in \La(M)$, entonces $N\subset M$,
  $\forall n\in N-\{0\}\quad n\in M-\{0\}$ y $M=<n>_M\leq N\leq M$. Por lo tanto $N=M$ y $\La(M)=\{0,M\}$.\\
  
 Con lo anterior tenemos que las primeras cuatro proposiciones son equivalentes, entonces para terminar se demostrarán 
 las siguientes equivalencias.\\
 
 \boxed{a)\Rightarrow e)} Ya sabemos que $M\neq 0$. Sea $f\in Hom_R(M,X)$ con $X\in \prescript{}{R}{Mod}$.\\
 Si $f=0$ no hay nada que demostrar. Supongamos $f\neq 0$, como \\
 $Ker(f)\leq M$ con $M$ simple, entonces 
 $Ker(f)=0$ o $Ker(f)=M$, pero $f\neq 0$, entonces $Ker(f)\neq M$ y en consecuencia $Ker(f)=0$.\\
 
 \boxed{e)\Rightarrow f)} Ya sabemos que $M\neq 0$. Sea $g\in Hom_R(X,M)$ con $X\in \prescript{}{R}{Mod}$.\\
 Si $g=0$ no hay nada que probar. Supongamos $g\neq 0$, como $Im(g)\leq M$ entonces podemos tomar 
 $\faktor{M}{Im(g)} \in \prescript{}{R}{Mod}$ y así \\
 $\pi_{IM(g)}\in Hom_R\left(M,\faktor{M}{Im(g)}\right)$, con $\pi_{Im(g)}\neq 0 $. Entonces por e) tenemos  que, $Ker(g)=0$, y así $Im(g)=M$.\\
 
 \boxed{f)\Rightarrow a)} Como $M\neq 0$ tenemos que para cada $ N\in \La(M)-\{0\}$ la inclusión $i_N:N\longrightarrow M$
  es morfismo y más aun $i_N\neq 0$. Entonces por f) se tiene que $N=Im(i_N)=M$ y así $\La(M)=\{0,M\}$. 
 \end{proof}
 
\item
\item
\item Para una $K$-álgebra $R$, defina de manera natural una estructura de \\ $K$-módulo (a izquierda y a derecha)
 en $Hom_R(M,N)\quad \forall M,N\in Mod(R).$
\begin{proof}
 Sean $M$ y $N$ $R$-módulos a izquierda y $(R,K,\varphi)$ una \\
 $K$-álgebra. Para toda $k\in K,\,\, f\in Hom_R(M,N),\,\, l\in R$ y $m\in M$, definiremos 
 $\alpha:K\times Hom_R(M,N)\longrightarrow Hom_R(M,N)$ dada por 
\[\alpha(k,f)(\varphi)(lm)=k\cdot l(\varphi)(lm):=l*(\varphi(k)*f(m)),\]
 donde $\cdot$ es la acción a izquierda de $N$ como $R$-módulo.\\
 Así \\
 \boxed{AC1} 
 \begin{align*}
( k\cdot(f_1+f_2))(lm)
 &=l*(\varphi(k)*(f_1+f_2)(m))\\
 &=l*(\varphi(k)*(f_1)(m)+\varphi(k)*(f_2)(m))\\
 &=l*(\varphi(k)*(f_1)(m))+l*(\varphi(k)*(f_2)(m))\\
 &=k\cdot(f_1)(lm)+k\cdot(f_2)(lm). 
  \end{align*}
 \boxed{AC2}
  \begin{align*}
 ((k_1+k_2)\cdot(f))(lm)
 &=l*(\varphi(k_1+k_2)*f(m))\\
 &=l*(\varphi(k_1)*f(m)+\varphi(k_2)*f(m))\\
 &=l*(\varphi(k_1)*f(m))+l*(\varphi(k_2)*f(m))\\
 &=k_1\cdot f(lm)+k_2\cdot f(lm).
  \end{align*}
 \boxed{AC3}
\begin{align*}
(1_K\cdot f)(lm)&=l*(\varphi(1_K)*f(m))\\
&=l*(1_R*f(m))\\
&=l*(f(m))\\
&=f(lm).
\end{align*}
 \boxed{AC4}
\begin{align*}
((k_1k_2)\cdot f)(lm)=&l*(\varphi(k_1k_2)*f(m))\\
&=l*(\varphi(k_1)\varphi(k_2)*f(m))\\
&=l*(\varphi(k_1)*(\varphi(k_2)*f(m)))\\
&=l*[\varphi(k_1)*(k_2\cdot f)(m)]\\
&=(k_1\cdot(k_2\cdot f))(lm).
\end{align*}

Entonces $\cdot$ es una acción a izquierda de $R$-módulos.\\

Por el ejercicio 8, se tiene que si $M$ y $N$ son $R$-módulos derechos entonces son $R^{op}$ módulos izquierdos.Asi
$\cdot$ es  una acción para $Hom_{R^{op}}(M,N)$ y por el ejercicio 8, $\cdot^{op}$ es una acción que vuelve a 
$Hom_R(M,N)$ un módulo derecho.
\end{proof}

\item
\item
\item Sea $R$ un anillo e $I\lhd R$. Considere el epimorfismo canónico de anillos $\pi:R\longrightarrow \faktor{R}{I}$,\,\,$r\mapsto r+I$.
\begin{itemize}
\item[a)] Pruebe que el funtor de cambio de anillos \\
$F_\pi:Mod\left(\faktor{R}{I}\right)\longrightarrow Mod(R)$ es fiel y pleno.
\item[b)] Sea $\zeta_I$ la subcategoría plena de $Mod(R)$ cuyos objetos son los \\
$M\in Mod(R)$ que son aniquilados por $I$, i.e. 
\[M\in \zeta_I\,\,\iff\,\,I\subseteq ann_R(M):=\{r\in R\,:\, rm=0\,\,\forall m\in M\}.\]
Pruebe que $F_\pi\left(Mod\left(\faktor{R}{I}\right)\right)=\zeta_I$ y que $F_\pi:Mod\left(\faktor{R}{I}\right)\longrightarrow \zeta_I$ 
es un isomorfismo de categorías.
\end{itemize}
\begin{proof}

\boxed{a)} Sean $A,B\in Mod(R)$. Por el ejercicio 22 a), F es un funtor, y como $\pi(R)=R/I$, entonces  por el ejercicio 22 b), $F_\pi$ es fiel y pleno.\\

\boxed{b)} Sea $A\in Mod\left(\faktor{R}{I}\right)$, entonces tenemos que $F_\pi(A)\in Mod(R)$. Ahora, si $m\in F_\pi(A)$, entonces para cada $x\in I,
 \,\,\,x\cdot m=\pi(X)*m=0*m=0$, donde $\cdot$ y $*$ son las acciones definidas en $A$ y $F_\pi(A)$ respectivamente.\\
Por lo tanto $I\subseteq ann_R(M)$ y $F_\pi(A)\in \zeta_I.$\\

Ahora, sea $A\in \zeta_I$, entonces $I\subseteq ann_R(A):=\{r\in R\,:\,rm=0\,\,\forall m\in A\}$, en particular $A\in Mod(R)$, así 
definimos la función $*:\faktor{R}{I}\times A\longrightarrow A$ dada por $[r]*m=(r+I)*m:=r\cdot m+I\cdot m$ donde $\cdot$ es la 
acción de $A$ como $R$-módulo, y se tiene que, como $I\cdot m=\{im\,:\,i\in I,\, m\in A\}$ y $A\in \zeta_I$, $I\cdot m=\{0\}.$\\

Por lo tanto, si $r,s\in [x]$ con $\, r=x+k_1$ y $s=x+k_2$, entonces $r*m=s*m,$ en particular $[r]*m=r\cdot m\,\,\forall r\in R$, es decir,
$*$ es una acción de $A$ (bien definida) como $\faktor{R}{I}$ -módulo, es un $R$-módulo y $[r]*m=r\cdot m\,\,\forall r\in R$,
es decir, $A\in F_\pi\left(Mod\left(\faktor{R}{I}\right)\right)$. Y en consecuencia $Mod\left(\faktor{R}{I}\right)=\zeta_I$\\

Con esto podemos ver que $F_\pi:Mod\left(\faktor{R}{I}\right)\longrightarrow\zeta_I$ es funtor, y más aun, definiendo 
$G_\pi:\zeta_I\longrightarrow Mod\left(\faktor{R}{I}\right)$ dado por $G_\pi(M)=M$ para cada $M\in \zeta_I$ y $G_\pi(f)([r]*_{{}_M} m)
 =[r]*_{{}_N}f(m)$ para cada $f\in Hom(M,N)$ donde $*$ es la acción a izquierda de $M$ como $\faktor{R}{I}$ -módulo,
es un funtor pues $G_\pi(1_M)([r]*_{{}_M}m)=[r]*_{{}_M}m=1_M([r]*_{{}_M}m)$ y para $ f\in Hom_R(M,N), g\in Hom_R(N,L)$
\begin{gather*}
 G_\pi(gf)([r]*_{{}_M}m)=[r]*_{{}_L}gf(m)=G_\pi(g)\left([r]*_{{}_N}f(m)\right)\\
 =\left(G_\pi(g)\circ G_\pi(f)\right)([r]*_{{}_M}m).
\end{gather*}

Así $F_\pi\circ G_\pi(A)=A=G_\pi\circ F_\pi(A)\quad \forall A\in Mod\left(\faktor{R}{I}\right)=\zeta_I$, y cumple las siguientes condiciones
\begin{itemize}
\item[i)] $F_\pi\circ G_\pi(1_A)=F_\pi(1_A)=1_A=G_\pi(1_A)=G_\pi\circ F_\pi(1_A)$.

\item[ii)] Considerando a $\cdot$ y $*$ como las acciones a izquierda como $R$ -módulo y $\faktor{R}{I}$ -módulo respectivamente.
Para cualquier $r\in R, m\in M,\\ f\in Hom_R(M,N)$ y $g\in Hom_{\faktor{R}{I}}(M,N)$, se tiene lo siguiente.
\end{itemize}
\[\left(F\pi\circ G_\pi(f)\right)(r\cdot_{{}_M}m)=[r] *_{{}_N} G_\pi(f)(m)=[r] *_{{}_N}f(m)=r\cdot_{{}_N}f(m)=f(r\cdot_{{}_M}m).\]

\[\left(G\pi\circ F_\pi(g)\right)([r]*_{{}_M}m)=r \cdot_{{}_N} F_\pi(g)(m)=r\cdot_{{}_N}g(m)=[r]*_{{}_N}g(m)=g([r]*_{{}_M}m).\]

Por lo tanto $F\pi\circ G_\pi(f)=1_{\zeta_I}$\quad y \quad $G\pi\circ F_\pi(g)=1_{Mod\left(\faktor{R}{I}\right)}$, es decir, $F_\pi$ es
un isomorfismo de categorías.

\end{proof}


\item
\item
\item Sea $R$ un anillo y $e^2=e\in R$, un idempotente en $R$.
\begin{itemize}
\item[a)] Pruebe que la estructura de anillo en $R$ induce, de manera natural, una estructura de anillo en $eRe:=\{ere\,\colon\,r\in R\}$.
¿Es $eRe$ un subanilo de $R$?
\item[b)] Para cada $M\in Mod(R)$, pruebe que la acción a izquierda \\
$eRe\times eM\longrightarrow eM,$ \,\, $(ere,em)\mapsto erem$ 
induce una estructura de $eRe$-módulo a izquierda en $eM$.
\item[c)] Pruebe que la correspondencia $m_e:Mod(R)\longrightarrow Mod(eRe)$ , dada por 
$\left(M \stackrel{f}{\longrightarrow} N\right)\longmapsto \left(eM\stackrel{f|_{eM}}{\longrightarrow} eN\right)$, es funtorial.
\end{itemize}
\begin{proof}
\boxed{a)} Como $R$ es anillo, se tienen los siguientes resultados: $\forall a,b\in R$
\[eae-ebe=e(ae-be)=e(a-b)e\in eRe\]
(por lo que $eRe$ es subgrupo abeliano bajo la suma),
\[(eae)(ebe)=eae^2be=e(aeb)e\in eRe
\]
y 
\[e(eae)=e^2ae=eae=eae^2=(eae)e.
\]
Por lo tanto $R$ es un anillo con $e$ su inverso, y por esto mismo no puede ser subanillo de $R$ exepto en el caso en que $e=1_R$. \\

\boxed{b)} Observemos que $\forall a\in eM$ y $\forall s\in eRe$.
\[a=em\quad \text{y} \quad s=ere\quad \text{para alguna} \quad m\in M\quad \text{y} \quad r\in R,\]
así, llamando $*$ a la acción que definida como $(ere,em)\mapsto erem$, entonces 
\[s*a=(ere)(em)=erem=ere^2m=(ere)(em).
\]
Por lo que 
\begin{align*}
\boxed{i)}\,\,(er_1e+er_2e)*(em)&=(er_1e+er_2e)(em)\\
&=er_1eem+er_2eem\\
&=(er_1e)*(em)+(er_2e)*(em).
\end{align*}

\begin{align*}
\boxed{ii)}\,\, (ere)*(em_1+em_2)&= (ere)(em_1+em_2)\\
&=ereem_1+ereem_2\\
&=(ere)*(em_1)+(ere)*(em_2).
\end{align*}
\begin{align*}
\boxed{iii)}\,\, 1_{eRe}*(em)=e(em)=e^2m=em.
\end{align*}
\[\boxed{iv)}\,\,[(er_1e)(er_2e)]*(em)=[(er_1e)(er_2e)](em)=er_1eer_2eem\]
por otro lado
\[(er_1e)*[(er_2e)*(em)]=(er_1e)*(er_2eem)=er_1eer_2eem.\]
Así $*$ es una acción de módulos.\\

\boxed{c)} Para que la correspondencia sea funtorial debe preservar identidades y composición de morfismos, es decir:\\

\boxed{i)}\,\, $m_e(1_M)=1_{me(M)}$.\\
\boxed{ii)}\,\,Si  $ M \stackrel{f}{\longrightarrow} N \stackrel{g}{\longrightarrow} L,$ se tiene que $m_e(gf)=m_e(g)\circ m_e(f)$.\\

Sea $m\in M$ entonces\\

\boxed{i)}\\
$m_e(1_M)(em)=1_M|_{eM}(em)=em$, entonces $m_e(1_M)=1_{e_m(M)}=1_{eM}.$\\

\boxed{ii)}
\begin{gather*}
m_e(gf)(em)=(gf)|_{eM}(em)=gf(em)=g(f(em))=g(f|_{eM}(em))\\
=g(f|_{eM}(em))=g|_{eM}(f|_{eM}(em))=(g|_{eM}\circ f|_{eM})(em).
\end{gather*}
\end{proof}




\end{enumerate}
\end{document}