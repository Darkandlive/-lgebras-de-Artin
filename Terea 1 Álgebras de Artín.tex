\documentclass{article}
\usepackage[utf8]{inputenc}
\usepackage{mathrsfs}
\usepackage[spanish,es-lcroman]{babel}
\usepackage{amsthm}
\usepackage{amsmath}
\usepackage{amssymb}
\usepackage{enumitem}
\usepackage{graphicx}
\usepackage{caption}
\usepackage{float}
\usepackage{eufrak}
\usepackage{nicefrac}
\usepackage{amsmath,stackengine,scalerel,mathtools}
\usepackage{tikz-cd}

\newcommand{\Z}{\mathbb{Z}}

\def\subnormeq{\mathrel{\scalerel*{\trianglelefteq}{A}}}

\newcommand{\crdnlty}[1]{
    \left|#1\right|
}
\newcommand{\lrprth}[1]{
    \left(#1\right)
}
\newcommand{\lrbrack}[1]{
    \left\{#1\right\}
}
\newcommand{\descset}[3]{
    \left\{#1\in#2\ \vline\ #3\right\}
}
\newcommand{\descapp}[6]{
    #1: #2 &\rightarrow #3\\
    #4 &\mapsto #5#6 
}
\newcommand{\arbtfam}[3]{
    {\left\{{#1}_{#2}\right\}}_{#2\in #3}
}
\newcommand{\arbtfmnsub}[3]{
    {\left\{{#1}\right\}}_{#2\in #3}
}
\newcommand{\fntfmnsub}[3]{
    {\left\{{#1}\right\}}_{#2=1}^{#3}
}
\newcommand{\fntfam}[3]{
    {\left\{{#1}_{#2}\right\}}_{#2=1}^{#3}
}
\newcommand{\fntfamsup}[4]{
    \lrbrack{{#1}^{#2}}_{#3=1}^{#4}
}
\newcommand{\arbtuple}[3]{
    {\left({#1}_{#2}\right)}_{#2\in #3}
}
\newcommand{\fntuple}[3]{
    {\left({#1}_{#2}\right)}_{#2=1}^{#3}
}
\newcommand{\gengroup}[1]{
    \left< #1\right>
}
\newcommand{\stblzer}[2]{
    St_{#1}\lrprth{#2}
}
\newcommand{\cmmttr}[1]{
    \left[#1,#1\right]
}
\newcommand{\grpindx}[2]{
    \left[#1:#2\right]
}
\newcommand{\syl}[2]{
    Syl_{#1}\lrprth{#2}
}
\newcommand{\grtcd}[2]{
    mcd\lrprth{#1,#2}
}
\newcommand{\lsttcm}[2]{
    mcm\lrprth{#1,#2}
}
\newcommand{\amntpSyl}[2]{
    \mu_{#1}\lrprth{#2}
}
\newcommand{\gen}[1]{
    gen\lrprth{#1}
}
\newcommand{\ringcenter}[1]{
    C\lrprth{#1}
}
\newcommand{\zend}[2]{
    End_{\mathbb{Z}}^{#2}\lrprth{#1}
}


\theoremstyle{definition}
\newtheorem{define}{Definición}

\theoremstyle{plain}
\newtheorem{teor}{Teorema}[section]

\theoremstyle{plain}
\newtheorem{prop}{Proposición}[section]

\theoremstyle{definition}
\newtheorem{ejemp}{Ejemplo}[section]

\theoremstyle{definition}
\newtheorem*{coro}{Corolario} 

\theoremstyle{definition}
\newtheorem*{obs}{Observaci\'on}

\theoremstyle{definition}
\newtheorem*{pron}{Proposición}

\theoremstyle{definition}
 
\newtheorem{lem}{Lema}

\theoremstyle{definition}
\newtheorem*{nota}{Nota}

\begin{document}

\textbf{Ejercicio 2} \\
Para un anillo $R$ pruebe que: 
\begin{itemize}
\item[a)] $C(R)$ es un subanillo conmutativo de $R$.
\item[b)] $C(R)=C(R[{op}])$.
\item[c)] $R=R^{op}$ como anillos\,\, $\Leftrightarrow$ \,\,$C(R)=R$\,\, $\Leftrightarrow$ \,\, $R$ es conmutativo.
\end{itemize}

\textbf{Demostración.}\\
\textbf{a)}\quad Sean $a,b\in C(R)$, por definición $ax=xa$ y $bx=xb\quad\forall x\in R$, entonces
\[(a-b)x=ax-(bx)=xa-(xb)=x(a-b).\]
Por lo tanto $a-c\in C(R)$. Ahora, $abx=axb=xab$, por lo que $a,b\in C(R)$ y en consecuencia $C(R)\leq R$.\\

\textbf{b)}\quad Sea $*:R^{op}\times R^{op}\to R^{op}$ la operación de anillo en $R^{op}$. Así se tiene que 
\begin{gather*}
a\in C(R^{op})\quad \Leftrightarrow\quad  \forall x\in R\quad a*x=x*a
\\ \Leftrightarrow\quad \forall x\in R\quad xa=ax\\
\Leftrightarrow\quad \forall x\in R\quad a\in C(R).
\end{gather*}
\textbf{c)}\quad $\cdot)\Rightarrow \cdot\cdot)$\quad Supongamos $R=R^{op}$ como anillos, entonces sus operaciones coincides, es decir,
$\forall a\in R$ se tiene que $\forall x\in R\quad ax=a*x=xa$, entonces $R\subset C(R)\subset R$ por lo que $R=C(R)$.\\
 
$\cdot\cdot)\Rightarrow \cdot\cdot\cdot)$\quad Si $R=C(R)$, entonces $\forall x,y\in R\quad xy=yx$, por lo que $R$ es conmutativo.\\
  
$\cdot\cdot\cdot)\Rightarrow \cdot)$\quad Si $R$ es conmutativo, entonces $\forall x,a\in R\quad xa=ax=x*a$, y como $R$ y $R^{op}$ 
coinciden como grupos abelianos, entonces $R=R^{op}$ como anillos.\\

\textbf{Ejercicio 5} \\

Para un anillo conmutativo $K$, pruebe que se tiene una biyección \\
$\alpha:K-Alg\longrightarrow K_{AC}-Rings,\quad  (R,K,\varphi)\longmapsto \alpha_{\varphi}$, donde 
$\alpha_\varphi:K\times R\to R$ está dada por $\alpha_\varphi(k,r):=\varphi(k)r$; cuya inversa está dada por
 $\varphi_\alpha:=\alpha(k,1_R).$\\
 
\textbf{Demostración.}\\
Para evitar abusos de notación en la prueba se redefinirán las funciones de la siguiente forma. Sean 
$D=\{\varphi \,:\, (R,K,\varphi)\in  K-Alg\}$ y 
\[f:D\longrightarrow K_{AC}-Rings, \quad f(\varphi)=f_\varphi\]
donde $f_\varphi:K\times R\to R$ está dada por $f_\varphi(k,r):=\varphi(k)r$. Y definimos \\
$f^{-1}:K_{AC}-Rings\longrightarrow D$ como $f^{-1}(\alpha):=f^{-1}_\alpha$, donde
 $\alpha:K\times R\to R$ y $f^{-1}_\alpha(k):=\alpha (k,1_R)=K\cdot 1_R$.\\
 
Entonces \[\left((ff^{-1})(\alpha)\right)(k,r)=\left(f(f^{-1}_\alpha)\right)(k,r)=f^{-1}_\alpha(k)r=\alpha(k,1_R)r=\alpha(k,r)\]
y
\[\left((f^{-1}f)(\varphi)\right)(k)=\left(f^{-1}f_\varphi\right)(k)=f_\varphi(k,1_R)=\varphi(k)1_R=\varphi(k).\]

Por lo que $f$ es biyectiva con $f^{-1}$ su inversa.\\

\textbf{Ejercicio 8} \\

Sea $R$ un anillo, $(M,+)$ un grupo abeliano y $\varphi:R\times M\to M$ una función. La acción opuesta 
$\varphi^{op}: M\times R^{op}\to M$, se define como sigue:
\[\varphi^{op}(m,r^{op}):= \varphi(r,m)\quad \forall r\in R, \,\,\, \forall m\in M.\]
Pruebe que
\[(\prescript{}{R}{M},\varphi)\in  \prescript{}{R}{Mod} \Leftrightarrow (M_{R^{op}},\varphi^{op})\in Mod_{R^{op}}.\]

\textbf{Demostración.}\\
Recordando que $r_2^{op}r_1^{op}=(r_2r_1)^{op}$, se tiene que:\\

$(M_{R^{op}},\varphi^{op})\in Mod_{R^{op}}$\\\\
$\Leftrightarrow$
\begin{gather*}
i)\,\,\varphi^{op}[(m_1+m_2),r^{op}]=\varphi^{op}(m_1,r^{op})+\varphi^{op}(m_2,r^{op})\\
ii)\,\,\varphi^{op}[m,(r_1^{op}+r_2^{op})]=\varphi^{op}(m,r_1^{op})+\varphi^{op}(m,r_2^{op})\\
iii)\,\, \varphi^{op}(m,1_R^{op})=m\\
iv)\,\, \varphi^{op}(m,r_1^{op}r_2^{op})=\varphi^{op}(\varphi^{op}(m,r_1^{op}),r_2^{op}).
\end{gather*}
 $\Leftrightarrow$
\begin{gather*}
i)\,\,\varphi[r,(m_1+m_2)]=\varphi(r,m_1)+\varphi(r,m_2)\\
ii)\,\,\varphi[(r_1+r_2),m]=\varphi(r_1,m)+\varphi(r_2,m)\\
iii)\,\, \varphi(1_R,m)=m\\
iv)\,\, \varphi(r_2r_1,m)=\varphi(r_2,\varphi(r_1,m)).
\end{gather*}
 $\Leftrightarrow$\\
 
 $(\prescript{}{R}{M},\varphi)\in  \prescript{}{R}{Mod}$.\\
 
\textbf{Ejercicio 11} \\

Dado un morfismo de anillos $\varphi:R\to S$, construya la correspondencia análoga (a la de módulos) 
$F_\varphi:\prescript{}{S}{Rep}\longrightarrow \prescript{}{R}{Rep}$.\\

\textbf{Demostración.}\\
Sean $\varphi:R\to S$ un morfismo de anillos y $(\lambda,M)\in \prescript{}{S}{Rep}$ una representación a izquierda del anillo
$S$. Se define la correspondencia \textbf{Cambio de anillos}\\
 $F_\varphi:\prescript{}{S}{Rep}\longrightarrow \prescript{}{R}{Rep}$. Como grupos abelianos, $F_\lambda(M):= M$ y
 la representación $R\to End_{\Z}^{\,l}(M),\quad (r)\longmapsto \lambda'(r)$, se define por  $\lambda'(r):=\lambda(\varphi(r))$.
 Cabe observar que, como $\lambda$ y $\varphi$ son  morfismos de anillos entonces $\lambda'$ es morfismo de anillos.



































































\end{document}


