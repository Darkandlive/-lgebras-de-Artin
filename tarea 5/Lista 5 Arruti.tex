\documentclass{article}
\usepackage[utf8]{inputenc}
\usepackage{mathrsfs}
\usepackage[spanish,es-lcroman]{babel}
\usepackage{amsthm}
\usepackage{amssymb}
\usepackage{enumitem}
\usepackage{graphicx}
\usepackage{caption}
\usepackage{float}
\usepackage{amsmath,stackengine,scalerel,mathtools}
\usepackage{xparse, tikz-cd, pgfplots}
\usepackage{faktor}
\usepackage[all]{xy}



\def\subnormeq{\mathrel{\scalerel*{\trianglelefteq}{A}}}
\newcommand{\Z}{\mathbb{Z}}
\newcommand{\La}{\mathscr{L}}
\newcommand{\crdnlty}[1]{
	\left|#1\right|
}
\newcommand{\lrprth}[1]{
	\left(#1\right)
}
\newcommand{\lrbrack}[1]{
	\left\{#1\right\}
}
\newcommand{\lrsqp}[1]{
	\left[#1\right]
}
\newcommand{\descset}[3]{
	\left\{#1\in#2\ \vline\ #3\right\}
}
\newcommand{\descapp}[6]{
	#1: #2 &\rightarrow #3\\
	#4 &\mapsto #5#6 
}
\newcommand{\arbtfam}[3]{
	{\left\{{#1}_{#2}\right\}}_{#2\in #3}
}
\newcommand{\arbtfmnsub}[3]{
	{\left\{{#1}\right\}}_{#2\in #3}
}
\newcommand{\fntfmnsub}[3]{
	{\left\{{#1}\right\}}_{#2=1}^{#3}
}
\newcommand{\fntfam}[3]{
	{\left\{{#1}_{#2}\right\}}_{#2=1}^{#3}
}
\newcommand{\fntfamsup}[4]{
	\lrbrack{{#1}^{#2}}_{#3=1}^{#4}
}
\newcommand{\arbtuple}[3]{
	{\left({#1}_{#2}\right)}_{#2\in #3}
}
\newcommand{\fntuple}[3]{
	{\left({#1}_{#2}\right)}_{#2=1}^{#3}
}
\newcommand{\gengroup}[1]{
	\left< #1\right>
}
\newcommand{\stblzer}[2]{
	St_{#1}\lrprth{#2}
}
\newcommand{\cmmttr}[1]{
	\left[#1,#1\right]
}
\newcommand{\grpindx}[2]{
	\left[#1:#2\right]
}
\newcommand{\syl}[2]{
	Syl_{#1}\lrprth{#2}
}
\newcommand{\grtcd}[2]{
	mcd\lrprth{#1,#2}
}
\newcommand{\lsttcm}[2]{
	mcm\lrprth{#1,#2}
}
\newcommand{\amntpSyl}[2]{
	\mu_{#1}\lrprth{#2}
}
\newcommand{\gen}[1]{
	gen\lrprth{#1}
}
\newcommand{\ringcenter}[1]{
	C\lrprth{#1}
}
\newcommand{\zend}[2]{
	End_{\mathbb{Z}}^{#2}\lrprth{#1}
}
\newcommand{\genmod}[2]{
	\left< #1\right>_{#2}
}
\newcommand{\genlin}[1]{
	\mathscr{L}\lrprth{#1}
}
\newcommand{\opst}[1]{
	{#1}^{op}
}
\newcommand{\ringmod}[3]{
	\if#3l
	{}_{#1}#2
	\else
	\if#3r
	#2_{#1}
	\fi
	\fi
}
\newcommand{\ringbimod}[4]{
	\if#4l
	{}_{#1-#2}#3
	\else
	\if#4r
	#3_{#1-#2}
	\else 
	\ifstrequal{#4}{lr}{
		{}_{#1}#3_{#2}
	}
	\fi
	\fi
}
\newcommand{\ringmodhom}[3]{
	Hom_{#1}\lrprth{#2,#3}
}

\ExplSyntaxOn

\NewDocumentCommand{\functor}{O{}m}
{
	\group_begin:
	\keys_set:nn {nicolas/functor}{#2}
	\nicolas_functor:n {#1}
	\group_end:
}

\keys_define:nn {nicolas/functor}
{
	name     .tl_set:N = \l_nicolas_functor_name_tl,
	dom   .tl_set:N = \l_nicolas_functor_dom_tl,
	codom .tl_set:N = \l_nicolas_functor_codom_tl,
	arrow      .tl_set:N = \l_nicolas_functor_arrow_tl,
	source   .tl_set:N = \l_nicolas_functor_source_tl,
	target   .tl_set:N = \l_nicolas_functor_target_tl,
	Farrow      .tl_set:N = \l_nicolas_functor_Farrow_tl,
	Fsource   .tl_set:N = \l_nicolas_functor_Fsource_tl,
	Ftarget   .tl_set:N = \l_nicolas_functor_Ftarget_tl,	
	delimiter .tl_set:N= \l_nicolas_functor_delimiter_tl,	
}

\dim_new:N \g_nicolas_functor_space_dim

\cs_new:Nn \nicolas_functor:n
{
	\begin{tikzcd}[ampersand~replacement=\&,#1]
		\dim_gset:Nn \g_nicolas_functor_space_dim {\pgfmatrixrowsep}		
		\l_nicolas_functor_dom_tl
		\arrow[r,"\l_nicolas_functor_name_tl"] \&
		\l_nicolas_functor_codom_tl
		\tl_if_blank:VF \l_nicolas_functor_source_tl {
			\\[\dim_eval:n {1ex-\g_nicolas_functor_space_dim}]
			\l_nicolas_functor_source_tl
			\xrightarrow{\l_nicolas_functor_arrow_tl}
			\l_nicolas_functor_target_tl
			\arrow[r,mapsto] \&
			\l_nicolas_functor_Fsource_tl
			\xrightarrow{\l_nicolas_functor_Farrow_tl}
			\l_nicolas_functor_Ftarget_tl
			\l_nicolas_functor_delimiter_tl
		}
	\end{tikzcd}
}
\ExplSyntaxOff

\ExplSyntaxOn

\NewDocumentCommand{\shortseq}{O{}m}
{
	\group_begin:
	\keys_set:nn {nicolas/shortseq}{#2}
	\nicolas_shortseq:n {#1}
	\group_end:
}

\keys_define:nn {nicolas/shortseq}
{
	A     .tl_set:N = \l_nicolas_shortseq_A_tl,
	B   .tl_set:N = \l_nicolas_shortseq_B_tl,
	C .tl_set:N = \l_nicolas_shortseq_C_tl,
	AtoB      .tl_set:N = \l_nicolas_shortseq_AtoB_tl,
	BtoC   .tl_set:N = \l_nicolas_shortseq_BtoC_tl,	
	lcr   .tl_set:N = \l_nicolas_shortseq_lcr_tl,	
	
	A		.initial:n =A,
	B		.initial:n =B,
	C		.initial:n =C,
	AtoB    .initial:n =,
	BtoC   	.initial:n=,
	lcr   	.initial:n=lr,
	
}

\cs_new:Nn \nicolas_shortseq:n
{
	\begin{tikzcd}[ampersand~replacement=\&,#1]
		\IfSubStr{\l_nicolas_shortseq_lcr_tl}{l}{0 \arrow{r} \&}{}
		\l_nicolas_shortseq_A_tl
		\arrow{r}{\l_nicolas_shortseq_AtoB_tl} \&
		\l_nicolas_shortseq_B_tl
		\arrow[r, "\l_nicolas_shortseq_BtoC_tl"] \&
		\l_nicolas_shortseq_C_tl
		\IfSubStr{\l_nicolas_shortseq_lcr_tl}{r}{ \arrow{r} \& 0}{}
	\end{tikzcd}
}

\ExplSyntaxOff
\newcommand{\limseq}[2]{
	\lim_{#2\to\infty}#1
}

\newcommand{\norm}[1]{
	\crdnlty{\crdnlty{#1}}
}

\newcommand{\inter}[1]{
	int\lrprth{#1}
}
\newcommand{\cerrad}[1]{
	cl\lrprth{#1}
}

\newcommand{\restrict}[2]{
	\left.#1\right|_{#2}
}
\newcommand{\functhom}[3]{
	\ifblank{#1}{
		Hom_{#3}\lrprth{-,#2}
	}{
		\ifblank{#2}{
			Hom_{#3}\lrprth{#1,-}
		}{
			Hom_{#3}\lrprth{#1,#2}	
		}
	}
}
\newcommand{\socle}[1]{
	Soc\lrprth{#1}
}

\theoremstyle{definition}
\newtheorem{define}{Definición}
\newtheorem{lem}{Lema}
\newtheorem{teo}{Teorema}
\newtheorem*{teosn}{Teorema}
\newtheorem*{obs}{Observación}
\title{Lista 5}
\author{Arruti, Sergio, Jesús}
\date{}


\begin{document}
	\maketitle
	\begin{enumerate}[label=\textbf{Ej \arabic*.}]
		\setcounter{enumi}{67}
		\item	 
		\item Sean $R$ artiniano a izquierda y $M\in mod\lrprth{R}$. Entonces
		$\forall\ N\in\genlin{M}$ 
		\begin{equation*}
			\faktor{M}{N}\text{ es semisimple }\implies rad\lrprth{M}\subseteq N.
		\end{equation*}
		\begin{proof}
			Sea $\mathscr{R}:=J(R)$. Como $M\in mod\lrprth{R}$, entonces por el Ej. 18 $\exists\ n\in\mathbb{N}$ y $f:R^n\to M$ un epimorfismo en  $Mod(R)$. Así, sí $\pi$ es el epimorfismo canónico en $Mod(R)$ de $M$ en $\faktor{M}{N}$, se tiene que $\pi f:R^n\to \faktor{M}{N}$ es un epimorfismo en $Mod(R)$ y por lo tanto, nuevamente por el Ej. 18, $\faktor{M}{N}\in mod(R)$ . Por lo anterior, dado que $S\in mod(R)\text{ es semisimple }\iff \mathscr{R}S=0$ (véase 2.7.13 (c)) y que $rad\lrprth{M}=\mathscr{R}M$ (véase 2.7.17 (b)), basta con verificar la siguiente implicación:
			\begin{equation*}
				\mathscr{R}\faktor{M}{N}=0\implies \mathscr{R} M\subseteq N.
			\end{equation*}
			Se tiene que
			\begin{align*}
				x\in\mathscr{R}M&\implies \exists\ t\in\mathbb{N}\text{ t.q. } x=\sum\limits_{i=1}^{t}r_im_i,\ r_i\in\mathscr{R}\text{ y } m_i\in M\ \forall\ i\in[1,t]\\
				&\implies \pi\lrprth{x}=\sum\limits_{i=1}^{t}r_i\pi\lrprth{m_i},\ r_i\in\mathscr{R}\text{ y } \pi\lrprth{m_i}\in \faktor{M}{N}\ \forall\ i\in[1,t]\\\\
				&\implies \pi\lrprth{x}\in\mathscr{R}\faktor{M}{N}=0\\
				&\implies x\in Ker\lrprth{\pi}=N.\\
				\implies \mathscr{R} M&\subseteq N.
			\end{align*}
		\end{proof}
		\item 
		\item
		\item Sean $f:P\to M$ y $g:Q\to M$ 	cubiertas proyectivas de $M\in Nod\lrprth{R}$. Entonces $\exists\ h:P\to Q$ isomorfismo en $Mod\lrprth{R}$ tal que $gh=f$.
		\begin{proof}
			Se tiene el siguiente esquema
			\begin{center}
				\begin{tikzcd}
					& \ar[dashrightarrow]{dl}[swap]{\exists\ h}P\ar{d}{f}\\
					Q\ar{r}[swap]{g}&M
				\end{tikzcd}
			\end{center}
			con $P$ proyectivo y $g$, en partícular por ser un epi-esencial, un epimorfismo en $Mod(R)$. Por lo tanto $\exists\ h\in\functhom{P}{Q}{R}$ tal que \begin{equation*}\tag{*}\label{factproy}
				gh=f.
			\end{equation*} Así pues, basta con verificar que $h$ es un isomorfismo en $Mod(R)$. De (\ref{factproy}) se sigue que, como $g$ es un epi-esencial y $f$ es en  partícular un epimorfismo en $Mod(R)$,  $h$ es un epimorfismo en $Mod(R)$. Con lo cual, si $i$ es la inclusión natural de $Ker\lrprth{h}$ en $P$, la sucesión
			\begin{center}
				\shortseq{
					A=Ker\lrprth{h}, B=P, C=Q, AtoB=i, BtoC=h
				}
			\end{center}
			es exacta. Más aún es una sucesión exacta que se parte, puesto que $Q$ es proyectivo (Ej. 62), con lo cual $h$ es un split-epi (Ej. 54) i.e. $\exists$ $j\in\functhom{Q}{P}{R}$ tal que $hj=Id_Q$. Notemos que lo anterior garantiza que $j$ es un split-mono y así en partícular es un monomorfismo. Además
			\begin{align*}
				gh=f&\implies fj=g,
			\end{align*}
			con lo cual $j$ es un epimorfismo, pues $g$ lo es y $f$ es un epi-esencial. Así $j$ es un isomorfismo en $Mod(R)$ y por lo tanto $h=j^{-1}$ también lo es.\\
		\end{proof}
		\item 
		\item
		\item Sean $R$ un anillo no trivial.
		\begin{enumerate}
			\item Sean $e\in R\setminus\lrbrack{0}$ un idempotente, $\fntfam{P}{i}{n}$ una familia en $\genlin{Re}$ y $\mathcal{A}:=\fntfam{e}{i}{n}\subseteq R$. Si $Re=\bigoplus\limits_{i=1}^n P_i$ $\forall\ i\in[1,n]$ $e_i\in P_i$ y $e=\sum\limits_{i=1}^ne_i$, entonces $\mathcal{A}$ es una familia de idempotentes ortogonales. Más aún $\forall\ i\in[1,n]$ $Re_i=P_i$.
			\item Si $\fntfam{e}{i}{n}$ es una familia de idempotentes ortogonales en $R$ y $e:=\sum\limits_{i=1}^ne_i$, entonces $\forall\ i\in[1,n]$ $Re_i\in\genlin{\ringmod{R}{Re}{l}}$ y $Re=\bigoplus\limits_{i=1}^n Re_i$.
		\end{enumerate}
		\begin{proof}
			\boxed{a)} Sea $u\in[1,n]$. Notemos primeramente que como $e_u\in Re$, entonces $\exists\ r_u\in R$ tal que $e_u=r_ue$, y así
			\begin{align*}
				e_ue&=\lrprth{r_ue}e=r_u\lrprth{ee}\\
				&=r_ue, && e^2=e\\
				&=e_u.
			\end{align*}
			Así
			\begin{align*}
				e_u&=e_ue=e_u\sum\limits_{i=1}^ne_i\\
				&=\sum\limits_{i=1}^n e_ue_i\\
				&=e_u^2+\sum\limits_{\substack{i=1\\i\neq u}}^n e_ue_i.
			\end{align*}			
			Como $e_u\in P_u$, $\forall\ i\in [1,n]$ $e_ue_i\in P_i$ y la desomposición en suma en $\sum\limits_{i=1}^nP_i$ es única, por formar $\fntfam{P}{i}{n}$ una suma directa para $Re$, lo anterior garantiza que
			$e_u=e_u^2$ y que $\forall\ i\neq u$ $e_ue_i=0$. Por lo tanto $\fntfam{e}{i}{n}$ es una familia de idempotentes ortogonales (f.i.o.). \\
			Por su parte, como $e_u\in P_u\leq Re$ entonces $Re_u\subseteq P_u$, así que basta con probar que $P_u\subseteq Re_u$. Sea $p\in P_u\leq Re$, entonces $\exists\ q\in R$ tal que \\
			\begin{align*}
				p&=qe=q\sum\limits_{i=1}^ne_i\\
				\implies &p-qe_u=\sum\limits_{\substack{i=1\\i\neq u}}^nqe_i, 
				\intertext{con}
				p-qe_u&\in P_u,\ \sum\limits_{\substack{i=1\\i\neq u}}^nqe_i\in \sum\limits_{\substack{i=1\\i\neq u}}^nP_i.
			\end{align*}
			Dado que $P_u\cap\sum\limits_{\substack{i=1\\i\neq u}}^nP_i=\gengroup{0}$, se sigue que $p=qe_u\in Re_u$\\
			\boxed{b)} Sea $r\in R$. Como $\forall\ i\in I$ $Re_i\in Mod(R)$, para verificar que $Re_i\in\genlin{\ringmod{R}{Re}{l}}$ basta con probar que $Re_i\subseteq Re$, y esto último es consecuencia de que si $re_i\in Re_i$ entonces $\lrprth{re_i}e\in Re$ y
			\begin{align*}
				\lrprth{re_i}e&=r\lrprth{e_ie}\\
				&=r\lrprth{e_i\sum_{j=1}^{n}e_j}\\
				&=re_i. && \fntfam{e}{j}{n}\text{ es una f. i. o.}
			\end{align*}
			Más aún, así se tiene que $\sum\limits_{i=1}^nRe_i\subseteq Re$. Notemos que $re=\sum\limits_{i=1}^nre_i\in\sum\limits_{i=1}^nRe_i$, así para verificar que $Re=\bigoplus\limits_{i=1}^nRe_i$ basta con verificar que esta descomposición es única. Sea $s\in R$ tal que $re=\sum\limits_{i=1}^n se_i$, entonces
			\begin{align*}
				\sum\limits_{i=1}^n re_i&=\sum\limits_{i=1}^n se_i\\
				\implies \sum\limits_{i=1}^n \lrprth{r-s}e_i&=0.\\
				\intertext{Sea $j\in[1,n]$. Multiplicando a ambos lados de la igualdad por $e_j$ y empleando nuevamente que $\fntfam{e}{j}{n}$ es una f. i. o. se obtiene que}
				\lrprth{r-s}e_j&=0,\ \forall\ j\in[1,n]\\
				\implies re_j&=se_j ,\ \forall\ j\in[1,n]
			\end{align*}
			y así se tiene lo deseado.\\
		\end{proof}
		\item 
		\item
		\item Sea $R$ un anillo. $R$ es semisimple si y sólo si $gldim\lrprth{R}=0$.
		\begin{proof}
			Afirmamos que $M$ es proyectivo, $\forall\ M\in Mod\lrprth{R}$, si y sólo sí $gldim\lrprth{R}=0$. En efecto:\\
			\boxed{\implies} Se tiene que si $M$ es proyectivo, entonces por el Ej. 77$a)$ $pd\lrprth{M}=0$. Luego bajo estas condiciones, como por el Teorema $2.9.1\ (a)$ \begin{equation*}
				gldim\lrprth{R}=\sup\limits_{M\in Mod\lrprth{R}}\lrbrack{pd\lrbrack{M}},
			\end{equation*} se tiene que $gldim\lrprth{R}=\sup\limits_{M\in Mod\lrprth{R}}\lrbrack{0}=0$.\\
			\boxed{\impliedby} Sea $M\in Mod\lrprth{R}$. Como en partícular $gldim\lrprth{R}$ es cota superior de $\lrbrack{pd\lrbrack{M}\ \vline\ M\in Mod\lrprth{R}}$, entonces $pd\lrprth{M}\leq 0$. En tal caso $pd\lrprth{M}\in\mathbb{N}$ y por tanto $pd\lrprth{M}\geq 0$. Con lo cual $pd\lrprth{M}=0$, así que, por el Ej. 77$a)$, $M$ es proyectivo.\\
			Por la equivalencia previamente demostrada, y dado que por la Proposición 2.6.8 \begin{equation*}
				 M \text{ es proyectivo, }\forall\ M\in Mod\lrprth{R} \iff R\text{ es semisimple,}
			\end{equation*}se tiene lo deseado.\\
		\end{proof}

	\end{enumerate}
\end{document}