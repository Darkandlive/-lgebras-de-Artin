\documentclass{article}
\usepackage[utf8]{inputenc}
\usepackage{mathrsfs}
\usepackage[spanish,es-lcroman]{babel}
\usepackage{amsthm}
\usepackage{amssymb}
\usepackage{enumitem}
\usepackage{graphicx}
\usepackage{caption}
\usepackage{float}
\usepackage{amsmath,stackengine,scalerel,mathtools}
\usepackage{xparse, tikz-cd, pgfplots}
\usepackage{comment}
\usepackage{faktor}
\usepackage[all]{xy}


\def\subnormeq{\mathrel{\scalerel*{\trianglelefteq}{A}}}
\newcommand{\Z}{\mathbb{Z}}
\newcommand{\La}{\mathscr{L}}
\newcommand{\crdnlty}[1]{
	\left|#1\right|
}
\newcommand{\lrprth}[1]{
	\left(#1\right)
}
\newcommand{\lrbrack}[1]{
	\left\{#1\right\}
}
\newcommand{\lrsqp}[1]{
	\left[#1\right]
}
\newcommand{\descset}[3]{
	\left\{#1\in#2\ \vline\ #3\right\}
}
\newcommand{\descapp}[6]{
	#1: #2 &\rightarrow #3\\
	#4 &\mapsto #5#6 
}
\newcommand{\arbtfam}[3]{
	{\left\{{#1}_{#2}\right\}}_{#2\in #3}
}
\newcommand{\arbtfmnsub}[3]{
	{\left\{{#1}\right\}}_{#2\in #3}
}
\newcommand{\fntfmnsub}[3]{
	{\left\{{#1}\right\}}_{#2=1}^{#3}
}
\newcommand{\fntfam}[3]{
	{\left\{{#1}_{#2}\right\}}_{#2=1}^{#3}
}
\newcommand{\fntfamsup}[4]{
	\lrbrack{{#1}^{#2}}_{#3=1}^{#4}
}
\newcommand{\arbtuple}[3]{
	{\left({#1}_{#2}\right)}_{#2\in #3}
}
\newcommand{\fntuple}[3]{
	{\left({#1}_{#2}\right)}_{#2=1}^{#3}
}
\newcommand{\gengroup}[1]{
	\left< #1\right>
}
\newcommand{\stblzer}[2]{
	St_{#1}\lrprth{#2}
}
\newcommand{\cmmttr}[1]{
	\left[#1,#1\right]
}
\newcommand{\grpindx}[2]{
	\left[#1:#2\right]
}
\newcommand{\syl}[2]{
	Syl_{#1}\lrprth{#2}
}
\newcommand{\grtcd}[2]{
	mcd\lrprth{#1,#2}
}
\newcommand{\lsttcm}[2]{
	mcm\lrprth{#1,#2}
}
\newcommand{\amntpSyl}[2]{
	\mu_{#1}\lrprth{#2}
}
\newcommand{\gen}[1]{
	gen\lrprth{#1}
}
\newcommand{\ringcenter}[1]{
	C\lrprth{#1}
}
\newcommand{\zend}[2]{
	End_{\mathbb{Z}}^{#2}\lrprth{#1}
}
\newcommand{\genmod}[2]{
	\left< #1\right>_{#2}
}
\newcommand{\genlin}[1]{
	\mathscr{L}\lrprth{#1}
}
\newcommand{\opst}[1]{
	{#1}^{op}
}
\newcommand{\ringmod}[3]{
	\if#3l
	{}_{#1}#2
	\else
	\if#3r
	#2_{#1}
	\fi
	\fi
}
\newcommand{\ringbimod}[4]{
	\if#4l
	{}_{#1-#2}#3
	\else
	\if#4r
	#3_{#1-#2}
	\else 
	\ifstrequal{#4}{lr}{
		{}_{#1}#3_{#2}
	}
	\fi
	\fi
}
\newcommand{\ringmodhom}[3]{
	Hom_{#1}\lrprth{#2,#3}
}

\ExplSyntaxOn

\NewDocumentCommand{\functor}{O{}m}
{
	\group_begin:
	\keys_set:nn {nicolas/functor}{#2}
	\nicolas_functor:n {#1}
	\group_end:
}

\keys_define:nn {nicolas/functor}
{
	name     .tl_set:N = \l_nicolas_functor_name_tl,
	dom   .tl_set:N = \l_nicolas_functor_dom_tl,
	codom .tl_set:N = \l_nicolas_functor_codom_tl,
	arrow      .tl_set:N = \l_nicolas_functor_arrow_tl,
	source   .tl_set:N = \l_nicolas_functor_source_tl,
	target   .tl_set:N = \l_nicolas_functor_target_tl,
	Farrow      .tl_set:N = \l_nicolas_functor_Farrow_tl,
	Fsource   .tl_set:N = \l_nicolas_functor_Fsource_tl,
	Ftarget   .tl_set:N = \l_nicolas_functor_Ftarget_tl,	
	delimiter .tl_set:N= \l_nicolas_functor_delimiter_tl,	
}

\dim_new:N \g_nicolas_functor_space_dim

\cs_new:Nn \nicolas_functor:n
{
	\begin{tikzcd}[ampersand~replacement=\&,#1]
		\dim_gset:Nn \g_nicolas_functor_space_dim {\pgfmatrixrowsep}		
		\l_nicolas_functor_dom_tl
		\arrow[r,"\l_nicolas_functor_name_tl"] \&
		\l_nicolas_functor_codom_tl
		\tl_if_blank:VF \l_nicolas_functor_source_tl {
			\\[\dim_eval:n {1ex-\g_nicolas_functor_space_dim}]
			\l_nicolas_functor_source_tl
			\xrightarrow{\l_nicolas_functor_arrow_tl}
			\l_nicolas_functor_target_tl
			\arrow[r,mapsto] \&
			\l_nicolas_functor_Fsource_tl
			\xrightarrow{\l_nicolas_functor_Farrow_tl}
			\l_nicolas_functor_Ftarget_tl
			\l_nicolas_functor_delimiter_tl
		}
	\end{tikzcd}
}
\ExplSyntaxOff

\ExplSyntaxOn

\NewDocumentCommand{\shortseq}{O{}m}
{
	\group_begin:
	\keys_set:nn {nicolas/shortseq}{#2}
	\nicolas_shortseq:n {#1}
	\group_end:
}

\keys_define:nn {nicolas/shortseq}
{
	A     .tl_set:N = \l_nicolas_shortseq_A_tl,
	B   .tl_set:N = \l_nicolas_shortseq_B_tl,
	C .tl_set:N = \l_nicolas_shortseq_C_tl,
	AtoB      .tl_set:N = \l_nicolas_shortseq_AtoB_tl,
	BtoC   .tl_set:N = \l_nicolas_shortseq_BtoC_tl,	
	lcr   .tl_set:N = \l_nicolas_shortseq_lcr_tl,	
	
	A		.initial:n =A,
	B		.initial:n =B,
	C		.initial:n =C,
	AtoB    .initial:n =,
	BtoC   	.initial:n=,
	lcr   	.initial:n=lr,
	
}

\cs_new:Nn \nicolas_shortseq:n
{
	\begin{tikzcd}[ampersand~replacement=\&,#1]
		\IfSubStr{\l_nicolas_shortseq_lcr_tl}{l}{0 \arrow{r} \&}{}
		\l_nicolas_shortseq_A_tl
		\arrow{r}{\l_nicolas_shortseq_AtoB_tl} \&
		\l_nicolas_shortseq_B_tl
		\arrow[r, "\l_nicolas_shortseq_BtoC_tl"] \&
		\l_nicolas_shortseq_C_tl
		\IfSubStr{\l_nicolas_shortseq_lcr_tl}{r}{ \arrow{r} \& 0}{}
	\end{tikzcd}
}

\ExplSyntaxOff
\newcommand{\limseq}[2]{
	\lim_{#2\to\infty}#1
}

\newcommand{\norm}[1]{
	\crdnlty{\crdnlty{#1}}
}

\newcommand{\inter}[1]{
	int\lrprth{#1}
}
\newcommand{\cerrad}[1]{
	cl\lrprth{#1}
}

\newcommand{\restrict}[2]{
	\left.#1\right|_{#2}
}
\newcommand{\functhom}[3]{
	\ifblank{#1}{
		Hom_{#3}\lrprth{-,#2}
	}{
		\ifblank{#2}{
			Hom_{#3}\lrprth{#1,-}
		}{
			Hom_{#3}\lrprth{#1,#2}	
		}
	}
}
\newcommand{\socle}[1]{
	Soc\lrprth{#1}
}

\theoremstyle{definition}
\newtheorem{define}{Definición}
\newtheorem{lem}{Lema}
\newtheorem{teo}{Teorema}
\newtheorem*{teosn}{Teorema}
\newtheorem*{obs}{Observación}
\title{Lista 4}
\author{Arruti, Sergio, Jesús}
\date{}


\begin{document}
	\maketitle
	\begin{enumerate}[label=\textbf{Ej \arabic*.}]
		\setcounter{enumi}{67}
\item
Sea $R$ un anillo artiniano (noetheriano) a izquierda. Pruebe que \\ $\forall M\in mod(R)$, $M$ es artiniano (noetheriano).
\begin{proof}
Sea $M$ un $R$-módulo finitamente generado, como $\displaystyle\bigoplus_{m\in M}R_m$ genera a $M$ entonces existe un subconjunto finito 
$A$ de $M$ tal que \\ $M\displaystyle\bigoplus_{m\in A}R_m$, por lo que si $m_0\in A$, la sucesión \[
\begin{tikzcd}
 0   \arrow{r}{}& R_{m_0} \arrow{r}{i_0} & M\arrow{r}{\pi_0} &\arrow{r}{} \displaystyle\bigoplus_{m\in M\backslash \{m_0\}}R_m &0.
\end{tikzcd}
\]

Ahora, si $R$ es artiniano (noetheriano) entonces \,\,\,$R_{m_0}$ y $ \displaystyle\bigoplus_{m\in M\backslash \{m_0\}}R_m$ \,\,\,también son 
artinianos (noetherianos) por ser $A$ finito.Y por la proposición 10.12 del libro de Anderson-Fuller, $M$ es artiniano (noetheriano).
\end{proof}

\item
\item
\item Para un anillo $R$ y $M\in Mod(R)$, pruebe que 
\begin{itemize}
\item[a)] Si $e\in End(\prescript{}{R}{M})$ es tal que $e^2=e$, entonces $M=eM\oplus (1-e)M$ y $eM=\{m\in M\,|\, e(m)=m\}.$
\item[b)] Sean $M_1,M_2\in \La(M)$. Si $M=M_1\oplus M_2$, entonces existe\\ $e\in End(\prescript{}{R}{M})$ tal que: $e^2=e$, $eM=M_1$\quad
y\quad $(1-e)M=M_2$.
\end{itemize}
\begin{proof}

\boxed{a)}\\
Supongamos $x\in M\cap(1-e)M$, entonces $x=ey=(1-e)z$, es decir, $ey=z-ez$ por lo que $e(y+z)=z$,\\
Así \[x=(1-e)(e(y+z))=e(y+z)-e^2(y+z)=e(y+z)-e(y-z)=0.
\]
por lo tanto $M\cap (1-e)M=\{0\}.$\\

Sea $x\in M$ entonces $x=(x-ex)+ex$ donde $(x-ex)=(1-e)x\in (1-e)M$ y\quad $ex\in eM$. Así $x\in eM\oplus (1-e)M$.\\

Por último, sea $y\in eM$ entonces $y=ex$ para alguna $x\in M$, y por lo anterior, $e(y)=eex=ex=y$. Así $eM=\{m\in M\,|\, e(m)=m\}.$\\
\boxed{b)}\\
Sea $e=\mu_1\pi_1$ donde $\pi\colon M\longrightarrow M_1$ es la proyección canónica y \\$\mu_1\colon M_1\longrightarrow M$ es la 
inclusión canónica. Entonces $\pi_1\mu_1=Id_{M_1}$, por lo que $e^2(m_1)=\mu_1\pi_1\mu_1\pi_1(m_1)=\mu_1\pi_1(m_1)=e(m_1)$ \,\,
para toda $m_1\in M_1$.\\

Sea $m\in M$ entonces $e(m)=\mu_1\pi_1(m)=\mu_1(\pi_1(m))\in M_1$, por lo que $eM\subseteq M_1$ y todo elemento $x\in M_1$ cumple 
que\\$e(x)=\mu_1\pi_1(x)=\mu_1(x)=x$ por lo que $M_1=eM$.\\

Por otra parte, por a), $M=M_1\oplus(1-e)M$ y por hipótesis $M=M_1\oplus M_2$, entonces $M_2=(1-e)M$ pues si $x\in M$, existe $m_1\in M_1,\,\,
m_2\in M_2\,$ y $m_3\in M$ tales que $x=m_1+m_2=m_1+(1-e)m_3$ por lo que $m_2=(1-e)m_3$.
\end{proof}
\item
\item
\item Para un anillo artiniano a izquierda $R$, pruebe que $R$ es local $\iff$ $R^{op}$ es local.
\begin{proof}
Por definición un anillo $A$ es local si $A\neq 0$ y satisface alguna de las condiciones de 2.7.20 (en particular c) de esta proposición). \\
Dado que $R^{op}-U(R^{op})=R-U(R)$ y $J(R)=J(R^{op})$, entonces $R$ es local si y sólo si 
\[R-U(R)=J(R)\]
si y sólo si
\[R^{op}-U(R^{op})=J(R^{op})\]
si y sólo si\quad  $R^{op}$ es local.
\end{proof}
\item
\item
\item Para un anillo $R$ y $M\in Mod(R)$, pruebe que 
\begin{itemize}
\item[a)] $M$ es proyectivo $\iff$  $pd(M)$=0.
\item[b)] $M$ es inyectivo $\iff$  $id(M)$=0.
\end{itemize}

\begin{proof}
\boxed{a)}\\
Supongamos $M$ es proyectivo, entonces tenemos la sucesión exacta 
\[
\begin{tikzcd}
 P_\bullet \colon\,\ldots\arrow{r}{} & P_1=0 \arrow{r}{0}& M=P_0 \arrow{r}{Id_M} & M\arrow{r}{} &0
\end{tikzcd}
\]

donde, como $M$ es proyectivo, $P_\bullet$ es resolución proyectiva. Así \\ $pd(M)=l(P_\bullet)=0$.\\

Por otra parte, si $pd(M)=0$ entonces existe resulución proyectiva $P_\bullet$ tal que $l(P_\bullet)=0$, es decir, se tiene la siguiente 
sucesión exacta con $P_0$ proyectivo,
\[
\begin{tikzcd}
 P_\bullet \colon\,\ldots\arrow{r}{} & P_1=0 \arrow{r}{0}& P_0 \arrow{r}{} & M\arrow{r}{} &0,
\end{tikzcd}
\]

pero por el ejercicio 38 esto implica que $P_0$ es isomorfo a $M$, por lo tanto $M$ es proyectivo.\\
\boxed{b)}\\
Supongamos $M$ es inyectivo, entonces tenemos la siguiente corelación inyectiva:

\[
\begin{tikzcd}
 I_\bullet \colon 0\arrow{r}{} & M \arrow{r}{Id_M} & M=P_0\arrow{r}{} &0=P_1\arrow{r}{}&\ldots
\end{tikzcd}
\]

entonces $l( I_\bullet)=0$ y por lo tanto $id(M)=0.$\\

Por otra parte, si $id(M)=0$ entonces existe una correlación inyectiva de longitud cero, es decir, una sucesión exacta de la siguiente forma:
\[
\begin{tikzcd}
 I_\bullet \colon 0\arrow{r}{} & M \arrow{r}{} & P_0\arrow{r}{} &0=P_1\arrow{r}{}&\ldots
\end{tikzcd}
\]

Como la sucesión es exacta, entonces por el ejercicio 38 se tiene que $M$ es isomorfo a $P_0$ el cual es inyectivo, por lo tanto $M$ es inyectivo.










\end{proof}








\end{enumerate}
\end{document}