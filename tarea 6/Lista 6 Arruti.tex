\documentclass{article}
\usepackage[utf8]{inputenc}
\usepackage{mathrsfs}
\usepackage[spanish,es-lcroman]{babel}
\usepackage{amsthm}
\usepackage{amssymb}
\usepackage{enumitem}
\usepackage{graphicx}
\usepackage{caption}
\usepackage{float}
\usepackage{amsmath,stackengine,scalerel,mathtools}
\usepackage{xparse, tikz-cd, pgfplots}
\usepackage{faktor}
\usepackage{xstring}
\usepackage[all]{xy}

\newcommand{\socle}[1]{
	Soc\lrprth{#1}
}



\def\subnormeq{\mathrel{\scalerel*{\trianglelefteq}{A}}}
\newcommand{\Z}{\mathbb{Z}}
\newcommand{\La}{\mathscr{L}}
\newcommand{\crdnlty}[1]{
	\left|#1\right|
}
\newcommand{\lrprth}[1]{
	\left(#1\right)
}
\newcommand{\lrbrack}[1]{
	\left\{#1\right\}
}
\newcommand{\descset}[3]{
	\left\{#1\in#2\ \vline\ #3\right\}
}
\newcommand{\descapp}[6]{
	#1: #2 &\rightarrow #3\\
	#4 &\mapsto #5#6 
}
\newcommand{\arbtfam}[3]{
	{\left\{{#1}_{#2}\right\}}_{#2\in #3}
}
\newcommand{\arbtfmnsub}[3]{
	{\left\{{#1}\right\}}_{#2\in #3}
}
\newcommand{\fntfmnsub}[3]{
	{\left\{{#1}\right\}}_{#2=1}^{#3}
}
\newcommand{\fntfam}[3]{
	{\left\{{#1}_{#2}\right\}}_{#2=1}^{#3}
}
\newcommand{\fntfamsup}[4]{
	\lrbrack{{#1}^{#2}}_{#3=1}^{#4}
}
\newcommand{\arbtuple}[3]{
	{\left({#1}_{#2}\right)}_{#2\in #3}
}
\newcommand{\fntuple}[3]{
	{\left({#1}_{#2}\right)}_{#2=1}^{#3}
}
\newcommand{\gengroup}[1]{
	\left< #1\right>
}
\newcommand{\stblzer}[2]{
	St_{#1}\lrprth{#2}
}
\newcommand{\cmmttr}[1]{
	\left[#1,#1\right]
}
\newcommand{\grpindx}[2]{
	\left[#1:#2\right]
}
\newcommand{\syl}[2]{
	Syl_{#1}\lrprth{#2}
}
\newcommand{\grtcd}[2]{
	mcd\lrprth{#1,#2}
}
\newcommand{\lsttcm}[2]{
	mcm\lrprth{#1,#2}
}
\newcommand{\amntpSyl}[2]{
	\mu_{#1}\lrprth{#2}
}
\newcommand{\gen}[1]{
	gen\lrprth{#1}
}
\newcommand{\ringcenter}[1]{
	C\lrprth{#1}
}
\newcommand{\zend}[2]{
	End_{\mathbb{Z}}^{#2}\lrprth{#1}
}
\newcommand{\genmod}[2]{
	\left< #1\right>_{#2}
}
\newcommand{\genlin}[1]{
	\mathscr{L}\lrprth{#1}
}
\newcommand{\opst}[1]{
	{#1}^{op}
}
\newcommand{\ringmod}[3]{
	\if#3l
	{}_{#1}#2
	\else
	\if#3r
	#2_{#1}
	\fi
	\fi
}
\newcommand{\ringbimod}[4]{
	\if#4l
	{}_{#1-#2}#3
	\else
	\if#4r
	#3_{#1-#2}
	\else 
	\ifstrequal{#4}{lr}{
		{}_{#1}#3_{#2}
	}
	\fi
	\fi
}
\newcommand{\ringmodhom}[3]{
	Hom_{#1}\lrprth{#2,#3}
}

\ExplSyntaxOn

\NewDocumentCommand{\functor}{O{}m}
{
	\group_begin:
	\keys_set:nn {nicolas/functor}{#2}
	\nicolas_functor:n {#1}
	\group_end:
}

\keys_define:nn {nicolas/functor}
{
	name     .tl_set:N = \l_nicolas_functor_name_tl,
	dom   .tl_set:N = \l_nicolas_functor_dom_tl,
	codom .tl_set:N = \l_nicolas_functor_codom_tl,
	arrow      .tl_set:N = \l_nicolas_functor_arrow_tl,
	source   .tl_set:N = \l_nicolas_functor_source_tl,
	target   .tl_set:N = \l_nicolas_functor_target_tl,
	Farrow      .tl_set:N = \l_nicolas_functor_Farrow_tl,
	Fsource   .tl_set:N = \l_nicolas_functor_Fsource_tl,
	Ftarget   .tl_set:N = \l_nicolas_functor_Ftarget_tl,	
	delimiter .tl_set:N= \_nicolas_functor_delimiter_tl,	
}

\dim_new:N \g_nicolas_functor_space_dim

\cs_new:Nn \nicolas_functor:n
{
	\begin{tikzcd}[ampersand~replacement=\&,#1]
		\dim_gset:Nn \g_nicolas_functor_space_dim {\pgfmatrixrowsep}		
		\l_nicolas_functor_dom_tl
		\arrow[r,"\l_nicolas_functor_name_tl"] \&
		\l_nicolas_functor_codom_tl
		\\[\dim_eval:n {1ex-\g_nicolas_functor_space_dim}]
		\l_nicolas_functor_source_tl
		\xrightarrow{\l_nicolas_functor_arrow_tl}
		\l_nicolas_functor_target_tl
		\arrow[r,mapsto] \&
		\l_nicolas_functor_Fsource_tl
		\xrightarrow{\l_nicolas_functor_Farrow_tl}
		\l_nicolas_functor_Ftarget_tl
		\_nicolas_functor_delimiter_tl
	\end{tikzcd}
}
\ExplSyntaxOff

\theoremstyle{definition}
\newtheorem{define}{Definición}
\newtheorem{lem}{Lema}
\newtheorem{teo}{Teorema}
\title{Lista 5}
\author{Arruti, Sergio, Jesús}
\date{}


\begin{document}
	\maketitle
	\begin{enumerate}[label=\textbf{Ej \arabic*.}]
		\setcounter{enumi}{78}
		%%%%% Ej.79 %%%%%
		\item
		%%%% Ej.80 %%%%%
		\item
		%%%% Ej.81 %%%%%
		\item Sean $R$ un anillo conmutativo artiniano, $\mathcal{C}$ una categoría tal que $Obj\lrprth{\mathcal{C}}=\lrprth{*}$ y $\circ$ la composición en $Hom\lrprth{\mathcal{C}}$. Entonces
		\begin{enumerate}
			\item $\mathcal{C}$ es una $R$-categoría $\iff$ $End_{\mathcal{C}}\lrbrack{*}$, con $\circ$ como producto, es una $R$-álgebra;
			\item $\mathcal{C}$ es una $R$-categoría $Hom$-finita $\iff$ $End_{\mathcal{C}}\lrbrack{*}$, con $\circ$ como producto, $End_{\mathcal{C}}\lrbrack{*}$, con $\circ$ como producto, es una $R$-álgebra de Artin.
		\end{enumerate}
		\begin{proof}
			Sea $S:=End_{\mathcal{C}}\lrbrack{*}$.\\
			\boxed{a)\implies} Dado que $Obj\lrprth{\mathcal{C}}=\lrprth{*}$ entonces el que $\mathcal{C}$ sea una $R$-categoría es equivalente a:
			\begin{enumerate}[label=\roman*)]
				\item $S\in Mod\lrprth{R}$,
				\item La operación $\circ$ es $R$-bilineal en $S$.
			\end{enumerate} 
			Notemos que de i) se sigue que $\exists\ \bullet:R\times S\to S$ una acción que hace de $S$ un $R$-módulo, y así en partícular existe una operación $+$ tal que $\lrprth{End_{\mathcal{C}}\lrbrack{*},+}$ es un grupo abeliano. De ii) se sigue que en partícular $\circ$ se distribuye con respecto a $+$ (es $\mathbb{Z}$-bilinieal). Dado que $S$ posee identidad con respecto a $\circ$, $Id_{\lrbrack{*}}$, y $\circ$ es asociativa se sigue que $S$ posee estructura de anillo con $+$ como suma y $\circ$ como producto.\\
			Notemos que ii) también garantiza que si $r\in R$ y $f,g\in S$, entonces
			\begin{align*}
				\lrprth{r\bullet f}\circ g=r\bullet\lrprth{f\circ g}=f\circ\lrprth{r\bullet g},
			\end{align*}
			de modo que $\bullet$ es una acción compatible del anillo conmutativo $R$ sobre $S$. Luego, por el Ej. 4, $\bullet$ permite inducir un morfismo de anillos 
			\begin{align*}
				\descapp{\varphi_\bullet}{R}{S}{r}{r\bullet Id_{*}}{}
			\end{align*}
			 por medio del cual $\lrprth{S,+,\circ}$ es una $R$-álgebra.\\
			\boxed{a)\impliedby} Supongamos que $S$, con $\circ$ como producto, es una $R$-álgebra por medio del morfismo $\varphi$. Entonces necesariamente $\exists$  $+$ operación en $S$ tal que $S:=\lrprth{End_{\mathcal{C}}\lrbrack{*},+,\circ}$ es un anillo y, por el Ej. 3, $\varphi$ induce una acción compatible del anillo conmutativo $R$ en $S$, $\bullet_\varphi$. Notemos que las propiedades de las acciones compatibles garantizan que por medio de $\bullet_\varphi$ $S\in Mod\lrprth{R}$ y que, si $r\in R$ y $f,g,h\in S$
			\begin{align*}
				\lrprth{r\bullet_\varphi f+g}\circ h&=\lrprth{r\bullet_\varphi f}\circ h+g\circ h=r\bullet_\varphi\lrprth{f\circ h}+ g\circ h,\\
				f\circ\lrprth{r\bullet_\varphi g+ h}&=f\circ\lrprth{r\bullet_\varphi g}+ f\circ h=r\bullet_\varphi\lrprth{f\circ h}+ f\circ h.
			\end{align*}
			Con lo cual se satisfacen las condiciones i) y ii) enunciadas en la demostración de la necesidad, y por lo tanto $\mathcal{C}$ es una $R$-categoría.
			\begin{obs}
				Notemos que tanto en la necesidad como en la suficiencia de lo previamente demostrado $S$ posee una estructura de anillo, por medio de la cual puede obtener una estructura natural de $S$-módulo. Más aún, por el Ej. 5, la estructura que posee $S$ cómo $R$-módulo  coincide con aquella que se puede obtener aplicando un cambio de anillos $\gamma:R\to S$ a $\ringmod{S}{S}{l}$  ($\gamma=\varphi_\bullet$ en la necesidad y $\gamma=\varphi$ en la suficiencia).
			\end{obs}
			\boxed{b)} Como $R$ es artiniano, por el Teorema 2.7.15$a)$ se tiene que
			\begin{align*}
				S\in f.l.\lrprth{R}\iff S&\in mod\lrprth{R}.
			\end{align*}
			De lo anterior, el inciso $a)$ y la Observación, se tiene lo deseado.\\			
		\end{proof}
		%%%% Ej.82 %%%%%
		\item
		%%%% Ej.83 %%%%%
		\item
		%%%% Ej.84 %%%%%
		\item Sean $R$ un anillo y \begin{equation*}\tag{*}\label{exact}
			\shortseq{A=M_1,B=M_0,C=M, AtoB=f, BtoC=g,lcr=r,}
		\end{equation*} una sucesión exacta en $Mod(R)$. Entonces, $\forall\ N\in Mod\lrprth{R}$
		\begin{equation*}\tag{**}\label{cfexact}
			\shortseq{A=\functhom{M}{N}{R},B=\functhom{M_0}{N}{R},C=\functhom{M_1}{N}{R}, AtoB=\lrprth{g,N}, BtoC=\lrprth{f,N},lcr=l,}
		\end{equation*}
		es una sucesión exacta de grupos abelianos.
		\begin{proof}
			Sea $N\in Mod\lrprth{R}$. Notemos que (\ref{cfexact}) se obtiene de aplicar el funtor contravariante $F_N:=\functhom{}{N}{R}$ a (\ref{exact}). Bajo esta notación se tiene $(g,N)=F_N\lrprth{g}$ y $\lrprth{f,N}=F_N\lrprth{f}$. Como $Mod\lrprth{R}$ es una categoría preaditiva entonces (\ref{cfexact}) es una sucesión en $Ab$ (ver Ej. 60) y por tanto únicamente resta verificar que 
			\begin{enumerate}
				\item $F_N\lrprth{g}$ es un monomorfismo (de grupos abelianos),
				\item $Im\lrprth{F_N\lrprth{g}}=Ker\lrprth{F_N\lrprth{f}}$.
			\end{enumerate}
			\boxed{a)} Sean $\alpha,\beta\in F_N\lrprth{M}$ tales que $F_N\lrprth{g}\lrprth{\alpha}=F_N\lrprth{g}\lrprth{\beta}$, entonces 
			$\alpha g=\beta g$. Dado que $g$ es en partícular sobre por ser (\ref{exact}) exacta, entonces esta es invertible por la derecha, lo cual aplicado a la igualdad anterior garantiza que $\alpha=\beta$. \\
			\boxed{b)} Sea $\alpha\in F_N\lrprth{M}$, entonces 
			\begin{align*}
				F_N\lrprth{f}\circ F_N\lrprth{g}\lrprth{\alpha}&=\alpha\circ\lrprth{gf}\\
				&=\alpha\circ\lrprth{0}, && (\ref{exact})\text{ es exacta}\\
				&=0,\\
				\implies F_N\lrprth{f}\circ F_N\lrprth{g}&=0,\\
				\implies Im\lrprth{F_N\lrprth{g}}&\subseteq Ker\lrprth{F_N\lrprth{f}}.
			\end{align*}
			Verifiquemos ahora que $Ker\lrprth{F_N\lrprth{f}}\subseteq Im\lrprth{F_N\lrprth{g}}$, esto es que si $\nu\in F_N\lrprth{M_0}$ es tal que $F_N\lrprth{f}\lrprth{\nu}=0$, entonces $\exists\ \mu\in F_N\lrprth{M}$ tal que $\nu=F_N\lrprth{g}\lrprth{\mu}$. Lo anterior es equivalente a verificar que si $\nu\in\functhom{M_0}{N}{R}$ es tal que \begin{equation*}\tag{I}\label{kernelement}
				\nu f=0,
			\end{equation*} entonces el siguiente diagrama conmuta
			\begin{center}
				\begin{tikzcd}
					& N\\
					M_0\ar{ur}{\nu}\ar{r}[swap]{g}&M\ar[dashrightarrow]{u}[swap]{\exists\ \mu}
				\end{tikzcd}
			\end{center}			
			Notemos primeramente que si $a,b\in M$ son tales que $a-b\in Im\lrprth{f}$, entonces $\exists\ c\in M_1$ tal que
			\begin{align*}
				a-b&=f\lrprth{c}\\
				\implies \nu\lrprth{a-b}&=\nu f \lrprth{c}=0, && (\text{\ref{kernelement}})\\
				\implies \nu\lrprth{a}&=\nu\lrprth{b}.
			\end{align*} 
			Por lo tanto la correspondencia
			\begin{align*}
				\descapp{\overline{\nu}}{\faktor{M_0}{Im\lrprth{f}}}{M}{a+Im\lrprth{f}}{\nu\lrprth{a}}{}
			\end{align*}
			es una función bien definida y más aún es un morfismo de $R$-módulos, pues $\nu$ lo es.\\
			Por otra parte, dado que $g$ es epi en $Mod(R)$ por el Primer Teorema de Isomorfisomos para $R$-módulos se tiene que la función
			\begin{align*}
				\descapp{\overline{g}}{\faktor{M_0}{Ker\lrprth{g}}}{M}{x+Ker\lrprth{g}}{g(x)}{}
				\intertext{es un isomorfismo en $Mod(R)$, con inversa}
				\descapp{\overline{g}^{-1}}{M}{\faktor{M_0}{Ker\lrprth{g}}}{g(x)}{x+Ker\lrprth{g}}{.}
			\end{align*} 
			Dado que $Im\lrprth{f}=Ker\lrprth{g}$ por ser (\ref{exact}) exacta, se tiene que 
			\begin{align*}
				\faktor{M_0}{Ker\lrprth{g}}&=\faktor{M_0}{Im\lrprth{f}}
				\intertext{y así si $m\in M_0$ y $\mu:=\overline{\nu}\circ\overline{g}^{-1}$, entonces}
				\mu g\lrprth{m}&=\overline{\nu}\lrprth{m+Ker\lrprth{g}}=\overline{\nu}\lrprth{m+Im\lrprth{f}}\\
				&=\nu\lrprth{m}.\\
				\therefore\ \mu g=\nu.
			\end{align*}
		\end{proof}
		%%%% Ej.85 %%%%%
		\item
		%%%% Ej.86 %%%%%
		\item
		%%%% Ej.87 %%%%%
		\item Sean $f:M\to I$ y $f':M\to I'$ cubiertas inyectivas de $M\in Nod\lrprth{R}$. Entonces $\exists\ t:I\overset{\sim}{\to} I'$ en $Mod\lrprth{R}$ tal que $tf=f'$.
		\begin{proof}
			Se tiene el siguiente esquema
			\begin{center}
				\begin{tikzcd}					
					M\ar{d}[swap]{f'}\ar{r}{f}&I\ar[dashrightarrow]{dl}{\exists\ t}\\
					I'& 
				\end{tikzcd}
			\end{center}
			con $I'$ inyectivo y $f$, en partícular, un monomorfismo en $Mod(R)$ por ser un mono-esencial. Por lo tanto $\exists\ t\in\functhom{I}{I'}{R}$ tal que \begin{equation*}
				tf=f'.
			\end{equation*} 
			Como $f$ es un mono-esencial y $f'$ es en  partícular un monomorfismo en $Mod(R)$, de la igualdad anterior se sigue que $t$ es un monomorfismo en $Mod(R)$. Con lo cual, si $\pi$ es el epi canónico de $I'$ en $\faktor{I'}{Im\lrprth{t}}$, la sucesión
			\begin{center}
				\shortseq{
					A=I, B=I', C=\faktor{I'}{Im\lrprth{t}}, AtoB=t, BtoC=\pi,
				}
			\end{center}
			es exacta. De modo que es una sucesión exacta que se parte, puesto que $I$ es inyectivo (ver Ej. 65), con lo cual $t$ es un split-mono (ver Ej. 54) i.e. $\exists$ $t'\in\functhom{I'}{I}{R}$ tal que $t't=Id_I$. La igualdad anterior garantiza que $j$ es un split-epi. Además
			\begin{align*}
				tf=f'&\implies f=t'f',
			\end{align*}
			con lo cual $t'$ es un monomorfismo, pues $f$ lo es y $f'$ es un mono-esencial. Así $t'$ es un isomorfismo en $Mod(R)$ y por lo tanto $t=\lrprth{t'}^{-1}$ también lo es.\\
		\end{proof}		
		%%%% Ej.88 %%%%%
		\item
		%%%% Ej.89 %%%%%
		\item
		%%%% Ej.90 %%%%%
		\item Sean $R$ un anillo y $M\in Mod(R)$. Entonces
		\begin{enumerate}
			\item $M$ es simple $\implies M$ es irreducible $\implies M$ es indescomponible;
			\item $\mathbb{\ringmod{\mathbb{Z}}{\mathbb{Z}}{l}}$ es irreducible pero no simple;
			\item $M$ es irreducible $\implies$ $Soc\lrprth{M}=\gengroup{0}$ ó $Soc\lrprth{M}$ es simple;
			\item $M$ es semisimple $\iff$ $Soc\lrprth{M}=M$;
			\item $Soc\lrprth{Soc\lrprth{M}}=Soc\lrprth{M}$.
		\end{enumerate}
		\begin{proof}
			\boxed{a)} Supongamos que $M$ es simple. Entonces $M\neq\gengroup{0}$ y $\genlin{M}\setminus\lrbrack{\gengroup{0}}=\lrbrack{M}$, y, dado que la inclusión $i$ de $M$ en sí mismo es $Id_M$, así se tiene que si $X\in Mod(R)$ y $f\in\functhom{M}{X}{R}$, entonces
			\begin{align*}
				f\circ i\text{ es monomorfismo } \iff				f\text{ es monomorfismo }.
			\end{align*}
			i.e. $i$ es un mono-esencial y por lo tanto $M$ es irreducible.\\
			
			Supongamos ahora que $M$ es irreducible.  Sean $M_1, M_2\in\genlin{M}$ tales que $M=M_1\oplus M_2$ y supongamos, sin pérdida de generalidad que $M_1\neq\gengroup{0}$. Como $M_1$ es un sumando directo de $M$ entonces la inclusión $i$ de $M_1$ en $M$ es un split-mono (ver el Teorema 1.12.5), es decir, $\exists\ j\in\functhom{M}{M_1}{R}$ tal que \begin{equation*}\tag{*}\label{jessplitepi}
				ji=Id_{M_1}.
			\end{equation*} Como $i$ es un mono-esencial, por ser $M$ irreducible, y $Id_{M_1}$ es un monomorfismo, entonces $j$ es un monomorfismo y, por (\ref{jessplitepi}), un split-epi. De modo que $j$ es en partícular biyectiva y por lo tanto $i=j^{-1}$ también lo es. Así $M_1=M$ y
			\begin{align*}
				M_2&=M\cap M_2=M_1\cap M_2=\gengroup{0}.\\
				&\therefore\ M\text{ es indescomponible.}
			\end{align*}
		\boxed{b)} Sea $M:=\ringmod{\mathbb{Z}}{\mathbb{Z}}{l}$. Dado que la estructura que posee $M$ como $\mathbb{Z}$-módulo viene dada por su multiplicación, la cual es conmutativa, entonces 
		\begin{align*}
			\genlin{M}&=\lrbrack{I\subseteq \mathbb{Z}\ \vline\ I\subnormeq \mathbb{Z}}\\
			&=\lrbrack{n\mathbb{Z}\ \vline\ n\in\mathbb{N}}.
			\intertext{Así}
			\genlin{M}\setminus\lrbrack{\gengroup{0}}&=\lrbrack{n\mathbb{Z}\ \vline\ n\in\mathbb{N}\setminus\lrbrack{1}}.
		\end{align*}
		Sean $n,m\in\mathbb{N}$ con $n\neq 1$, $i$ la inclusión de $n\mathbb{Z}$ en $\mathbb{Z}$. Supongamos que $i^{-1}\lrprth{m\mathbb{Z}}=\gengroup{0}$, entonces
		\begin{align*}
			\gengroup{0}&=i^{-1}\lrprth{m\mathbb{Z}}=\lrbrack{a\in n\mathbb{Z}\ \vline\ i(a)\in m\mathbb{Z}}=n\mathbb{Z}\cap m\mathbb{Z}\\
			&=mcm\lrprth{n,m}\mathbb{Z},\\
			\implies 0&=mcm\lrprth{n,m}\\
			\implies m=0, && n\neq 0
		\end{align*}
		Así $m\mathbb{Z}=\gengroup{0}$, de modo que por la Proposición 3.3.1 $i$ es un mono-esencial. Por lo tanto $M$ es irreducible.\\
		Finalmente si $n\in\mathbb{N}\setminus\lrbrack{0,1}$, entonces $\gengroup{0}\subsetneq n\mathbb{Z}\subsetneq \mathbb{Z}$ y por tanto $M$ no es simple.\\
		
		\boxed{c} Supongamos que $Soc\lrprth{M}\neq\gengroup{0}$, entonces $Simp\lrprth{M}\neq\varnothing$ y así sea $N\in\genlin{M}\neq\lrbrack{\gengroup{0}}$ simple. Como  $N\subseteq Soc\lrprth{M}$, pues $Soc\lrprth{M}$ está generado por $\bigcup Simp\lrprth{M}$, entonces si $i_N$ es la inclusión de $N$ en $M$, ${i_N}'$ la de $N$ en $Soc\lrprth{M}$ e $i_{\socle{M}}$ la de $Soc\lrprth{M}$ en $M$, se tiene la siguiente sucesión
		\begin{equation*}
			\shortseq{A=N, B=\socle{M},C=M, AtoB={i_N}', BtoC=i_{\socle{M}},lcr=c,},
		\end{equation*} con $i_N=i_{\socle{M}}{i_N}'$. Por lo anterior, como $i_N$ es un mono-esencial por ser $M$ irreducible y todas las inclusiones antes mencionadas son monomorfismos, aplicando la Proposición 3.3.3 $a)$ se tiene que, en partícular, ${i_N}'$ es un mono-esencial. Así, dado que $\socle{M}$ es semisimple, de la Proposición 3.3.3 $b)$ se tiene que ${i_N}'$ es un isomorfismo. De modo que $\socle{M}=N$ y por lo tanto es simple.\\
		
		\boxed{c)} Se tiene que $\socle{M}$ es semisimple, de lo cual se sigue la sucificiencia. Más aún se tiene que $\socle{M}$ es el máximo submódulo semisimple de $M$ (ver la Proposición 3.3.6$a)$), de modo que $M\leq \socle{M}$ si $M$ es semisimple y así se verifica la necesidad. \\
		
		\boxed{e)} Se sigue de aplicar el inciso anterior al modulo semisimple $M':=\socle{M}$.\\
		\end{proof}
		%%%% Ej.91 %%%%%
		\item 
		%%%% Ej.92 %%%%%
		\item
		%%%% Ej.93 %%%%%
		\item Sean $R$ un anillo artiniano a izquierda y $M\in Mod\lrprth{R}$. Si $M\neq 0$ es no trivial, entonces $Soc(M)\neq 0$ . 
		\begin{proof}
			Basta con verificar que $Simp\lrprth{M}\neq\varnothing$. Sea $0\neq m\in M$, luego $0\neq\gengroup{m}\in mod\lrprth{R}=f.l.\lrprth{R}$ (pues $R$ es artiniano a izquierda) y por lo tanto $\gengroup{m}$ es en partícular artiniano (ver la Proposición 2.1.4). Así, por el Ej. 43, $\lrprth{\genlin{\gengroup{m}},\leq}$ posee por lo menos un elemento mínimal, digamos $S$. La minimalidad de $S$ con respecto a $\leq$ junto al hecho de que $\genlin{S}\subseteq\genlin{\gengroup{m}}$ garantizan que $S$ es un submódulo simple de $M$.\\
		\end{proof}
		%%%% Ej.94 %%%%%
		\item
		%%%% Ej.95 %%%%%
		\item
		%%%% Ej.96 %%%%%
		\item Sean $R$ un anillo, $I\subnormeq R$, $\pi:R\to\faktor{R}{I}$ el epi-canónico de anillos y $M\in Mod\lrprth{\faktor{R}{I}}$. Se tiene que
		\begin{enumerate}
			\item $\pi\lrprth{ann_R\lrprth{M}}=ann_{\faktor{R}{I}}\lrprth{M}$;
			\item $\ringmod{\faktor{R}{I}}{M}{l}$ es fiel $\iff I=ann_R \lrprth{M}$.
		\end{enumerate}
		\begin{proof}
			Consideraremos la estructura de $M$ como $R$-módulo como aquella obtenida a partir del cambio de anillos dado por $\pi$.\\
			\boxed{a)} Notemos que
			 \begin{align*}
				a\in\pi\lrprth{ann_R\lrprth{M}}&\iff  a=r+I, r\in ann_{R}\lrprth{M}\\
				&\iff  a=r+I, rm=0\ \forall\ m\in M\\
				&\iff  a=r+I, \lrprth{r+I}m=\pi\lrprth{r}m=0\ \forall\ m\in M\\
				&\iff  a\in ann_{\faktor{R}{I}}.
			\end{align*}
			De lo anterior se tiene lo deseado.\\
			
			\boxed{b)} Notemos primeramente que, si $r\in I$ y $m\in M$, entonces $rm=\lrprth{r+I}m=\lrprth{I}m=0$, pues $I$ es el neutro aditivo de $\faktor{R}{I}$. Así \begin{equation*}\tag{*}\label{IcontAnn}
				I\subseteq ann_R\lrprth{M}.
			\end{equation*}
			Ahora
			\begin{align*}
				\ringmod{\faktor{R}{I}}{M}{l}\text{ es fiel }&\iff ann_{\faktor{R}{I}}\lrprth{M}=\gengroup{I}\\
				&\iff \pi\lrprth{ann_R\lrprth{M}}=\gengroup{I}, && a)\\
				&\iff ann_R\lrprth{M}\subseteq Ker\lrprth{\pi}=I\\
				&\iff ann_R\lrprth{M}=I. && \lrprth{\text{\ref{IcontAnn}}}
			\end{align*}
		\end{proof}
		%%%% Ej.97 %%%%%
		\item
		%%%% Ej.98 %%%%%
		\item
		%%%% Ej.99 %%%%%
		\item Sean $\Lambda$ una $R$-álgebra de Artin, $D_{\Lambda}=\functhom{}{I}{R}:mod\lrprth{\lambda}\to mod\lrprth{\opst{\lambda}}$ la dualidad usual, $M\in mod\lrprth{\Lambda}$ y $N\in mod\lrprth{\opst{\Lambda}}$. Entonces
		\begin{enumerate}
			\item  $\opst{\lrprth{ann_{\Lambda}\lrprth{M}}}=ann_{\opst{\Lambda}}\lrprth{D_{\Lambda}\lrprth{M}}$;
			\item $\opst{ann_{\Lambda}\lrprth{D_{\opst{\Lambda}}\lrprth{N}}}=ann_{\opst{\Lambda}}\lrprth{N}$.
		\end{enumerate}
		\begin{proof}
			Recordemos que $D_{\Lambda}\lrprth{M}\in mod\lrprth{\opst{\Lambda}}$ via la acción $\opst{\lambda}\bullet f$, con $\opst{\lambda}\bullet f\lrprth{m}:=f\lrprth{\lambda m}$ y $\lambda m$ dado por la acción de $\lambda$ sobre $M$. Sea $\lambda\in\Lambda$, entonces
			\begin{align*}
				\lambda\in ann_{\Lambda}\lrprth{M}&\implies \lambda m=0, \forall\ m\in M\\
				&\implies f\lrprth{\lambda m}=f\lrprth{0}=0,\forall\ m\in M, \forall\ f\in D_{\Lambda}\lrprth{M}\\\
				&\implies \opst{\lambda}\bullet f=0, \forall\ f\in D_{\Lambda}\lrprth{M}\\
				&\implies \lambda\in \opst{\lrprth{ann_{\opst{\Lambda}}\lrprth{D_{\Lambda}\lrprth{M}}}}\\
				&\implies ann_{\Lambda}\lrprth{M}\subseteq \opst{\lrprth{ann_{\opst{\Lambda}}\lrprth{D_{\Lambda}\lrprth{M}}}},\tag{*}\label{contarbt}\\
				&\therefore\ \opst{\lrprth{ann_{\Lambda}\lrprth{M}}}\subseteq ann_{\opst{\Lambda}}\lrprth{D_{\Lambda}\lrprth{M}}.
			\end{align*}
			Dado que $\Lambda$ es un álgebra de Artin arbitraria y $M$ es un $\Lambda$-módulo arbitrario, le contención (\ref{contarbt}) es válida para $\opst{\Lambda}$ y $D_{\Lambda}\lrprth{M}\in mod\lrprth{\opst{\Lambda}}$, de modo que
			\begin{align*}
				 ann_{\opst{\Lambda}}\lrprth{D_{\Lambda}\lrprth{M}}&\subseteq \opst{\lrprth{ann_{\Lambda}\lrprth{D_{\opst{\Lambda}}\lrprth{D_{\Lambda}\lrprth{M}}}}}\\
				 &=\opst{ann_{\Lambda}\lrprth{M}}.
			\end{align*}
			\boxed{b)} Se sigue de $a)$ aplicado a $D_{\opst{\Lambda}}\lrprth{N}\in mod\lrprth{\Lambda}$.\\
		\end{proof}
		%%%% Ej.100 %%%%%
		\item
		%%%% Ej.101 %%%%%
		\item
	\end{enumerate}
\end{document}