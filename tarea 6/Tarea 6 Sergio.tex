\documentclass{article}
\usepackage[utf8]{inputenc}
\usepackage{mathrsfs}
\usepackage[spanish,es-lcroman]{babel}
\usepackage{amsthm}
\usepackage{amssymb}
\usepackage{enumitem}
\usepackage{graphicx}
\usepackage{caption}
\usepackage{float}
\usepackage{amsmath,stackengine,scalerel,mathtools}
\usepackage{xparse, tikz-cd, pgfplots}
\usepackage{comment}
\usepackage{faktor}
\usepackage[all]{xy}


\def\subnormeq{\mathrel{\scalerel*{\trianglelefteq}{A}}}
\newcommand{\Z}{\mathbb{Z}}
\newcommand{\La}{\mathscr{L}}
\newcommand{\crdnlty}[1]{
	\left|#1\right|
}
\newcommand{\lrprth}[1]{
	\left(#1\right)
}
\newcommand{\lrbrack}[1]{
	\left\{#1\right\}
}
\newcommand{\lrsqp}[1]{
	\left[#1\right]
}
\newcommand{\descset}[3]{
	\left\{#1\in#2\ \vline\ #3\right\}
}
\newcommand{\descapp}[6]{
	#1: #2 &\rightarrow #3\\
	#4 &\mapsto #5#6 
}
\newcommand{\arbtfam}[3]{
	{\left\{{#1}_{#2}\right\}}_{#2\in #3}
}
\newcommand{\arbtfmnsub}[3]{
	{\left\{{#1}\right\}}_{#2\in #3}
}
\newcommand{\fntfmnsub}[3]{
	{\left\{{#1}\right\}}_{#2=1}^{#3}
}
\newcommand{\fntfam}[3]{
	{\left\{{#1}_{#2}\right\}}_{#2=1}^{#3}
}
\newcommand{\fntfamsup}[4]{
	\lrbrack{{#1}^{#2}}_{#3=1}^{#4}
}
\newcommand{\arbtuple}[3]{
	{\left({#1}_{#2}\right)}_{#2\in #3}
}
\newcommand{\fntuple}[3]{
	{\left({#1}_{#2}\right)}_{#2=1}^{#3}
}
\newcommand{\gengroup}[1]{
	\left< #1\right>
}
\newcommand{\stblzer}[2]{
	St_{#1}\lrprth{#2}
}
\newcommand{\cmmttr}[1]{
	\left[#1,#1\right]
}
\newcommand{\grpindx}[2]{
	\left[#1:#2\right]
}
\newcommand{\syl}[2]{
	Syl_{#1}\lrprth{#2}
}
\newcommand{\grtcd}[2]{
	mcd\lrprth{#1,#2}
}
\newcommand{\lsttcm}[2]{
	mcm\lrprth{#1,#2}
}
\newcommand{\amntpSyl}[2]{
	\mu_{#1}\lrprth{#2}
}
\newcommand{\gen}[1]{
	gen\lrprth{#1}
}
\newcommand{\ringcenter}[1]{
	C\lrprth{#1}
}
\newcommand{\zend}[2]{
	End_{\mathbb{Z}}^{#2}\lrprth{#1}
}
\newcommand{\genmod}[2]{
	\left< #1\right>_{#2}
}
\newcommand{\genlin}[1]{
	\mathscr{L}\lrprth{#1}
}
\newcommand{\opst}[1]{
	{#1}^{op}
}
\newcommand{\ringmod}[3]{
	\if#3l
	{}_{#1}#2
	\else
	\if#3r
	#2_{#1}
	\fi
	\fi
}
\newcommand{\ringbimod}[4]{
	\if#4l
	{}_{#1-#2}#3
	\else
	\if#4r
	#3_{#1-#2}
	\else 
	\ifstrequal{#4}{lr}{
		{}_{#1}#3_{#2}
	}
	\fi
	\fi
}
\newcommand{\ringmodhom}[3]{
	Hom_{#1}\lrprth{#2,#3}
}

\ExplSyntaxOn

\NewDocumentCommand{\functor}{O{}m}
{
	\group_begin:
	\keys_set:nn {nicolas/functor}{#2}
	\nicolas_functor:n {#1}
	\group_end:
}

\keys_define:nn {nicolas/functor}
{
	name     .tl_set:N = \l_nicolas_functor_name_tl,
	dom   .tl_set:N = \l_nicolas_functor_dom_tl,
	codom .tl_set:N = \l_nicolas_functor_codom_tl,
	arrow      .tl_set:N = \l_nicolas_functor_arrow_tl,
	source   .tl_set:N = \l_nicolas_functor_source_tl,
	target   .tl_set:N = \l_nicolas_functor_target_tl,
	Farrow      .tl_set:N = \l_nicolas_functor_Farrow_tl,
	Fsource   .tl_set:N = \l_nicolas_functor_Fsource_tl,
	Ftarget   .tl_set:N = \l_nicolas_functor_Ftarget_tl,	
	delimiter .tl_set:N= \l_nicolas_functor_delimiter_tl,	
}

\dim_new:N \g_nicolas_functor_space_dim

\cs_new:Nn \nicolas_functor:n
{
	\begin{tikzcd}[ampersand~replacement=\&,#1]
		\dim_gset:Nn \g_nicolas_functor_space_dim {\pgfmatrixrowsep}		
		\l_nicolas_functor_dom_tl
		\arrow[r,"\l_nicolas_functor_name_tl"] \&
		\l_nicolas_functor_codom_tl
		\tl_if_blank:VF \l_nicolas_functor_source_tl {
			\\[\dim_eval:n {1ex-\g_nicolas_functor_space_dim}]
			\l_nicolas_functor_source_tl
			\xrightarrow{\l_nicolas_functor_arrow_tl}
			\l_nicolas_functor_target_tl
			\arrow[r,mapsto] \&
			\l_nicolas_functor_Fsource_tl
			\xrightarrow{\l_nicolas_functor_Farrow_tl}
			\l_nicolas_functor_Ftarget_tl
			\l_nicolas_functor_delimiter_tl
		}
	\end{tikzcd}
}
\ExplSyntaxOff

\ExplSyntaxOn

\NewDocumentCommand{\shortseq}{O{}m}
{
	\group_begin:
	\keys_set:nn {nicolas/shortseq}{#2}
	\nicolas_shortseq:n {#1}
	\group_end:
}

\keys_define:nn {nicolas/shortseq}
{
	A     .tl_set:N = \l_nicolas_shortseq_A_tl,
	B   .tl_set:N = \l_nicolas_shortseq_B_tl,
	C .tl_set:N = \l_nicolas_shortseq_C_tl,
	AtoB      .tl_set:N = \l_nicolas_shortseq_AtoB_tl,
	BtoC   .tl_set:N = \l_nicolas_shortseq_BtoC_tl,	
	lcr   .tl_set:N = \l_nicolas_shortseq_lcr_tl,	
	
	A		.initial:n =A,
	B		.initial:n =B,
	C		.initial:n =C,
	AtoB    .initial:n =,
	BtoC   	.initial:n=,
	lcr   	.initial:n=lr,
	
}

\cs_new:Nn \nicolas_shortseq:n
{
	\begin{tikzcd}[ampersand~replacement=\&,#1]
		\IfSubStr{\l_nicolas_shortseq_lcr_tl}{l}{0 \arrow{r} \&}{}
		\l_nicolas_shortseq_A_tl
		\arrow{r}{\l_nicolas_shortseq_AtoB_tl} \&
		\l_nicolas_shortseq_B_tl
		\arrow[r, "\l_nicolas_shortseq_BtoC_tl"] \&
		\l_nicolas_shortseq_C_tl
		\IfSubStr{\l_nicolas_shortseq_lcr_tl}{r}{ \arrow{r} \& 0}{}
	\end{tikzcd}
}

\ExplSyntaxOff
\newcommand{\limseq}[2]{
	\lim_{#2\to\infty}#1
}

\newcommand{\norm}[1]{
	\crdnlty{\crdnlty{#1}}
}

\newcommand{\inter}[1]{
	int\lrprth{#1}
}
\newcommand{\cerrad}[1]{
	cl\lrprth{#1}
}

\newcommand{\restrict}[2]{
	\left.#1\right|_{#2}
}
\newcommand{\functhom}[3]{
	\ifblank{#1}{
		Hom_{#3}\lrprth{-,#2}
	}{
		\ifblank{#2}{
			Hom_{#3}\lrprth{#1,-}
		}{
			Hom_{#3}\lrprth{#1,#2}	
		}
	}
}
\newcommand{\socle}[1]{
	Soc\lrprth{#1}
}

\theoremstyle{definition}
\newtheorem{define}{Definición}
\newtheorem{lem}{Lema}
\newtheorem{teo}{Teorema}
\newtheorem*{teosn}{Teorema}
\newtheorem*{obs}{Observación}
\title{Lista 4}
\author{Arruti, Sergio, Jesús}
\date{}


\begin{document}
	\maketitle
	\begin{enumerate}[label=\textbf{Ej \arabic*.}]
		\setcounter{enumi}{78}
\item
\item Sea $\Lambda$ una $R$-Álgebra de Artín y 
\begin{tikzcd}
0\arrow{r}{} & A\arrow{r}{f} &B\arrow{r}{g}& C\arrow{r}{}& 0
\end{tikzcd}

una sucesión exacta en $Mod(\Lambda)$ (respectivamente en $mod(\Lambda)$). Pruebe que $\forall X\in Mod(\Lambda)$ (respectivamente 
$\forall X\in mod(\Lambda)$), se tienen las siguientes sucesiones exactas en $Mod(R)$ (respectivamente en $mod(R)$).

a)
\[\begin{tikzcd}
0\arrow{r}{} & \operatorname{Hom}_\Lambda(X,A)\arrow{r}{f_*} & \operatorname{Hom}_\Lambda(X,B)\arrow{r}{g_*}&
 \operatorname{Hom}_\Lambda(X,C)\arrow{r}{}& 0.
\end{tikzcd}
\]
b)
\[\begin{tikzcd}
0\arrow{r}{} & \operatorname{Hom}_\Lambda(C,X)\arrow{r}{g^*} & \operatorname{Hom}_\Lambda(B,X)\arrow{r}{f^*}&
 \operatorname{Hom}_\Lambda(A,X)\arrow{r}{}& 0.
\end{tikzcd}
\]
donde 
\begin{align*}
f_*= \operatorname{Hom}_\Lambda(X,f)\,,\quad
f^*= \operatorname{Hom}_\Lambda(f,X)\\
g_*= \operatorname{Hom}_\Lambda(X,g)\quad \text{y}\quad
g^*= \operatorname{Hom}_\Lambda(g,X)
\end{align*}
\begin{proof}
Como $\Lambda$ es una $R$-Álgebra de Artín, entonces por el ejercicio 79 $\Lambda$ es un anillo artiniano, así 
$ \operatorname{Hom}_\Lambda(X,\bullet)$ es un funtor exacto covariante y $ \operatorname{Hom}_\Lambda(\bullet,X)$ es un 
funtor exacto contravariante. Por esto se tiene que las sucesiones a) y b) son exactas en $Mod(\Lambda)$, y por 3.1.1 se tiene que para todo 
$J,K\in Mod(\Lambda),\,\,\, \operatorname{Hom}_\Lambda(J,K)$ es un $R$-submódulo de $ \operatorname{Hom}_R(J,K)$. Así a) y b)
 son sucesiones exactas en $Mod(R)$.\\

Por otra parte si nuestra sucesión es exacta en $mod(\Lambda)$ y $X\in mod(\Lambda)$, por la proposición 3.1.3 y lo anterior, las sucesiones
exactas a) y b) estarán compuestas por $R$-módulos finitamente generados, por lo que a) y b) son sucesiones exactas en $mod(R)$.
\end{proof}
\item
\item
\item Pruebe que para un anillo artiniano a izquierda $R$, se tiene que\\ $mod(\prescript{}{R}{R})=mod(R)$
\begin{proof}
Por definición $mod(\prescript{}{R}{R})$ es la subcategoría plena de $mod(R)$ cuyos objetos son los $A\in mod(R)$ tales que existe una 
sucesión exacta 
\begin{tikzcd}
P_1\arrow{r}{}&P_0\arrow{r}{}&A\arrow{r}{}&0
\end{tikzcd}
en $mod(R)$ con $P_1,P_0\in add(R)$.\\

Como $mod(\prescript{}{R}{R})$ es subcategoría plena de $mod(R)$, basta ver que si\\ $M\in mod(R)$, entonces $M\in mod(\prescript{}{R}{R})$.\\
Sea $M\in mod(R)$ entonces $M=\displaystyle\bigoplus_{m\in A}Rm$ con $A\subset M$ finito, así, considerando $|A|=n$, se tiene
la sucesión exacta 
\[
\begin{tikzcd}
A_1\oplus A_2\oplus M\arrow{r}{\pi_1}& A_2\oplus M\arrow{r}{\pi_2}&M\arrow{r}{}&0.
\end{tikzcd}
\]
Donde $A_1\cong A_2\cong R$ y $\pi_1,\pi_2$ son proyecciones canonicas, en particular $A_1$ y $A_2$ son objetos en $add(R)$ pues 
$A_1\coprod A_2\cong R\coprod R=R^2$, así $M\in mod(\prescript{}{R}{R})$. \\
\end{proof}

\item
\item
\item    ??????


\item
\item
\item Para un anillo $R$, pruebe que la correspondencia\\ $Soc:Mod(R)\longrightarrow Mod(R)$ donde\\
\xymatrix{
X\ar[dd]_f   &\,& Soc(X)   \ar[dd]^{Soc(f):=f|_{Soc(X)}}                \\
         \,  \ar[rr] & \,       &                \\
Y &\,& Soc(Y)
}

es un funtor aditivo que conmuta con productos arbitrarios y preserva monomorfismos.
\begin{proof}
Funtor aditivo:\\
Sean $f,g\in  \operatorname{Hom}_R(X,Y)$ con $X,Y\in Mod(R)$, entonces $f+g\in \operatorname{Hom}_R(X,Y)$ y 
\begin{tikzcd}
F(X\arrow{r}{f+g}&Y)=(Soc(X)\arrow{r}{(f+g)|_{Soc(X)}}& Soc(Y))
\end{tikzcd}
pero \[F(f+g)=(f+g)|_{Soc(X)}=f|_{Soc(X)}+g|_{Soc(X)}=F(f)+F(g),
\]
pues por 3.3.6 b), $f(Soc(X))\subset Soc(Y)$\quad y \quad $g(Soc(X))\subset Soc(Y).$\\

Conmuta con coproductos arbitrarios:\\

Basta mostrar que $\displaystyle\coprod_{i\in A}Soc(M_i)$ es el submódulo simple más grande contenido en $\displaystyle\coprod_{i\in A}M_i$.\\
Supongamos $N$ es semisimple en $\displaystyle\coprod_{i\in A}M_i$, entonces $N=\displaystyle\bigoplus_{j\in F}S_j$ donde $S_k$ es simple en
$\displaystyle\coprod_{i\in A}M_i$ para toda $k\in A$ y $F\neq \emptyset$.\\

Como todo simple en $\displaystyle\coprod_{i\in A}M_i$ es de la forma $\displaystyle\coprod_{i\in A}S_i$ con $S_i\leq M_i$ simple o cero, 
entonces
\[
N=\bigoplus_{i\in F}\coprod_{j\in A}S_{ij}=\coprod_{j\in A}\bigoplus_{i\in F}S_{ij}\subset \coprod_{i\in A}Soc(M_i), 
\]
pues $Soc(M_i)$ es el submódulo semisimple mas grande contenido en $M_i$, por lo tanto 
$Soc(\displaystyle\coprod_{i\in A}M_i)=\displaystyle\coprod_{i\in A}Soc(M_i).$

\end{proof}

\item
\item
\item Pruebe que 
\begin{itemize}
\item[a)] $Soc\left(\prescript{}{\Z}{\Z}\right)=Soc\left(\prescript{}{\Z}{\Q}\right)=0.$
\item[b)] $\faktor{\Z}{m\Z}$ es un $\Z$-módulo simple $\iff$ $m$ es primo.
\item[c)] $Soc\left(\faktor{\Z}{p^n\Z}\right)=\faktor{p^{n-1}\Z}{p^n\Z}\cong \faktor{\Z}{p\Z}$ para todo primo $p$ y $n\geq 0$.
\item[d)] $Soc\left(\faktor{\Z}{n\Z}\right)\cong \faktor{\Z}{(p_1\ldots p_k)\Z}$ donde $n=p_1^{m_1}\ldots p_k^{m_k}$ en la descomposición
en productos de primos con $p_i\neq p_j$ para toda $i\neq j$.
\end{itemize}
\begin{proof}\boxed{a)}\\
Por una parte, como $\prescript{}{\Z}{\Z}$ no tiene submódulos simples, entonces\\ $Soc\left(\prescript{}{\Z}{\Z}\right)=0.$\\
Por otra, Como $\Z$ es mono-escencial en $\Q$ (como se aprecia en el ejercicio anterior) entonces todo módulo $M$ de 
$\prescript{}{\Z}{\Q}$ cumple que $\Z\cap M\neq \emptyset$ y como $\Z$ no es simple, entonces $\prescript{}{\Z}{\Q}$ no tiene
submódulos simples, es decir \\$Soc\left(\prescript{}{\Z}{\Z}\right)=Soc\left(\prescript{}{\Z}{\Q}\right)=0.$\\
\boxed{b)}\\
Como $M$ es submódulo de  $\faktor{\Z}{m\Z}$ si y sólo si $M=k\faktor{\Z}{m\Z}$ donde $k|m$, entonces si $\faktor{\Z}{m\Z}$ es simple
$k$ sólo puede ser $1$ o $m$, es decir, $m$ tiene que ser primo.\\
Si $m$ es primo $\faktor{\Z}{m\Z}$ es campo y por lo tanto simple.\\
\boxed{c)}\\
Sea $p$ primo y $n\geq 2$, entonces $\faktor{\Z}{p\Z}$ es simple en $\faktor{\Z}{p^n\Z}$, sin embrgo es el único simple, pues
si $M\leq \faktor{\Z}{p^n\Z}$ es simple, entonces $M=\faktor{\Z}{p^k\Z}$ y esto pasa sólo si $p^k$  es primo, es decir, si $k=1$.
Por lo tanto \\$Soc\left(\faktor{\Z}{p^n\Z}\right)=\faktor{\Z}{p\Z}$.\\
\boxed{d)}\\
Sea $n=p_1^{m_1}\ldots p_k^{m_k}$ su descomposición en primos.\\
Como $n\Z=p_1^{m_1}\ldots p_k^{m_k}\Z$ entonces $\faktor{\Z}{n\Z}=\faktor{\Z}{p_1^{m_1}\ldots p_k^{m_k}\Z}$, en particular 
$\faktor{\Z}{p_j\Z}\leq \faktor{\Z}{n\Z}$ es simple para toda $j\in \{1,\ldots , k\}$, pues $p_j\Z\geq n\Z$.\\
Por otra parte si $M$ es simple en $\faktor{\Z}{n\Z}$, entonces $M=\faktor{\Z}{p\Z}$ para algún $p$ primo y $p\Z\geq n\Z$, por lo que $p|n$
es decir, existe $j\in \{1,\ldots , k\}$ tal que $p|p_j^{m_j}$, entonces $p=p_j$ y así $M=\faktor{\Z}{p_j\Z}$ para algún  $j\in \{1,\ldots , k\}$.
Por lo tanto, como $p_j\Z\geq n\Z$ para toda $j\in \{1,\ldots , k\}$,
\[ Soc\left(\faktor{\Z}{n\Z}\right)\cong \displaystyle\sum_{i\leq k}
\faktor{\Z}{p_j\Z}=\displaystyle\bigoplus_{j\leq k}\faktor{\Z}{p_j\Z}\cong \faktor{\Z}{(p_1\ldots p_k)\Z}.\]
\end{proof}

\item
\item
\item Para un anillo $R$ y $M\in Mod(R)$, pruebe que 
\begin{itemize}
\item[a)] $ann_R(M)\unlhd R$.
\item[b)] $M$ es un $\left(\faktor{R}{ann_R(M)}\right)$-módulo fiel.
\item[c)] $\forall f\in \operatorname{Hom}_R(R,M)$,\quad $ann_R(M)\leq Ker(f)$.
\item[d)] $\forall N\in Mod(R)$,\quad $N\cong M\Longrightarrow ann_R(M)=ann_R(N)$.
\end{itemize}
\begin{proof}
\boxed{a)}\\
$ann_R(M)=\{r\in R\,|\,r\cdot m=0\,\,\forall m\in M\}$.\\
Sean $r,s\in ann_R(M)$, entonces $(r+s)\cdot m=r\cdot m+s\cdot m=0$ por lo que $(r+s)\in ann_R(M)$.\\

Ahora, si $a\in R$, $(zr)\cdot m=z\cdot(r\cdot m)=z\cdot 0=0$. Por lo tanto $ann_R(M)\unlhd R$.\\
\boxed{b)}\\
$\displaystyle ann_{\faktor{R}{ann_R(M)}}(M)=\{r\in \faktor{R}{ann_R(M)}\,|\, [r]\cdot m=0\}$ con $[r]$ denotando la clase de $r\in R$ bajo la
relación de equivalencia. Ahora, como $[r]\cdot m=0,$ entonces 
\[0=(r+ann_R(M))\cdot m=r\cdot m+0,\]
y así $r\in ann_R(M)$, es decir, $[r]=0$. Por lo tanto $M$ es un \\ $\faktor{R}{ann_R(M)}$-módulo fiel.\\
\boxed{c)}\\
Sean $f\in\operatorname{Hom}_R(R,M)$ y $r\in ann_R(M)$, entonces $r\cdot m=0\,\,\forall m\in M$ así, como $f$ es morfismo
$f(r)=r\cdot f(1)=0$ pues $f(1)\in M$. Por lo tanto $ann_R(M)\leq Ker(f)$.\\
\boxed{d)}\\
Sea $N\in Mod(R)$ tal que existe $h\in \operatorname{Hom}_R(M,N)$ isomorfismo. Entonces para cada $n\in N$ existe un único $m\in M$
tal que $h(m)=n$, así 
\begin{align*}
r\in ann_R(M) &\iff r\cdot m=0\,\,\,\forall m\in M\\
&\iff h(r\cdot m)=0\,\,\,\forall m\in M\\
&\iff r\cdot h(m)=0\,\,\,\forall m\in M\\
&\iff r\cdot n=0\,\,\,\forall n\in N\\
& \iff r\in ann_R(N).
\end{align*}
\end{proof}
\item
\item
\item Sea $\Lambda$ una $R$-álgebra de Artin. Pruebe que, $\forall M\in mod(\Lambda)$ se tiene que: 
\begin{itemize}
\item[a)] $M$ es inescindible $\iff$ $D_\Lambda(M)$ es inescindible.
\item[b)] $M$ es simple $\iff D_\Lambda(M)$ es simple.
\item[c)] $I_0(D_\Lambda(M))\in mod(\Lambda^{op})$.
\item[d)] $I_0(M)\in mod(\Lambda)$.
\item[e)] $l_{\Lambda^{op}}(D_\Lambda(M))=l_\Lambda(M).$
\end{itemize}
\begin{proof}
\boxed{a)}\\
Supongamos $M\neq 0$ es inescindible y supongamos además que\\ $D_\Lambda(M))=D_1\oplus D_2$. Como $D_\Lambda$ y 
$D_{\Lambda^{op}}$ son equivalencias de categorías y $D_\Lambda(M)=D_1\oplus D_2$, entonces 
\[M=D_{\Lambda^{op}}(D_1\oplus D_2)=D_{\Lambda^{op}}(D_1)\oplus D_{\Lambda^{op}}(D_2).\]
Pero $M$ es inescindible, entonces $D_{\Lambda^{op}}(D_1)=0$ o  $D_{\Lambda^{op}}(D_2)=0$, así $D_1=0$ o $D_2=0$, por lo que
$D_\Lambda(M)$ es inescindible.\\

Si $D_\Lambda(M)$ es inescindible y $M=M_1\oplus M_2$, entonces 
\[M=D_{\Lambda^{op}}(M_1\oplus M_2)=D_{\Lambda^{op}}(M_1)\oplus D_{\Lambda^{op}}(M_2),\]
y como $D_{\Lambda}(M)$ es inescindible entonces $D_{\Lambda^{op}}(M_1)=0$ o $D_{\Lambda^{op}}(M_2)=0$ por lo tanto 
$M_1=0$ o $M_2=0$ lo cual implica que $M$  es inescindible.\\
\boxed{b)}\\
Como $D_\Lambda$ y $D_{\Lambda^{op}}$ son equivalencias de categorías, a todo submódulo propio de $K$ de $M$ le corresponde 
un submódulo propio $S$ de $D_\Lambda(M)$, así
\begin{align*}
D_\Lambda\,\,\text{es simple}\,\,&\iff \forall K\lneq D_\Lambda (M)\,\,\,K=0\\
&\iff  S=D_{\Lambda^{op}}(K)\lneq  M\cong D_{\Lambda^{op}}(D_\Lambda (M))\quad K=0=S\\
&\iff \forall S\lneq M\quad S=0\\
&\iff M \text{\,\,\,es simple.}
\end{align*}
\boxed{d)}\\
Como $M\in mod(\Lambda)$ y como $\Lambda$ es un $R$-álgebra de artín, $M=\displaystyle\coprod_{i=1}^nRx_i$ con $x_i\in M$. Entonces,
aplicando 3.3.5, $I_0(M)\cong\displaystyle\coprod_{i=1}^nI_0(Rx_i)$, es decir, $I_0(M)\in mod(\Lambda)$.\\
\boxed{c)}\\
Como $D_\Lambda(M)\in \Lambda^{op}$, entonces aplicando c) en  $D_\Lambda(M)$ se tiene el resultado.\\
\boxed{e}\\
Como $\Lambda$ es una $R$-álgebra de artín, $D_\Lambda(M)$ y $M$ son artinianos y finitamente generados, por lo que ambos son de 
longitud finita. Sea $F$ una serie generalizada de composición de $M$ con longitud mínima, entonces tomaremos por $D_\Lambda(F)$ como
la filtración resultante de aplicar $D_\Lambda$ a cada terimino de $F$. \\
Observamos que $D_\Lambda(F_i)\leq D_\Lambda(F_{i-1})$ para toda $0\leq i\leq l(M)$ y además por ser equivalencia de categorias
$\faktor{D_\Lambda(F_i)}{D_\Lambda\left(F_{i-1}\right)}\cong D_\Lambda\left(\faktor{F_i}{F_{i-1}}\right)$ que es simple por b), así 
$D_\Lambda(F)$ es una serie generalizada de descomposición de longitud $l(M)$.\\

Análogamente $D_{\Lambda^{op}}(G)$ con $G$ una serie generalizada de composición de $D_\Lambda(M)$ define una serie generalizada de 
composición de longitud $l(D_{\Lambda^{op}}(G))$ en $M$, por lo que  $l_{\Lambda^{op}}(D_\Lambda(M))=l_\Lambda(M).$

\end{proof}

\item
\item
\item 






































\end{enumerate}
\end{document}